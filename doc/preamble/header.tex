\usepackage[utf8]{inputenc}   % LaTeX, comprends les accents !
\usepackage[T1]{fontenc}      % Police contenant les caractères français
%\usepackage[french]{babel}  % Placez ici une liste de langues
%\usepackage{multicol}

%%%%%%%%%%%%%%
%% comment uncomment
%\usepackage[notref,notcite]{showkeys}
%%%%


 % \usepackage[notref,notcite]{showkeys}  %  comment out for final version
 % \renewcommand*\showkeyslabelformat[1]{\fbox{\normalfont\scriptsize\sffamily#1}}   % for showkeys

\usepackage{comment}
\usepackage{geometry}
\geometry{a4paper,margin=1in}
\usepackage{natbib}
% \usepackage[bibstyle=trad-abbrv, natbib=true, citestyle=numeric-comp, backref=true, useprefix, uniquename=false,maxcitenames=2]{biblatex}
% \newcommand{\citep}[]{}
%\setcitestyle{square}

\usepackage[tbtags]{amsmath}
\usepackage{amsthm}
\allowdisplaybreaks
\usepackage{amssymb,mathrsfs}
\usepackage{nccmath}
\usepackage{amsfonts}
\usepackage{upgreek}
\usepackage{xspace}

% \usepackage{nicefrac}

%\usepackage[numbers]{natbib}
\usepackage{graphicx}
% \usepackage{subfig}
%\usepackage[caption = false]{subfig} %package pour faire sous-figures
\usepackage{color}
%\usepackage[ruled,vlined]{algorithm2e}
%\usepackage{algpseudocode,algorithm,algorithmicx}
\usepackage{algorithm, algpseudocode}
\begin{comment}

\algnewcommand{\Inputs}[1]{%
  \State \textbf{Inputs:}
  \Statex \hspace*{\algorithmicindent}\parbox[t]{.8\linewidth}{\raggedright #1}
}
\algnewcommand{\Initialize}[1]{%
  \State \textbf{Initialize:}
  \Statex \hspace*{\algorithmicindent}\parbox[t]{.8\linewidth}{\raggedright #1}
}
\algnewcommand{\Outputs}[1]{%
  \State \textbf{Outputs:}
  \Statex \hspace*{\algorithmicindent}\parbox[t]{.8\linewidth}{\raggedright #1}
}
\end{comment}

%###########
%\usepackage{manuColor}
\usepackage{stmaryrd}
\usepackage[inline]{enumitem}
%[wide, labelwidth=!, labelindent=0pt]
\usepackage{url}
\def\UrlBreaks{\do\/\do-}
\usepackage{tikz}
\usetikzlibrary{calc}
\newcommand\yBlock{1}
\newcommand\yNode{0.75}

\newcommand\xNodemoinstiny{-1}
\newcommand\xNodemoins{-1.5}
\newcommand\xNodemoinsint{-2.}
\newcommand\xNodeMoins{-3}
\newcommand\xNodeMOINS{-4.5}

\newcommand\xNodeplustiny{1}
\newcommand\xNodeplus{1.5}
\newcommand\xNodeplusint{2}
\newcommand\xNodePlus{3}
\newcommand\xNodePLUS{4.5}

\usepackage{pgfplots}
\usepackage{xcolor}
\usepackage{bbm}
\usepackage{ifthen}
\usepackage{xargs}
\usepackage[textwidth=1.8cm]{todonotes}

\usepackage{aliascnt}
% \usepackage{cleveref}
\usepackage[capitalise,noabbrev]{cleveref}
\usepackage{autonum}
\makeatletter
\newtheorem{theorem}{Theorem}
% \crefname{theorem}{theorem}{Theorems}
% \Crefname{Theorem}{Theorem}{Theorems}


\newtheorem*{lemma_nonumber*}{Lemma}


\newaliascnt{lemma}{theorem}
\newtheorem{lemma}[lemma]{Lemma}
\aliascntresetthe{lemma}
% \crefname{lemma}{lemma}{lemmas}
% \Crefname{Lemma}{Lemma}{Lemmas}



\newaliascnt{corollary}{theorem}
\newtheorem{corollary}[corollary]{Corollary}
\aliascntresetthe{corollary}
% \crefname{corollary}{corollary}{corollaries}
% \Crefname{Corollary}{Corollary}{Corollaries}

\newaliascnt{proposition}{theorem}
\newtheorem{proposition}[proposition]{Proposition}
\aliascntresetthe{proposition}
% \crefname{proposition}{proposition}{propositions}
% \Crefname{Proposition}{Proposition}{Propositions}

\newaliascnt{definition}{theorem}
\newtheorem{definition}[definition]{Definition}
\aliascntresetthe{definition}
% \crefname{definition}{definition}{definitions}
% \Crefname{Definition}{Definition}{Definitions}

\newaliascnt{remark}{theorem}
\newtheorem{remark}[remark]{Remark}
\aliascntresetthe{remark}
% \crefname{remark}{remark}{remarks}
% \Crefname{Remark}{Remark}{Remarks}


\newtheorem{example}[theorem]{Example}
% \crefname{example}{example}{examples}
% \Crefname{Example}{Example}{Examples}

\newtheorem{technique}{Technique}
% \crefname{technique}{technique}{techniques}
% \Crefname{Technique}{Technique}{Techniques}


% \crefname{figure}{figure}{figures}
% \Crefname{Figure}{Figure}{Figures}


%\newtheorem{assumption}{\textbf{A}\hspace{-3pt}}
%\Crefname{assumption}{\textbf{A}\hspace{-3pt}}{\textbf{A}\hspace{-3pt}}
%\crefname{assumption}{\textbf{A}}{\textbf{A}}
\newtheorem{assumption}{\textbf{A}\hspace{-3pt}}
\crefformat{assumption}{{\textbf{A}}#2#1#3}

\newtheorem{assumptionF}{\textbf{F}\hspace{-3pt}}
\crefformat{assumptionF}{{\textbf{F}}#2#1#3}

\newenvironment{assumptionbis}[1]
  {\renewcommand{\theassumptionF}{\ref*{#1}$\mathbf{b}$}%
   \addtocounter{assumptionF}{-1}%
   \begin{assumptionF}}
  {\end{assumptionF}}



\newtheorem{assumptionB}{\textbf{B}\hspace{-3pt}}
\Crefname{assumptionB}{\textbf{B}\hspace{-3pt}}{\textbf{B}\hspace{-3pt}}
\crefname{assumptionB}{\textbf{B}}{\textbf{B}}

\newtheorem{assumptionC}{\textbf{C}\hspace{-3pt}}
\Crefname{assumptionC}{\textbf{C}\hspace{-3pt}}{\textbf{C}\hspace{-3pt}}
\crefname{assumptionC}{\textbf{C}}{\textbf{C}}


\newtheorem{assumptionH}{\textbf{H}\hspace{-3pt}}
\Crefname{assumptionH}{\textbf{H}\hspace{-3pt}}{\textbf{H}\hspace{-3pt}}
\crefname{assumptionH}{\textbf{H}}{\textbf{H}}

\newtheorem{assumptionT}{\textbf{T}\hspace{-3pt}}
\Crefname{assumptionT}{\textbf{T}\hspace{-3pt}}{\textbf{T}\hspace{-3pt}}
\crefname{assumptionT}{\textbf{T}}{\textbf{T}}

\newtheorem{assumptionD}{\textbf{D}\hspace{-3pt}}
\Crefname{assumptionT}{\textbf{T}\hspace{-3pt}}{\textbf{T}\hspace{-3pt}}
\crefname{assumptionT}{\textbf{T}}{\textbf{T}}


\newtheorem{assumptionL}{\textbf{L}\hspace{-3pt}}
\Crefname{assumptionL}{\textbf{L}\hspace{-3pt}}{\textbf{L}\hspace{-3pt}}
\crefname{assumptionL}{\textbf{L}}{\textbf{L}}

\newtheorem{assumptionQ}{\textbf{Q}\hspace{-3pt}}
\Crefname{assumptionQ}{\textbf{Q}\hspace{-3pt}}{\textbf{Q}\hspace{-3pt}}
\crefname{assumptionQ}{\textbf{Q}}{\textbf{Q}}

% \newtheorem{assumptionD*}{\textbf{D}\hspace{-3pt}}
% \Crefname{assumptionD}{\textbf{D}\hspace{-3pt}}{\textbf{D}\hspace{-3pt}}
% \crefname{assumptionD}{\textbf{D}}{\textbf{D}}

\newtheorem{assumptionAR}{\textbf{AR}\hspace{-3pt}}
\Crefname{assumptionAR}{\textbf{AR}\hspace{-3pt}}{\textbf{AR}\hspace{-3pt}}
\crefname{assumptionAR}{\textbf{AR}}{\textbf{AR}}



\newcommand\diaW{11}
\newcommand\diaH{5}
\newcommand\diaJump{2.75}
\newcommand\nextRow{1.25}
\newcommand\imW{0.08}
\newcommand\imWB{0.1}
\newcommand\imOp{0.6}
\newcommand\bend{5}

\newcommand\offset{2}
\newcommand\offsety{2.3}
\newcommand\h{2.25}
\newcommand\hsmall{1.75}
\newcommand\ww{3.25}
\newcommand\www{1.8}
\newcommand\wwww{3.5}
\newcommand\wwwww{4.8}
\newcommand{\offsetsmall}{1.5}


\usepackage{bm}
\usepackage{wrapfig}
