\documentclass[11pt,a4paper]{article}
\usepackage{tmlr}

\usepackage[utf8]{inputenc} % allow utf-8 input
\usepackage[T1]{fontenc}    % use 8-bit T1 fonts
\usepackage{hyperref}       % hyperlinks
\usepackage{url}            % simple URL typesetting
\usepackage{booktabs}       % professional-quality tables
\usepackage{amsfonts}       % blackboard math symbols
\usepackage{nicefrac}       % compact symbols for 1/2, etc.
\usepackage{microtype}      % microtypography
\usepackage{xcolor}         % colors
\usepackage{tikz}
% \usepackage{caption}
\usepackage{float}
\usetikzlibrary{arrows.meta}
\usetikzlibrary{calc}
\usepackage[utf8]{inputenc}   % LaTeX, comprends les accents !
\usepackage[T1]{fontenc}      % Police contenant les caractères français
%\usepackage[french]{babel}  % Placez ici une liste de langues
%\usepackage{multicol}

%%%%%%%%%%%%%%
%% comment uncomment
%\usepackage[notref,notcite]{showkeys}
%%%%


 % \usepackage[notref,notcite]{showkeys}  %  comment out for final version
 % \renewcommand*\showkeyslabelformat[1]{\fbox{\normalfont\scriptsize\sffamily#1}}   % for showkeys

\usepackage{comment}
\usepackage{geometry}
\geometry{a4paper,margin=1in}
\usepackage{natbib}
% \usepackage[bibstyle=trad-abbrv, natbib=true, citestyle=numeric-comp, backref=true, useprefix, uniquename=false,maxcitenames=2]{biblatex}
% \newcommand{\citep}[]{}
%\setcitestyle{square}

\usepackage[tbtags]{amsmath}
\usepackage{amsthm}
\allowdisplaybreaks
\usepackage{amssymb,mathrsfs}
\usepackage{nccmath}
\usepackage{amsfonts}
\usepackage{upgreek}
\usepackage{xspace}

% \usepackage{nicefrac}

%\usepackage[numbers]{natbib}
\usepackage{graphicx}
% \usepackage{subfig}
%\usepackage[caption = false]{subfig} %package pour faire sous-figures
\usepackage{color}
%\usepackage[ruled,vlined]{algorithm2e}
%\usepackage{algpseudocode,algorithm,algorithmicx}
\usepackage{algorithm, algpseudocode}
\begin{comment}

\algnewcommand{\Inputs}[1]{%
  \State \textbf{Inputs:}
  \Statex \hspace*{\algorithmicindent}\parbox[t]{.8\linewidth}{\raggedright #1}
}
\algnewcommand{\Initialize}[1]{%
  \State \textbf{Initialize:}
  \Statex \hspace*{\algorithmicindent}\parbox[t]{.8\linewidth}{\raggedright #1}
}
\algnewcommand{\Outputs}[1]{%
  \State \textbf{Outputs:}
  \Statex \hspace*{\algorithmicindent}\parbox[t]{.8\linewidth}{\raggedright #1}
}
\end{comment}

%###########
%\usepackage{manuColor}
\usepackage{stmaryrd}
\usepackage[inline]{enumitem}
%[wide, labelwidth=!, labelindent=0pt]
\usepackage{url}
\def\UrlBreaks{\do\/\do-}
\usepackage{tikz}
\usetikzlibrary{calc}
\newcommand\yBlock{1}
\newcommand\yNode{0.75}

\newcommand\xNodemoinstiny{-1}
\newcommand\xNodemoins{-1.5}
\newcommand\xNodemoinsint{-2.}
\newcommand\xNodeMoins{-3}
\newcommand\xNodeMOINS{-4.5}

\newcommand\xNodeplustiny{1}
\newcommand\xNodeplus{1.5}
\newcommand\xNodeplusint{2}
\newcommand\xNodePlus{3}
\newcommand\xNodePLUS{4.5}

\usepackage{pgfplots}
\usepackage{xcolor}
\usepackage{bbm}
\usepackage{ifthen}
\usepackage{xargs}
\usepackage[textwidth=1.8cm]{todonotes}

\usepackage{aliascnt}
% \usepackage{cleveref}
\usepackage[capitalise,noabbrev]{cleveref}
\usepackage{autonum}
\makeatletter
\newtheorem{theorem}{Theorem}
% \crefname{theorem}{theorem}{Theorems}
% \Crefname{Theorem}{Theorem}{Theorems}


\newtheorem*{lemma_nonumber*}{Lemma}


\newaliascnt{lemma}{theorem}
\newtheorem{lemma}[lemma]{Lemma}
\aliascntresetthe{lemma}
% \crefname{lemma}{lemma}{lemmas}
% \Crefname{Lemma}{Lemma}{Lemmas}



\newaliascnt{corollary}{theorem}
\newtheorem{corollary}[corollary]{Corollary}
\aliascntresetthe{corollary}
% \crefname{corollary}{corollary}{corollaries}
% \Crefname{Corollary}{Corollary}{Corollaries}

\newaliascnt{proposition}{theorem}
\newtheorem{proposition}[proposition]{Proposition}
\aliascntresetthe{proposition}
% \crefname{proposition}{proposition}{propositions}
% \Crefname{Proposition}{Proposition}{Propositions}

\newaliascnt{definition}{theorem}
\newtheorem{definition}[definition]{Definition}
\aliascntresetthe{definition}
% \crefname{definition}{definition}{definitions}
% \Crefname{Definition}{Definition}{Definitions}

\newaliascnt{remark}{theorem}
\newtheorem{remark}[remark]{Remark}
\aliascntresetthe{remark}
% \crefname{remark}{remark}{remarks}
% \Crefname{Remark}{Remark}{Remarks}


\newtheorem{example}[theorem]{Example}
% \crefname{example}{example}{examples}
% \Crefname{Example}{Example}{Examples}

\newtheorem{technique}{Technique}
% \crefname{technique}{technique}{techniques}
% \Crefname{Technique}{Technique}{Techniques}


% \crefname{figure}{figure}{figures}
% \Crefname{Figure}{Figure}{Figures}


%\newtheorem{assumption}{\textbf{A}\hspace{-3pt}}
%\Crefname{assumption}{\textbf{A}\hspace{-3pt}}{\textbf{A}\hspace{-3pt}}
%\crefname{assumption}{\textbf{A}}{\textbf{A}}
\newtheorem{assumption}{\textbf{A}\hspace{-3pt}}
\crefformat{assumption}{{\textbf{A}}#2#1#3}

\newtheorem{assumptionF}{\textbf{F}\hspace{-3pt}}
\crefformat{assumptionF}{{\textbf{F}}#2#1#3}

\newenvironment{assumptionbis}[1]
  {\renewcommand{\theassumptionF}{\ref*{#1}$\mathbf{b}$}%
   \addtocounter{assumptionF}{-1}%
   \begin{assumptionF}}
  {\end{assumptionF}}



\newtheorem{assumptionB}{\textbf{B}\hspace{-3pt}}
\Crefname{assumptionB}{\textbf{B}\hspace{-3pt}}{\textbf{B}\hspace{-3pt}}
\crefname{assumptionB}{\textbf{B}}{\textbf{B}}

\newtheorem{assumptionC}{\textbf{C}\hspace{-3pt}}
\Crefname{assumptionC}{\textbf{C}\hspace{-3pt}}{\textbf{C}\hspace{-3pt}}
\crefname{assumptionC}{\textbf{C}}{\textbf{C}}


\newtheorem{assumptionH}{\textbf{H}\hspace{-3pt}}
\Crefname{assumptionH}{\textbf{H}\hspace{-3pt}}{\textbf{H}\hspace{-3pt}}
\crefname{assumptionH}{\textbf{H}}{\textbf{H}}

\newtheorem{assumptionT}{\textbf{T}\hspace{-3pt}}
\Crefname{assumptionT}{\textbf{T}\hspace{-3pt}}{\textbf{T}\hspace{-3pt}}
\crefname{assumptionT}{\textbf{T}}{\textbf{T}}

\newtheorem{assumptionD}{\textbf{D}\hspace{-3pt}}
\Crefname{assumptionT}{\textbf{T}\hspace{-3pt}}{\textbf{T}\hspace{-3pt}}
\crefname{assumptionT}{\textbf{T}}{\textbf{T}}


\newtheorem{assumptionL}{\textbf{L}\hspace{-3pt}}
\Crefname{assumptionL}{\textbf{L}\hspace{-3pt}}{\textbf{L}\hspace{-3pt}}
\crefname{assumptionL}{\textbf{L}}{\textbf{L}}

\newtheorem{assumptionQ}{\textbf{Q}\hspace{-3pt}}
\Crefname{assumptionQ}{\textbf{Q}\hspace{-3pt}}{\textbf{Q}\hspace{-3pt}}
\crefname{assumptionQ}{\textbf{Q}}{\textbf{Q}}

% \newtheorem{assumptionD*}{\textbf{D}\hspace{-3pt}}
% \Crefname{assumptionD}{\textbf{D}\hspace{-3pt}}{\textbf{D}\hspace{-3pt}}
% \crefname{assumptionD}{\textbf{D}}{\textbf{D}}

\newtheorem{assumptionAR}{\textbf{AR}\hspace{-3pt}}
\Crefname{assumptionAR}{\textbf{AR}\hspace{-3pt}}{\textbf{AR}\hspace{-3pt}}
\crefname{assumptionAR}{\textbf{AR}}{\textbf{AR}}



\newcommand\diaW{11}
\newcommand\diaH{5}
\newcommand\diaJump{2.75}
\newcommand\nextRow{1.25}
\newcommand\imW{0.08}
\newcommand\imWB{0.1}
\newcommand\imOp{0.6}
\newcommand\bend{5}

\newcommand\offset{2}
\newcommand\offsety{2.3}
\newcommand\h{2.25}
\newcommand\hsmall{1.75}
\newcommand\ww{3.25}
\newcommand\www{1.8}
\newcommand\wwww{3.5}
\newcommand\wwwww{4.8}
\newcommand{\offsetsmall}{1.5}


\usepackage{bm}
\usepackage{wrapfig}

\def\rmB{\mathrm{B}}
\def\ellim{\ell^{\mathrm{im}}}
\def\piinv{p_{\textup{ref}}}
\def\piinvb{\bar{\pi}_{\mathrm{inv}}}
% \def\pizero{\pi_0}
\def\piinv{p_{\textup{ref}}}
\def\pizero{p_0}

\newcommand{\mjh}[1]{\textcolor{blue}{#1}}


\newcommand{\mY}{\bm{Y}}
\newcommand{\mX}{\bm{X}}
\newcommand{\mW}{\bm{W}}
\newcommand{\mZ}{\bm{Z}}
\newcommand{\mz}{\bm{z}}
\newcommand{\mB}{\bm{B}}
\newcommand{\vf}[1]{\bm{v}(\#1)}

\newcommand{\grad}{\mathrm{grad}}
\newcommand{\dive}{\mathrm{div}}

\newcommand{\prodM}[2]{\langle #1, #2 \rangle_\M}
\newcommand{\XM}{\mathcal{X}(\mathcal{M})}
\newcommand{\XMdeux}{\mathcal{X}^2(\mathcal{M})}
\newcommand{\Xgamma}{\mathcal{X}(\gamma)}
\newcommand{\TM}{\mathrm{T}\mathcal{M}}
\newcommand{\FM}{\mathrm{F}\mathcal{M}}
\newcommand{\OM}{\mathrm{O}\mathcal{M}}
\newcommand{\TMstar}{\mathrm{T}^\star\mathcal{M}}
\newcommand{\detLigne}[1]{\det(#1)}
\def\hlf{\hat{\ell}^f}
\def\hlb{\hat{\ell}^b}
\def\Ent{\mathrm{H}}
\def\lyap{V_{p,t,x_t}}
\def\lyapp{V_{p}}
\def\carrechamp{\Upsilon}
\def\carrechampb{\bar{\Upsilon}}

\def\contspace{\mathcal{C}}
\def\pdata{p_{\textup{data}}}
\def\qdata{q_{\textup{data}}}
\def\pprior{p_{\textup{prior}}}

\def\for{\mathrm{f}}
\def\back{\mathrm{b}}
\def\lf{\ell^{\mathrm{f}}}
\def\lb{\ell^{\mathrm{b}}}
\def\sf{s^{\mathrm{f}}}
\def\sb{s^{\mathrm{b}}}

\def\Tcal{\mathcal{T}}
\def\bfpi{\bm{\pi}}
\def\bfnu{\bm{\nu}}

% \def\Pens{\mathscr{P}}
\def\Pens{\mathcal{P}}
\def\Mens{\mathscr{M}}
\def\pif{\overrightarrow{\pi}}
\def\lambdabff{\overrightarrow{\bm{\lambda}}}
\def\lambdabfb{\overleftarrow{\bm{\lambda}}}
\newcommand{\mail}[1]{\footnote{Email: \href{mailto:#1}{\textcolor{black}{#1}}}}
\def\Phif{\overrightarrow{\Phi}}
\def\Phib{\overleftarrow{\Phi}}
\def\scoref{\overrightarrow{\mathrm{S}}}
\def\scoreb{\overleftarrow{\mathrm{S}}}
\def\netf{\overrightarrow{\mathrm{NN}}}
\def\netb{\overleftarrow{\mathrm{NN}}}
\newcommand{\schro}{Schr\"{o}dinger\xspace}
\newcommand{\Cweakapp}{\ttd}
\def\ttfp{\Cweakapp_{p}}
\def\ttfpun{\Cweakapp_{p,1}}
\def\ttfpdeux{\Cweakapp_{p,2}}
\def\ttfptrois{\Cweakapp_{p,3}}
\def\ttfpquatre{\Cweakapp_{p,4}}
\def\ttamin{\mathtt{a}}
\def\ttfun{\Cweakapp_4}
\def\ttfdeux{\Cweakapp_5}
\def\btta{\bar{\mathtt{A}}}
\def\bfb{\mathbf{b}}
\def\bfsigma{\pmb{\sigma}}
\def\KuLo{Kurdyka-\L ojasiewicz}
\newcommand{\tta}{\mathtt{A}}
\newcommand{\ttb}{\mathtt{B}}
\newcommand{\ttc}{\mathtt{C}}
\newcommand{\ttd}{\mathtt{D}}
\def\tte{\mathtt{E}}
\newcommand{\ttM}{\mathtt{M}}
\def\boundLSig{\Lip\eta}

\newcommand{\Capprox}{\tta}
\newcommand{\Ctech}{\ttc}
\newcommand{\Cstrong}{\ttb}
\newcommand{\Cconv}{\ttc}
\newcommand{\Cweak}{C}

\def\conj{\varkappa}
\def\mtta{\mathtt{a}}
\def\explog{\vareps}
\newcommand{\note}[1]{\textcolor{red}{#1}}
\def\Cbeta{\Cweak_{\beta, \explog}}
\def\Aar{\Capprox_{\alpha, r}}
\def\xo{x_0}
\def\Db{\Ctech}
\def\intk{\int_{k\gua}^{(k+1)\gua}}
\newcommandx\ctun[1][1=T]{\Capprox_{#1,1}}
\def\btun{\mathtt{B}_1}
\def\btdeux{\mathtt{B}_2}
\def\dtun{\mathtt{D}_1}
\def\cttun{\tilde{\Capprox}_{T,1}}
\def\dtdeux{\mathtt{D}_2}
\def\ctdeux{\Capprox_{T,2}}
\def\cttrois{\Capprox_{T,3}}
\def\ctquatre{\Capprox_{T,4}}
\def\ctcinq{\Capprox_{T,5}}
\def\ctsix{\Capprox_{T,6}}
\def\ctsept{\Capprox_{T,7}}
\def\cthuit{\Capprox_{T,8}}
\def\ctneuf{\Capprox_{T,9}}
\def\gfun{\mathbb{G}}
\def\hash{\sharp}
\def\Cconvcontun{\Cconv_{1,\alpha}^{(c)}}
\def\Cconvcontdeux{\Cconv_{2,\alpha}^{(c)}}
\def\Cconvconttrois{\Cconv_{3,\alpha}^{(c)}}
\def\Cconvdiscun{\Cconv_{1,\alpha}^{(d)}}
\def\Cconvdiscdeux{\Cconv_{2,\alpha}^{(d)}}
\def\Cconvdisctrois{\Cconv_{3,\alpha}^{(d)}}
\def\Cconvcont{\Phibf_{\alpha}^{(c)}}
\def\Cconvdisc{\Phibf_{\alpha}^{(d)}}
\def\Csham{\Cconv_1}
\def\Cshamd{\Cconv_2}
\def\Cshama{\Cconv_{\alpha}}
\def\Cshamamoins{\Cshama^-}
\def\Cshamaplus{\Cshama^+}
\def\Ccont{\Cconv^{(c)}}
\def\Cdisc{\Cconv^{(d)}}
\def\Cconvk{{\Cconv^{(a)}_k}}
%\def\Cconvdun{\Cconv^{(b)}_1}
%\def\Cconvddeux{\Cconv^{(b)}_2}
\def\Cconvdtrois{\Cconv^{(b)}}
\def\Cconvdun{(\gamma\eta/2)}
\def\Cconvddeux{(\gamma/2)}
\def\Cshamdisc{\Cconv_{0}}
\def\Cshamt{\tilde{\Cconv}_{\alpha}}
\def\Psial{\Psibf_{\alpha}}
\def\Cstrongcont{\Cstrong_1}
\def\Cstrongcontf{\Cstrong_2}
\def\Cstrongdisc{\Cstrong_3}
\def\Cstrongdiscf{\Cstrong_4}
\def\Cstrongloj{\Cstrong_5}
\def\Cstronglojdisc{\Cstrong_6}
\def\Cstrongtilde{\tilde{\Cstrong}}
\def\maxnorm{C}
\newcommand{\pinv}{^{-1}}
\newcommand{\st}{^{\star}}
\newcommand{\gb}{\gamma^{\beta}}
\newcommand{\tr}{^{\top}}
\def\scrE{\mathscr{E}}
\def\scrV{\mathscr{V}}
\def\scrF{\mathscr{F}}
\newcommand{\rref}[1]{\tup{\Cref{#1}}}
\newcommand{\la}{\langle}
\newcommand{\ra}{\rangle}
\newcommand{\LL}{\L ojasciewicz~}
\newcommand{\gua}{\gamma_{\alpha}}
\newcommand{\bgua}{\bgamma_{\alpha}}
\newcommand{\gda}{\gua^{1/2}}
\newcommand{\tgua}{(t+\gua)^{\alpha}}
\newcommand{\guac}{c}
\newcommand{\et}{\quad\mbox{and}\quad}
%\newcommand{\sigb}{\ttM_{\Sigma}}
\newcommand{\sigb}{\eta}
\newcommand{\phe}{\varphi_{\varepsilon}}
\newcommand{\feps}{f_{\varepsilon}}
\newcommand{\nfeps}{\nabla f_{\varepsilon}}
\newcommand{\intd}{\int_{\bR^{\dim}}}
\newcommandx{\expec}[2]{{\mathbb E}\left[#1 \middle \vert #2  \right]} %%%% esperance conditionnelle
\newcommand{\expek}[1]{\expec{#1}{\cF_k}}
\newcommand{\expen}[1]{\expec{#1}{\cF_n}}
\newcommand{\nn}{_{n+1}}
\newcommand{\kk}{_{k+1}}
\newcommand{\pal}{^{\alpha}}
\newcommand{\pmal}{^{-\alpha}}
\newcommand{\cH}{\mathcal{H}}

\def\En{\tilde{E}_n}
\def\varepsn{\tilde{\vareps}_n}
\def\pow{p}
\def\ntt{\mathtt{n}_0}
\def\tlambda{\tilde{\lambda}}
\def\dim{d}
\newcommand{\tb}{\tilde{b}}
\newcommand{\Time}{T}
\newcommand{\mttun}{\mathtt{k}_1}
\newcommand{\mttdeux}{\mathtt{k}_2}
\newcommand{\mtttrois}{\mtt_3^+}
\newcommand{\bvareps}{\bar{\vareps}}
\newcommand{\transference}{\mathbf{T}}
\newcommand{\esssup}{\mathrm{ess sup}}
\newcommand{\ring}{\mathcal{C}_{\varrho}}
\newcommand{\measx}{\mathcal{X}}
\newcommand{\bkappa}{\bar{\kappa}}
\newcommand{\probaspace}[1]{\mathbb{P}\left( #1 \right)}
\newcommand{\dTVdeux}{d_{\mathrm{TV}, 2}}
\newcommand{\dTVDeux}[1]{d_{\mathrm{TV}, 2}\left( #1 \right)}
\newcommand{\bgM}{b_{\gamma, n}}
\newcommand{\bbgM}{\bar{b}_{\gamma, M}}
\newcommand{\rme}{\mathrm{e}}
\newcommand{\rmF}{\mathrm{F}}
\newcommand{\rmE}{\mathrm{E}}
\newcommand{\Fdr}{\mathrm{f}}
\newcommand{\Gdr}{\mathrm{g}}
\newcommand{\alphastar}{\alpha_{\star}}
\newcommand{\LipVset}{\mathrm{Lip}_{V, \alpha}}
\newcommand{\Lip}{\mathtt{L}}
\newcommand{\Lipset}{\mathrm{Lip}}
\newcommand{\Mtt}{\mathtt{M}}
\newcommand{\Ktt}{\mathtt{K}}
\newcommand{\tLip}{\tilde{\mathtt{L}}}
\newcommand{\tell}{\tilde{\ell}}
\newcommand{\Lipb}{\mtt_b}
\newcommand{\step}{\ceil{1/\gamma}}
\newcommand{\bstep}{\ceil{1/\bgamma}}
\def\bdisc{b}
\def\bfDd{\mathbf{D}_{\mathrm{d}}}
\def\bfDc{\mathbf{D}_{\mathrm{c}}}
\newcommand{\SDE}{\mathrm{SDE}}

\newcommand{\bbeta}{\bar{\beta}}
\newcommand{\measfun}{\mathbb{F}}
\newcommand{\btheta}{\boldsymbol{\theta}}
\newcommand{\bdeta}{\boldsymbol{\eta}}
\newcommand{\bvarphi}{\boldsymbol{\varphi}}

%\newcommand{\tau}{\boldsymbol{\tau}}
%\newcommand{\x}{\boldsymbol{x}}
%\newcommand{\X}{\boldsymbol{X}}
%\newcommand{\y}{\boldsymbol{y}}
%%\newcommand{\u}{\boldsymbol{u}}
%\newcommand{\w}{\boldsymbol{w}}
%\newcommand{\z}{\boldsymbol{z}}
%\newcommand{\p}{\boldsymbol{p}}
%\newcommand{\s}{\mathcal{S}}
%\newcommand{\ind}{\boldsymbol{1}}
%\newcommand{\dx}{\boldsymbol{\delta}\boldsymbol{x}}
%\newcommand{\argmax}{\operatornamewithlimits{argmax}}
%\newcommand{\argmin}{\operatornamewithlimits{argmin}}
%\newcommand{\prox}{\operatorname{prox}}
\def\x{{ \boldsymbol x}}
\def\u{{ \boldsymbol u}}
\def\y{{\boldsymbol y}}
\def\z{{\boldsymbol z}}
\def\w{{\boldsymbol w}}

\def\xt{ \boldsymbol x^t}
\newcommandx{\norm}[2][1=]{\ifthenelse{\equal{#1}{}}{\left\Vert #2 \right\Vert}{\left\Vert #2 \right\Vert^{#1}}}
\newcommandx{\normLigne}[2][1=]{\ifthenelse{\equal{#1}{}}{\Vert #2 \Vert}{\Vert #2\Vert^{#1}}}


\newcommand\mycomment[1]{\textcolor{red}{#1}}

%\theoremstyle{definition}
%\newtheorem{defn}{Definition}[section]
%\newtheorem{assump}{A}[paragraph]
%\newtheorem{prop}{Proposition}[section]
%\newtheorem{theo}{Theorem}[section]
%\newtheorem{coro}{Corollary}[section]
%\newtheorem{lemma}{Lemma}[section]
%\newtheorem{exmp}{Example}[section]

\def\xstart{x^{\star}_{\theta}}

%%%%%%%%%%%%%%%
%% mathbf

\def\bfn{\mathbf{n}}
\def\bfw{\mathbf{w}}
\def\bfc{\mathbf{c}}
\def\bfY{\mathbf{Y}}
\def\bfhY{\hat{\mathbf{Y}}}
\def\bbfY{\bar{\mathbf{Y}}}
\def\bfX{\mathbf{X}}
\def\bfhX{\hat{\mathbf{X}}}
\def\bfW{\mathbf{W}}
\def\bfU{\mathbf{U}}
\def\bfE{\mathbf{E}}
\def\bfs{\mathbf{s}}
\def\bfZ{\mathbf{Z}}
\def\bfXt{\tilde{\mathbf{X}}}
\def\bfXd{\overline{\mathbf{X}}}
\def\bfYd{\overline{\mathbf{Y}}}
\def\bfZ{\mathbf{Z}}
\def\bbfX{\tilde{\mathbf{X}}}
\def\bfM{\mathbf{M}}
\def\bfB{\mathbf{B}}
\def\bfP{\mathbf{P}}
%%% mathsf
\def\msi{\mathsf{I}}
\def\msa{\mathsf{A}}
\def\msd{\mathsf{D}}
\def\msk{\mathsf{K}}
\def\mss{\mathsf{S}}
\def\msn{\mathsf{N}}
\def\msat{\tilde{\mathsf{A}}}
\def\msb{\mathsf{B}}
\def\msc{\mathsf{C}}
\def\tmsc{\tilde{\msc}}
\def\mse{\mathsf{E}}
\def\msf{\mathsf{F}}
\def\tmsf{\tilde{\msf}}
\def\mso{\mathsf{o}}
\def\msg{\mathsf{G}}
\def\msh{\mathsf{H}}
\def\msm{\mathsf{M}}
\def\msu{\mathsf{U}}
\def\msv{\mathsf{V}}
\def\msr{\mathsf{R}}
\newcommand{\msff}[2]{\mathsf{F}_{#1}^{#2}}
\def\msp{\mathsf{P}}
\def\msq{\mathsf{Q}}
\def\msx{\mathsf{X}}
\def\msz{\mathsf{Z}}
\def\msy{\mathsf{Y}}
\def\ddx{d_\msx}
\def\ddy{d_\msy}

%% mathcal
\def\mca{\mathcal{A}}
\def\mct{\mathcal{T}}
\def\mcat{\tilde{\mathcal{A}}}
\def\mcab{\bar{\mathcal{A}}}
\def\mcbb{\mathcal{B}}  %%% \mcb est déjà pris
\newcommand{\mcb}[1]{\mathcal{B}(#1)}
\def\mcc{\mathcal{C}}
\def\mcz{\mathcal{Z}}
\def\mcy{\mathcal{Y}}
\def\mcx{\mathcal{X}}
\def\mce{\mathcal{E}}
\def\mcs{\mathcal{S}}
\def\mcf{\mathcal{F}}
\def\mcg{\mathcal{G}}
\def\mch{\mathcal{H}}
\def\mcm{\mathcal{M}}
\def\mcu{\mathcal{U}}
\def\mcv{\mathcal{V}}
\def\mcr{\mathcal{R}}
\newcommand{\mcff}[2]{\mathcal{F}_{#1}^{#2}}
\def\mcfb{\bar{\mathcal{F}}}
\def\bmcf{\bar{\mathcal{F}}}
\def\mcft{\tilde{\mathcal{F}}}
\def\tmcf{\tilde{\mathcal{F}}}
\def\mcp{\mathcal{P}}
\def\mcq{\mathcal{Q}}

%% mathbb

\def\Qbb{\mathbb{Q}}
\def\Rbb{\mathbb{R}}
\def\Mbb{\mathbb{M}}
\def\Pbb{\mathbb{P}}
\def\Hbb{\mathbb{H}}
\newcommand{\Qit}[1]{\Qbb^{(#1)}}
\newcommand{\Pit}[1]{\Pbb^{(#1)}}

\def\rset{\mathbb{R}}
\def\rsets{\mathbb{R}^*}
\def\cset{\mathbb{C}}
\def\zset{\mathbb{Z}}
\def\tset{\mathbb{T}}
\def\nset{\mathbb{N}}
\def\nsets{\mathbb{N}^{\star}}
\def\qset{\mathbb{Q}}
\def\Rset{\mathbb{R}}
\def\Cset{\mathbb{C}}
\def\Zset{\mathbb{Z}}
\def\Nset{\mathbb{N}}
\def\Tset{\mathbb{T}}

\def\bN{\mathbb{N}}
\def\bR{\mathbb{R}}
\def\bRd{\mathbb{R}^{\dim}}
\def\cF{\mathcal{F}}


%%%% mathrm

\def\rmP{\mathrm{P}}
\def\rmQ{\mathrm{Q}}
\def\rmR{\mathrm{R}}
\def\rmb{\mathrm{b}}
\def\mrb{\mathrm{b}}
\def\wrm{\mathrm{w}}
\def\rmw{\mathrm{w}}
\def\rmd{\mathrm{d}}
\def\rmm{\mathrm{m}}
\def\rms{\mathrm{s}}
\def\rmZ{\mathrm{Z}}
\def\rmS{\mathrm{S}}
\def\mrd{\mathrm{d}}
\def\mre{\mathrm{e}}
\def\rme{\mathrm{e}}
\def\rmn{\mathrm{n}}
\def\mrn{\mathrm{n}}
\def\mrc{\mathrm{C}}
\def\mrcc{\mathrm{c}}
\def\rmc{\mathrm{C}}
\def\rmC{\mathrm{C}}
\def\GaStep{\Gamma}
\def\rmcc{\mathrm{c}}
\def\rma{\mathrm{a}}
\def\rmf{\mathrm{f}}
\def\rmg{\mathrm{g}}
\def\rmh{\mathrm{h}}
\def\rmv{\mathrm{v}}
\def\mra{\mathrm{a}}

\def\cov{\mathrm{Cov}}

\newcommand{\cco}{\llbracket}
\newcommand{\ccf}{\rrbracket}
\newcommand{\po}{\left(}
\newcommand{\pf}{\right)}
\newcommand{\co}{\left[}
\newcommand{\cf}{\right]}
\newcommand{\R}{\mathbb R}
\newcommand{\Z}{\mathbb Z}
\newcommand{\D}{\mathcal D}
\newcommand{\dd}{\mathrm{d}}
\newcommand{\A}{\mathcal A}
\newcommand{\M}{\mathcal M}
\newcommand{\na}{\nabla}
\newcommand{\loiy}{\mu_{\mathrm{v}}}


\def\MeasFspace{\mathbb{M}}
\def\xstar{x^\star}
\def\Tr{\operatorname{T}}
\def\trace{\operatorname{Tr}}
\newcommandx{\functionspace}[2][1=+]{\mathbb{F}_{#1}(#2)}
%% argmin, argmax
\newcommand{\argmax}{\operatorname*{arg\,max}}
\newcommand{\argmin}{\operatorname*{arg\,min}}
\newcommand{\estimateur}[1]{\hat{\pi}_n^N(#1)}
\def\RichR{\operatorname{R}}
\def\piR{\hat{\pi}^{\RichR}}
\def\estimatorRR{\piR}
\newcommandx{\VarDeux}[3][3=]{\operatorname{Var}^{#3}_{#1}\left\{#2 \right\}}
\newcommand{\VarDeuxLigne}[2]{\operatorname{Var}_{#1}\{#2 \}}
\newcommand{\gramm}{\operatorname{Gramm}}
\newcommand{\1}{\mathbbm{1}}
\newcommand{\2}[1]{\mathbbm{1}_{\{#1\}}}




\newcommand{\LeftEqNo}{\let\veqno\@@leqno}

\newcommand{\lambdast}{\lambda^{s \rightarrow t}}
\newcommand{\etast}{\eta^{s \rightarrow t}}
\newcommand{\mst}{m^{s \rightarrow t}}
\newcommand{\mun}{m^{1 \rightarrow 2}}
\newcommand{\mdeux}{m^{2 \rightarrow 1}}
\newcommand{\lambdaun}{\lambda^{2 \rightarrow 1}}
\newcommand{\etaun}{\eta^{2 \rightarrow 1}}
\newcommand{\lambdadeux}{\lambda^{1 \rightarrow 2}}
\newcommand{\etadeux}{\eta^{1 \rightarrow 2}}
\newcommand{\mnun}{m^{n+1 \rightarrow \pi(n+1)}}
\newcommand{\etanun}{\eta^{n+1 \rightarrow \pi(n+1)}}
\newcommand{\lambdanun}{\lambda^{n+1 \rightarrow \pi(n+1)}}
\newcommand{\xpinun}{x_{\pi(n+1)}}
\newcommand{\xnun}{x_{n+1}}
\newcommand{\mpinun}{m^{\pi(n+1) \rightarrow n+1}}
\newcommand{\etapinun}{\eta^{\pi(n+1) \rightarrow n+1}}
\newcommand{\lambdapinun}{\lambda^{\pi(n+1) \rightarrow n+1}}
\newcommand{\pinun}{\pi(n+1)}
\newcommand{\vois}{\mathcal{N}}
\newcommand{\mpii}{m^{i \rightarrow \pi(n+1)}}
\newcommand{\etapii}{\eta^{i \rightarrow \pi(n+1)}}
\newcommand{\lambdapii}{\lambda^{i \rightarrow \pi(n+1)}}
\newcommand{\alphahat}{\widehat{\alpha}}
\newcommand{\betahat}{\widehat{\beta}}
\newcommand{\tildegamma}{\widetilde{\gamma}}
\newcommand{\tildeP}{\widetilde{P}}

\newcommand{\myeqref}[1]{Eq.~\eqref{#1}}



%%%% Floating Points Notation

\newcommand{\fpround}[1]{\lfloor #1 \rceil}
\newcommand{\floor}[1]{\left\lfloor #1 \right\rfloor}
\newcommand{\ceil}[1]{\left\lceil #1 \right\rceil}



%voc
\newcommand{\pth}{\ensuremath{p^{\text{th}}}}
\newcommand{\qth}{\ensuremath{q^{\text{th}}}}
\newcommand{\nth}{\ensuremath{n^{\text{th}}}}

%order
\newcommand{\ord}{\ensuremath{\operatorname{ord}}}
\newcommand{\rad}{\ensuremath{\operatorname{rad}}}



% Sets
\newcommand{\N}{\ensuremath{\mathbb{N}}}
\newcommand{\Q}{\ensuremath{\mathbb{Q}}}
\newcommand{\C}{\ensuremath{\mathbb{C}}}

%\newcommand{\F}{\ensuremath{\mathbb{F}}}
\newcommand{\primes}{\ensuremath{\mathcal P}}

\newcommand{\sfi}{\ensuremath{\mathcal{S}\!\mathcal{F}}}
\newcommand{\sfibt}{\ensuremath{\mathcal{S}\!\mathcal{F}'}}

\newcommand{\reghat}{\widehat{R}}

\newcommand{\reghatn}{\widehat{R}_n}

\newcommand{\arm}{\mathcal{A}}

%\newcommand{\mX}{\widehat{X}}
\newcommand{\PE}{\mathbb{E}}
\newcommand{\PP}{\mathbb{P}}
\newcommand{\Ft}{\mathcal{F}}

\newcommand{\Sy}{\mathbf{S}}

\newcommand{\Kfrac}{\mathscr{K}}

% Operands
\newcommand{\absolute}[1]{\left\vert #1 \right\vert}
\newcommand{\abs}[1]{\left\vert #1 \right\vert}
\newcommand{\absLigne}[1]{\vert #1 \vert}
\newcommand{\tvnorm}[1]{\| #1 \|_{\mathrm{TV}}}
\newcommand{\tvnormLigne}[1]{\| #1 \|_{\mathrm{TV}}}
\newcommand{\tvnormEq}[1]{\left \| #1 \right \|_{\mathrm{TV}}}
\newcommandx{\Vnorm}[2][1=V]{\| #2 \|_{#1}}
\newcommandx{\VnormEq}[2][1=V]{\left\| #2 \right\|_{#1}}
% \newcommandx{\norm}[2][1=]{\ifthenelse{\equal{#1}{}}{\left\Vert #2 \right\Vert}{\left\Vert #2 \right\Vert^{#1}}}
% \newcommandx{\normLigne}[2][1=]{\ifthenelse{\equal{#1}{}}{\Vert #2 \Vert}{\Vert #2\Vert^{#1}}}
\newcommand{\crochet}[1]{\left\langle#1 \right\rangle}
\newcommand{\parenthese}[1]{\left(#1 \right)}
\newcommand{\parentheseLigne}[1]{(#1 )}
\newcommand{\parentheseDeux}[1]{\left[ #1 \right]}
\newcommand{\parentheseDeuxLigne}[1]{[ #1 ]}
\newcommand{\defEns}[1]{\left\lbrace #1 \right\rbrace }
\newcommand{\defEnsLigne}[1]{\lbrace #1 \rbrace }
\newcommand{\defEnsPoint}[1]{\left\lbrace #1 \right. }
\newcommand{\defEnsPointDeux}[1]{\left. #1 \right  \rbrace }
\newcommand{\defEnsL}[1]{\left\lbrace #1 \right. }
\newcommand{\defEnsR}[1]{\left. #1 \right  \rbrace }

%\newcommand{\defSystem}[1]{\left\lbrace #1 \right. }

\newcommand{\ps}[2]{\left\langle#1,#2 \right\rangle}
\newcommand{\eqdef}{=}
\newcommand{\defeq}{=}

% Relations
\newcommand{\divid}{\mid}
\newcommand{\ndivide}{\nmid}

% Proba
\newcommand{\proba}[1]{\mathbb{P}\left( #1 \right)}
\newcommand{\probaCond}[2]{\mathbb{P}\left( \left. #1  \middle\vert #2 \right.\right)}
\newcommand{\probaCondLigne}[2]{\mathbb{P}(#1  \vert #2 )}
\newcommand{\probaCondLignePi}[2]{\Pi(#1  \vert #2 )}
\newcommand{\probaLigne}[1]{\mathbb{P}( #1 )}
\newcommandx\probaMarkovTilde[2][2=]
{\ifthenelse{\equal{#2}{}}{{\widetilde{\mathbb{P}}_{#1}}}{\widetilde{\mathbb{P}}_{#1}\left[ #2\right]}}
\newcommand{\probaMarkov}[2]{\mathbb{P}_{#1}\left[ #2\right]}
\newcommand{\probaMarkovDD}[1]{\mathbb{P}_{#1}}
\newcommand{\expe}[1]{\PE \left[ #1 \right]}
\newcommand{\expesq}[1]{\PE^{1/2} \left[ #1 \right]}
\newcommand{\expeExpo}[2]{\PE^{#1} \left[ #2 \right]}
\newcommand{\expeLigne}[1]{\PE [ #1 ]}
\newcommand{\expeLine}[1]{\PE [ #1 ]}
\newcommand{\expeMarkov}[2]{\PE_{#1} \left[ #2 \right]}
\newcommand{\expeMarkovD}[3]{\PE_{#1}^{#3} \left[ #2 \right]}
\newcommand{\expeMarkovDD}[1]{\PE_{#1}}
\newcommand{\expeMarkovLigne}[2]{\PE_{#1} [ #2 ]}
\newcommand{\expeMarkovExpo}[3]{\PE_{#1}^{#2} \left[ #3 \right]}
\newcommand{\probaMarkovTildeDeux}[2]{\widetilde{\mathbb{P}}_{#1} \left[ #2 \right]}
\newcommand{\expeMarkovTilde}[2]{\widetilde{\PE}_{#1} \left[ #2 \right]}

% Landau notation (big O)
\newcommand{\bigO}{\ensuremath{\mathcal O}}
\newcommand{\softO}{\Tilde{\ensuremath{\mathcal O}}}

% Environments

%\renewenvironment{proof}[1][{\textit{Proof:}}]{\begin{trivlist} \item[\em{\hskip \labelsep #1}]}{\ensuremath{\qed} \end{trivlist}}

%\renewenvironment{proof}[1][{\textit{Proof:}}]{\begin{trivlist} \item[\em{\hskip \labelsep #1}]}{\ensuremath{\qed} \end{trivlist}}



%fleche limite
\newcommand{\flecheLimite}{\underset{n\to+\infty}{\longrightarrow}}
\newcommand{\flecheLimiteOption}[2]{\underset{#1\to#2}{\longrightarrow}}
\newcommand{\flecheLimiteHaut}{\overset{n\to+\infty}{\longrightarrow}}


%notation infini
\newcommand{\plusinfty}{+\infty}

%notation egale
\newcommand{\egale}[1]{\ensuremath{\underset{#1}{=}}}

%plusieurs ligne indice
%\sum\limits_{\substack{i=0 \\ i \neq i_0}}^{n}{A_



\newcommand\numberthis{\addtocounter{equation}{1}\tag{\theequation}}


\newcommand{\hilbert}{\mathcal{H}}


\def\ie{\textit{i.e.}}
\def\as{\textit{a.s}}
\def\cadlag{càdlàg}
\def\eqsp{\;}
\newcommand{\coint}[1]{\left[#1\right)}
\newcommand{\ocint}[1]{\left(#1\right]}
\newcommand{\ooint}[1]{\left(#1\right)}
\newcommand{\ccint}[1]{\left[#1\right]}
\newcommand{\cointLigne}[1]{[#1)}
\newcommand{\ocintLigne}[1]{(#1]}
\newcommand{\oointLigne}[1]{(#1)}
\newcommand{\ccintLigne}[1]{[#1]}

\def\primr{f_r}
\def\primrO{f_{r_0}}




\newcommand{\indi}[1]{\1_{#1}}
\newcommandx{\weight}[2][2=n]{\omega_{#1,#2}^N}
\newcommand{\loi}{\mathcal{L}}
\newcommand{\boule}[2]{\operatorname{B}(#1,#2)}
\newcommand{\ball}[2]{\operatorname{B}(#1,#2)}
\newcommand{\boulefermee}[2]{\bar{B}(#1,#2)}
\newcommand{\cball}[2]{\bar{\operatorname{B}}(#1,#2)}
\newcommand{\diameter}{\operatorname{diam}}
\newcommand{\deta}{d_{\eta}}

\def\TV{\mathrm{TV}}

\newcommand{\george}[1]{\todo[color=orange!20]{{\bf GD:} #1}}
\newcommand{\james}[1]{\todo[color=blue!20]{{\bf JT:} #1}}
\newcommand{\arnaud}[1]{\todo[color=blue!20]{{\bf AD:} #1}}
\newcommand{\arnaudi}[1]{\todo[color=blue!20,inline]{{\bf AL:} #1}}
\newcommand{\valentin}[1]{\todo[color=blue!20]{{\bf VDB:} #1}}
\newcommand{\valentintxt}[1]{\textcolor{red}{\textbf{VDB}: #1}}
 \newcommand{\valentini}[1]{\todo[color=blue!20,inline]{{\bf VDB:} #1}}
 
\newcommand{\emile}[1]{\todo[color=red!20]{{\bf EM:} #1}}

\newcommand{\michael}[1]{\todo[color=green!20]{{\bf MJH:} #1}}
 
% \newcommand{\aymeric}[1]{\todo[color=blue!20]{{\bf AD:} #1}}
% \newcommand{\francis}[1]{\todo[color=black!20]{{\bf FB:} #1}}
 \newcommand{\tcr}[1]{\textcolor{red}{#1}}
% \newcommand{\tcb}[1]{\textcolor{blue}{#1}}


\def\as{\ensuremath{\text{a.s.}}}
\def\dist{\operatorname{dist}}

\newcommandx\sequence[3][2=,3=]
{\ifthenelse{\equal{#3}{}}{\ensuremath{\{ #1_{#2}\}}}{\ensuremath{\{ #1_{#2}, \eqsp #2 \in #3 \}}}}

\newcommandx\sequenceD[3][2=,3=]
{\ifthenelse{\equal{#3}{}}{\ensuremath{\{ #1_{#2}\}}}{\ensuremath{( #1)_{ #2 \in #3} }}}

\newcommandx{\sequencen}[2][2=n\in\N]{\ensuremath{\{ #1_n, \eqsp #2 \}}}
\newcommandx\sequenceDouble[4][3=,4=]
{\ifthenelse{\equal{#3}{}}{\ensuremath{\{ (#1_{#3},#2_{#3}) \}}}{\ensuremath{\{  (#1_{#3},#2_{#3}), \eqsp #3 \in #4 \}}}}
\newcommandx{\sequencenDouble}[3][3=n\in\N]{\ensuremath{\{ (#1_{n},#2_{n}), \eqsp #3 \}}}


\newcommand{\wrt}{w.r.t.}
\newcommand{\Withoutlog}{w.l.o.g.}
\def\iid{i.i.d.}
\def\ifof{if and only if}
\def\eg{\textit{e.g.}}


\newcommand{\notered}[1]{{\textbf{\color{red}#1}}}


\newcommand{\opnorm}[1]{{\left\vert\kern-0.25ex\left\vert\kern-0.25ex\left\vert #1
    \right\vert\kern-0.25ex\right\vert\kern-0.25ex\right\vert}}



\def\Lip{\operatorname{Lip}}
\def\Ltt{\mathtt{L}}
\def\generator{\mathcal{A}}
\def\generatorb{\bar{\mathcal{A}}}
\def\generatort{\tilde{\mathcal{A}}}
\def\generatorsp{\generator^{\sphere^d}}
\def\generatorr{\generator^{\rset^d}}

\def\momentNoise{\mathrm{m}}
\def\bfe{\mathbf{e}}

\def\bfv{\mathbf{v}}
\def\ebf{\mathbf{e}}
\def\vbf{\mathbf{v}}


\def\Id{\operatorname{Id}}
\def\Idbf{\mathbf{I}}

\def\tildetheta{\tilde{\theta}}

\def\calC{\mathcal{C}}


\newcommandx{\CPE}[3][1=]{{\mathbb E}_{#1}\left[#2 \middle \vert #3  \right]} %%%% esperance conditionnelle
\newcommandx{\CPELigne}[3][1=]{{\mathbb E}_{#1}[#2  \vert #3  ]} %%%% esperance conditionnelle
\newcommandx{\CPEsq}[3][1=]{{\mathbb{E}^{1/2}}_{#1}\left[#2 \middle \vert #3  \right]} %%%% esperance conditionnelle
\newcommandx{\CPVar}[3][1=]{\mathrm{Var}^{#3}_{#1}\left\{ #2 \right\}}
\newcommand{\CPP}[3][]
{\ifthenelse{\equal{#1}{}}{{\mathbb P}\left(\left. #2 \, \right| #3 \right)}{{\mathbb P}_{#1}\left(\left. #2 \, \right | #3 \right)}}

\def\Ascr{\mathscr{A}}
\def\scrA{\mathscr{A}}
\def\scrB{\mathscr{B}}
\def\scrC{\mathscr{C}}

\def\barL{\bar{L}}

\def\YL{\mathbf{Y}}
\def\XEM{X}
\def\steps{\gamma}
\def\measSet{\mathbb{M}}

%\newcommand\Ent[2]{\mathrm{Ent}_{#1}\left(#2\right)}
\newcommandx{\osc}[2][1=]{\mathrm{osc}_{#1}(#2)}

\def\Ybar{\bar{Y}}
\def\Id{\operatorname{Id}}
\def\IdM{\operatorname{I}_d}
\newcommand\EntDeux[2]{\Ent_{#1}\left[#2 \right]}
\def\Ltwo{\mathrm{L}^2}
\def\Lone{\mathrm{L}^1}
\newcommand\densityPi[1]{\frac{\rmd #1}{\rmd \pi}}
\newcommand\densityPiLigne[1]{\rmd #1 /\rmd \pi}
\newcommand\density[2]{\frac{\rmd #1}{\rmd #2}}
\newcommand\densityLigne[2]{\rmd #1/\rmd #2}

%\def\V{V}
\def\VD{V}
\def\Vsp{V^{\sphere^d}_{\b,\beta}}
\def\Vr{V^{\rset^d}_{\b,\c,\beta}}

\def\Prset{P^{\rset^d}}
\def\Psphere{P^{\sphere^d}}

\def\n{\mathrm{n}}
\def\Vpsi{\psi}
\def\Vkappa{\kappa}
\def\Vkappat{\tilde{\kappa}}
\def\Vchi{\chi}
\def\Vchit{\tilde{\chi}}
\def\Vphi{\phi}
\def\Vrho{\rho}
\def\psiV{\Vpsi}
\def\rhoV{\Vrho}
\def\phiV{\Vphi}
\def\fV{f}
\def\Vf{\fV}
\def\kappaVt{\tilde{\Vkappa}}
\def\kappaV{\Vkappa}
\def\chiV{\Vchi}
\def\chiVt{\Vchit}


\def\a{a}
\def\b{b}
\def\c{c}
\def\e{e}
\def\rU{\mathrm{r}}

\def\domain{\mathrm{D}}
\def\dom{\mathrm{dom}}

\def\martfg{M^{f,g}}
\newcommand\Ddir[1]{D_{#1}}
\newcommand\maxplus[1]{\parenthese{#1}_+}
\def\Refl{\mathrm{R}}
\def\phibf{\pmb{\phi}}
\def\Gammabf{\mathbf{\Gamma}}


\def\transpose{\top}
%\def\v{v}
\def\w{w}
\def\y{y}
\def\z{z}
%%%% bar
\def\bD{\bar{D}}
\def\bC{\bar{C}}
\def\brho{\bar{\rho}}
\def\bt{\bar{t}}
\def\bA{\bar{A}}
\def\bb{\overline{b}}
\def\bc{\bar{c}}
\def\bgamma{\bar{\gamma}}
\def\bU{\bar{U}}
\def\Ub{\bU}
\def\lambdab{\bar{\lambda}}
\def\blambda{\bar{\lambda}}
\def\blambdab{\bar{\lambda}}
\def\bv{\bar{v}}
\def\vb{\bv}
\def\yb{\bar{y}}
\def\by{\yb}
\def\Xb{\bar{X}}
\def\Yb{\bar{Y}}
\def\Gb{\bar{G}}
\def\Eb{\bar{E}}
\def\Tb{\bar{T}}
\def\taub{\bar{\tau}}

\def\bX{\bar{X}}
\def\bY{\bar{Y}}
\def\bG{\bar{G}}
\def\bE{\bar{E}}
\def\bT{\bar{T}}
\def\btau{\bar{\tau}}

\def\pib{\bar{\pi}}
\def\bpi{\pib}

\def\S{S}

%%%% tilde
\def\tgamma{\tilde{\gamma}}
\def\tC{\tilde{C}}
\def\tB{\tilde{B}}
\def\tc{\tilde{c}}
\def\tvareps{\tilde{\vareps}}
\def\trho{\tilde{\rho}}
\def\tmsk{\tilde{\msk}}
\def\tW{\tilde{W}}
\def\tvarsigma{\tilde{\varsigma}}
\def\tv{\tilde{v}}
\def\vt{\tv}
\def\yt{\tilde{y}}
\def\ty{\yt}
\def\Mt{\tilde{M}}
\def\tM{\Mt}

\def\tx{\tilde{x}}
\def\xt{\tx}
\def\Xt{\tilde{X}}
\def\Yt{\tilde{Y}}
\def\Gt{\tilde{G}}
\def\Et{\tilde{E}}
\def\Tt{\tilde{T}}
\def\St{\tilde{S}}
\def\taut{\tilde{\tau}}

\def\tX{\tilde{X}}
\def\tY{\tilde{Y}}
\def\tG{\tilde{G}}
\def\tE{\tilde{E}}
\def\tT{\tilde{T}}
\def\tS{\tilde{S}}
\def\ttau{\tilde{\tau}}


\def\Xb{\bar{X}}
\def\Yb{\bar{Y}}
\def\Gb{\bar{G}}
\def\Eb{\bar{E}}
\def\Tb{\bar{T}}
\def\Sb{\bar{S}}
\def\taub{\bar{\tau}}
\def\Hb{\bar{H}}
\def\Nb{\bar{N}}


\def\bX{\bar{X}}
\def\bY{\bar{Y}}
\def\bG{\bar{G}}
\def\bE{\bar{E}}
\def\bT{\bar{T}}
\def\btau{\bar{\tau}}
\def\bS{\bar{S}}
\def\bH{\bar{H}}
%\def\bN{\bar{N}}

%%%%%%%%

\def\mgU{\mathrm{m}_{\nabla U}}
\def\MintDrift{I}
\def\CU{C_U}
\def\RU{R_1}
\def\RV{R}
\def\Reps{R_{\epsilon}}
\def\Resp{\Reps}
\def\veps{\varepsilon}

\def\sphere{\mss}

\def\nablaUt{\overline{\nabla U}}
\def\measureSphere{\nu^d}

\def\etaU{\eta}
\def\epsilonU{\epsilon}

\def\Jac{\operatorname{Jac}}
\def\jac{\operatorname{Jac}}
\def\sign{\operatorname{sign}}
\def\rate{\lambda_{\mathrm{r}}}







\def\sigmaS{\sigma^2}

\newcommand{\ensemble}[2]{\left\{#1\,:\eqsp #2\right\}}
\newcommand{\ensembleLigne}[2]{\{#1\,:\eqsp #2\}}
\newcommand{\set}[2]{\ensemble{#1}{#2}}

\def\rmD{\mathrm{D}}%%rmd déjà pris
\def\mrd{\mathrm{D}}
\def\mrc{\mathrm{C}}

\def\diag{\Delta_{\rset^d}}

%\def\lyap{W}
\newcommand\coupling[2]{\Gamma(\mu,\nu)}
\def\supp{\mathrm{supp}}
\def\tpi{\tilde{\pi}}
\newcommand\adh[1]{\overline{#1}}

\def\ACb{\mathrm{AC}_{\mathrm{b}}}

\def\opK{\mathrm{K}}

\newcommand{\fracm}[2]{\left. #1 \middle / #2 \right.}
\newcommand{\fraca}[2]{ #1  / #2 }
\newcommand{\fracaa}[2]{ #1  / (#2) }

\newcommand{\complementary}{\mathrm{c}}

% \renewcommand{\geq}{\geqslant}
% \renewcommand{\leq}{\leqslant}
\def\poty{H}
% \def\diam{\mathrm{diam}}
\def\diam{\mathfrak{d}}
\def\talpha{\tilde{\alpha}}
% \def\Leb{\mathrm{Leb}}
\def\Leb{\lambda}
\newcommand{\iintD}[2]{\{#1,\ldots,#2\}}
\def\interior{\mathrm{int}}
\def\iff{ if and only if }

\def\vareps{\varepsilon}
\def\bvareps{\bar{\varepsilon}}
\def\varespilon{\varepsilon}
\def\si{\text{ if } }
\def\proj{\operatorname{proj}}
\def\projd{\operatorname{proj}^{\msd}}
\def\Phibf{\mathbf{\Phi}}
\def\Psibf{\mathbf{\Psi}}

\def\rker{\mathrm{R}}
\def\kker{\mathrm{K}}

\def\VEa{V}
\def\KUa{K}
\newcommandx{\KL}[2]{\operatorname{KL}\left( #1 | #2 \right)}
\newcommandx{\KLsqrt}[2]{\operatorname{KL}^{1/2}\left( #1 | #2 \right)}
\newcommandx{\Jef}[2]{\operatorname{J}\left( #1 , #2 \right)}
\newcommandx{\JefLigne}[2]{\operatorname{J}( #1 , #2 )}
\newcommandx{\KLLigne}[2]{\operatorname{KL}( #1 | #2 )}

\def\gaStep
\def\QKer{Q}
\def\Tg{\mathcal{T}_{\gamma}}
\def\Tk{\mathcal{T}_{k}}
\def\Tn{\mathcal{T}_{k}}
\def\Tnplusun{\mathcal{T}_{k+1}}
\def\mcurb{m}
%\newcommand{\coupling}[1]{\Gamma\left( #1 \right)}
\newcommand{\couplingLine}[1]{\Gamma( #1 )}
\def\distance{\mathbf{d}}
\newcommandx{\wasserstein}[3][1=\distance,3=]{\mathbf{W}_{#1}^{#3}\left(#2\right)}
\newcommandx{\wassersteinLigne}[3][1=\distance,3=]{\mathbf{W}_{#1}^{#3}(#2)}
\newcommandx{\wassersteinD}[1][1=\distance]{\mathbf{W}_{#1}}
\newcommandx{\wassersteinDLigne}[1][1=\distance]{\mathbf{W}_{#1}}


\def\Rcoupling{\mathrm{R}}
\def\Qcoupling{\mathrm{Q}}
\def\Sker{\mathrm{S}}
\def\Kcoupling{\mathrm{K}}
\def\tKcoupling{\tilde{\mathrm{K}}}
\def\Lcoupling{\mathrm{L}}
\def\Kcouplingproj{\mathrm{K}^P}
\def\vepsilon{\varepsilon}


\newcommand{\defEnsE}[2]{\ensemble{#1}{#2}}
\newcommand{\expeMarkovTildeD}[3]{\widetilde{\PE}_{#1}^{#3} \left[ #2 \right]}
\newcommand{\probaMarkovTildeD}[3]{\widetilde{\PP}_{#1}^{#3} \left[ #2 \right]}
\def\coordtildex{\mathrm{w}}
\def\PPtilde{\widetilde{\PP}}
\def\PEtilde{\widetilde{\PE}}
\def\transfrr{\mathrm{F}}
\def\diagSet{\Delta_{\msx}}
\def\Deltar{\diagSet}
\def\complem{\operatorname{c}}
\def\alphar{\alpha}
\def\tildex{\tilde{x}}
\def\tildez{\tilde{z}}
\def\tildey{\tilde{y}}
\def\ar{\mathrm{a}}
\def\Kr{\mathsf{K}}
\def\Kar{K^{(\mathrm{a})}}
\def\Xr{\mathrm{X}}
\def\Yr{\mathrm{Y}}
\def\Xrd{\mathit{X}}
\def\Yrd{\mathit{Y}}
\def\Zr{\mathrm{Z}}
\def\Ur{\mathrm{U}}
\def\sigmaD{\sigma^2}
\def\sigmakD{\sigma^2_k}
\newcommandx{\phibfs}[1][1=]{\pmb{\varphi}_{\sigmaD_{#1}}}
\def\vphibf{\pmb{\varphi}}
\def\varphibf{\pmb{\varphi}}
\def\phibfvs{\pmb{\varphi}_{\varsigma^2}}
\def\funreg{\mct}
\def\kappar{\varpi}
\def\Pr{\mathsf{P}}
\def\Par{P^{(\mathrm{a})}}
\def\Qr{\mathsf{Q}}
\def\Qar{Q^{(\mathrm{a})}}
\def\eventA{\msa}

\def\borelSet{\B}
\def\Er{\mathrm{E}}
\def\E{\mathbb{E}}
\def\er{\mathrm{e}}
\def\transp{\operatorname{T}}

\newcommandx\sequenceg[3][2=,3=]
{\ifthenelse{\equal{#3}{}}{\ensuremath{( #1_{#2})}}{\ensuremath{( #1_{#2})_{ #2 \geq #3}}}}


\def\indiar{\iota}
\def\rated{\chi}
\def\transar{\tau}
\def\filtrationTilde{\tilde{\mcf}}

\def\discrete{\mathrm{d}}
\def\continuous{\mathrm{c}}


\def\Xar{X^{(\mathrm{a})}}
\def\Yar{Y^{(\mathrm{a})}}
\def\War{W^{(\mathrm{a})}}
\def\Xiar{\Xi^{(\mathrm{a})}}
\def\mcfar{\mcf^{(\mathrm{a})}}

\def\Xart{\tilde{X}^{(\mathrm{a})}}
\def\Yart{\tilde{Y}^{(\mathrm{a})}}


\def\Kker{\Kcoupling}
\def\KkerD{\tilde{\Kcoupling}}
\def\Rker{\Rcoupling}
\def\tRker{\tilde{\Rker}}
\def\Pker{\mathrm{P}}
\def\Pkerf{\overrightarrow{\mathrm{P}}}
\def\Pkerfou{\overrightarrow{\mathrm{P}}_{\mathrm{OU}}}
\def\Pkerb{\overleftarrow{\mathrm{P}}}
\def\Rkerb{\overleftarrow{\mathrm{R}}}
\def\Skerb{\overleftarrow{\mathrm{S}}}
\def\Qker{\mathrm{Q}}
\def\Lker{\mathrm{L}}
\def\rmL{\mathrm{L}}
\def\rmG{\mathrm{G}}
\def\bfmu{\bm{\mu}}

\def\VlyapD{W}
\def\VlyapDun{W_1}
\def\VlyapDdeux{W_2}
\def\VlyapDtrois{W_3}
% \newcommandx{\distV}[1][1=W]{\mathbf{d}_{#1}}
\newcommandx{\distV}[1][1=\bfc]{\mathbf{W}_{#1}}
\newcommandx{\distVdeux}[1][1=W_2]{\mathbf{d}_{#1}}

\def\inv{\leftarrow}
\newcommand{\couplage}[2]{\Pi(#1,#2)}
\def\mtt{\mathtt{m}}
\def\mttzero{\mathtt{m}_0}
\def\tmtt{\tilde{\mathtt{m}}}
\def\ttm{\mathtt{m}}
\def\mttplus{\mathtt{m}^{+}}
\def\mttplusun{\mathtt{m}_1^{+}}
\def\mttplusdeux{\mathtt{m}_2^{+}}
\def\ttmplus{\mathtt{m}^{+}}
\def\cconst{\mathtt{a}}
\def\Run{R_1}
\def\Rdeux{R_2}
\def\Rtrois{R_3}
\def\Rquatre{R_4}
\def\tR{\tilde{R}}
\def\tmttplus{\tilde{\mtt}^+}
\newcommand{\tup}[1]{\textup{#1}}
\def\Fix{\operatorname{Fix}}
\newcommand{\stopping}[1]{\T_{\msc,\mathtt{n}_0}^{(#1)}}
\def\wass{\mathcal{W}}
\def\distY{\mathbf{d}}
\def\Xibf{\boldsymbol{\Xi}}
\def\rhomax{\rho_{\rm{max}}}
\def\rhof{\overrightarrow{\rho}}
\def\familydrift{\mathscr{B}}

\def\wasscun{\mathbf{W}_{\bfc_1}}
\def\wasscdeux{\mathbf{W}_{\bfc_2}}
\def\wassctrois{\mathbf{W}_{\bfc_3}}

\def\loiz{\mu_{\msz}}
\def\muz{\loiz}
\def\funH{H}

\renewcommand{\doteq}{=}
\newcommand{\Idd}{\operatorname{I}_d}


\def\driftb{b}
\def\Lttb{\mathtt{L}}

%\def\upsigma

%%% Local Variables:
%%% mode: latex
%%% TeX-master: "main
%%% End:


\usepackage{comment}
\usepackage{authblk}
\usepackage{cancel}

% \usepackage[commands]{MJH}
%  \usepackage{bibspacing}
\setlength{\bibsep}{2pt}
\makeatletter
\renewcommand\AB@affilsepx{, \protect\Affilfont}
\makeatother
%  \usepackage{showlabels}
\providecommand{\keywords}[1]
{
  \small	
  \textbf{\textit{Keywords---}} #1
}
% \hypersetup{colorlinks,citecolor=blue!50!black}
\newcommand{\appendixhead}{
  \centerline{\textbf{\LARGE Supplementary to: }\vspace{0.15in}}
  \centerline{\textbf{\LARGE Riemannian Score-Based Generative Modeling}\vspace{0.25in}}
  }
\usepackage{xcolor}
\colorlet{linkcolor}{blue!70!black}
\hypersetup{
  colorlinks,
  linkcolor={red!50!black},
  citecolor={blue!50!black},
  urlcolor={blue!80!black}
}
% \hypersetup{
%     colorlinks=true,       % false: boxed links; true: colored links
%     linkcolor=linkcolor,          % color of internal links (change box color with linkbordercolor)
%     citecolor=linkcolor,        % color of links to bibliography
%     filecolor=linkcolor,      % color of file links
%     urlcolor=linkcolor           % color of external links
% }
\usepackage[font={small}]{caption, subcaption}
\graphicspath{{images/}}

\title{Riemannian Score-Based Generative Modeling}

% The \author macro works with any number of authors. There are two commands
% used to separate the names and addresses of multiple authors: \And and \AND.
%
% Using \And between authors leaves it to LaTeX to determine where to break the
% lines. Using \AND forces a line break at that point. So, if LaTeX puts 3 of 4
% authors names on the first line, and the last on the second line, try using
% \AND instead of \And before the third author name.

% \author{George, Valentin and Arnaud}

\author{Valentin De Bortoli, Arnaud Doucet, Michael Hutchinson, \'Emile Mathieu, Yee Whye Teh, James Thornton}
\affil{Oxford University}
% \affil[1]{deligian@stats.ox.ac.uk}
% \affil[2]{valentin.debortoli@gmail.com}
% \affil[3]{doucet@stats.ox.ac.uk}


\begin{document}

\maketitle

\begin{abstract}
  \small
 Score-based generative models (SGMs) are a novel class of generative models demonstrating remarkable empirical performance. One uses a diffusion to add progressively Gaussian noise to the data, while the generative model is a ``denoising'' process obtained by approximating the time-reversal of this ``noising'' diffusion. However, current SGMs make the underlying assumption that the data is supported on a Euclidean manifold with flat geometry. This prevents the use of these models for applications in
  robotics, geoscience or protein modeling which rely on distributions defined
  on Riemannian manifolds. To overcome this issue, we introduce \emph{Riemannian
    Score-based Generative Models} (RSGMs) which extend current SGMs to the
   setting of compact Riemannian manifolds. %RGSMs rely on the extension of results on   time-reversal of diffusions to non-Euclidean geometry. 
We also show how RSGMs can be accelerated by solving a Schr\"odinger bridge problem on manifolds. We illustrate our
%   approach with synthetic examples on the sphere.
approach with earth and climate science data.
\end{abstract}
\keywords{Diffusion processes, Generative modeling, Riemannian manifold, Score-based generative models, Schr\"odinger bridge}

%\tableofcontents

\section{Introduction}
\label{sec:introduction}

Score-based Generative Modeling (SGM) is a recently developed approach to
generative modeling exhibiting state-of-the-art performances on various tasks
including image and audio synthesis
D\citep{song2019generative,song2020score,ho2020denoising,nichol2021improved,nichol2021beatgans}. These
models proceed as follows. We add noise to the data progressively using a
diffusion process targeting a reference Gaussian distribution. The corresponding
time-reversal process is also a diffusion whose drift depends on the logarithmic
gradients of the perturbed data distributions, i.e. the scores. The generative
model is obtained by approximating this time-reversal denoising diffusion by
initializing it at the reference Gaussian distribution and using neural networks
estimates of the scores obtained using score matching
\cite{hyvarinen2005estimation,vincent2011connection}. It can be shown rigorously
that the obtained final samples are approximately distributed according to the
data distribution \citep{debortoli2021neurips}.

Until now, SGM has been applied to Euclidean data, i.e. data with flat
geometry. However, in a large number of scientific domains, the underlying
assumption is that the distributions of interest are supported on a Riemannian
manifold. These include, amongst others, protein modeling
\citep{boomsma2008generative,hamelryck2006sampling,mardia2008multivariate,shapovalov2011smoothed,mardia2007protein},
cell development \citep{klimovskaia2020poincare}, image recognition
\citep{lui2012advances}, geological sciences
\citep{karpatne2018machine,peel2001fitting}, graph-structured and hierarchical
data \citep{roy2007learning,steyvers2005large}, robotics
\citep{feiten2013rigid,senanayake2018directional} and high-energy physics
\citep{brehmer2020flows}. The choice of a Riemannian metric is associated with a
description of the interactions between the points of the dataset and therefore
can be seen as a geometric prior.

In this paper we introduce \emph{Riemannian Score-based Generative Models}
(RSGM), an extension of SGMs to compact Riemannian manifolds. Contrary to
classical SGMs which rely on forward and time-reversed diffusion processes
defined on an Euclidean space, we incorporate the geometry of the data in our
algorithm by defining our diffusion processes directly on the Riemannian
manifold. However, switching from the classical Euclidean setting to the
Riemannian is non-trivial. First, one must be able to define a noising process
on the manifold that converges to an easy-to-sample reference distribution. In
the setting of compact Riemannian manifolds, a natural choice is given by the
Brownian motion. Indeed, due to the compactness, this diffusion is geometrically
ergodic and targets the uniform distribution on the manifold \citep{he2013lower}
from which one can either sample exactly or approximately with high
accuracy. Second, we must identify the corresponding time-reversal process. We
show here that, as in the Euclidean case, this process is also a diffusion whose
infinitesimal generator is given by the generator of the forward process with an
extra term corresponding to the scores of the marginal distributions of the
Brownian diffusion initialized at the data distribution. Third, while score
matching ideas \citep{hyvarinen2005estimation,vincent2011connection} can be
easily used to estimate the score in the Euclidean case when the forward
dynamics is given by a Ornstein--Ulhenbeck or a Brownian motion, adapting these
ideas to the Riemmanian framework is complicated by the fact that the heat
kernel, i.e. the transition kernel of the Brownian motion, is typically only
available as an infinite sum through the Sturm-Liouville
decomposition. Similarly, diffusions on manifold cannot be sampled
exactly. Hence, we use geodesic random walks which converge to the diffusion of
interest in the limit of small stepsizes \citep{jorgensen1975central}.

We further consider the following extensions of RSGMs. By using tools from
neural ODEs on manifolds
\citep{mathieu2020riemannian,falorsi2020neural,lou2020neural}, we show how we
can compute the likelihood of our model, generalizing the approach proposed in
the Euclidean case in
\citep{song2020score,durkan2021maximum,huang2021variational}. Finally, RGSMs
like standard SGMs are computationally expensive at generation time as they
require to run a discretized diffusion over many time steps. For speeding up
generation, it has been proposed in the Euclidean setting to solve instead a
Schr\"odinger Bridge (SB) problem
\citep{debortoli2021neurips,chen2021likelihood}, i.e. a dynamical version of
an entropy-regularized Optimal Transport (OT) problem between the data and the
easy-to-sample reference distribution. In particular, we generalize the
Diffusion Schr\"odinger Bridge (DSB) algorithm introduced in
\citep{debortoli2021neurips} to solve the SB problem on compact Riemmanian
manifolds.

% We also investigate the connection between SGMs and entropy-regularized Optimal Transport
% (OT) \citep{debortoli2021neurips,vargas2021solving,chen2021likelihood}. More
% precisely, we define the Schr\"odinger Bridge (SB) problem on Riemannian
% manifolds, which corresponds to a dynamical version of regularized OT. We solve
% this problem with a procedure akin to Diffusion Schr\"odinger Bridge (DSB)
% introduced in \citep{debortoli2021neurips}.  DSB allows to speed up
% significantly the sampling process in generative modeling and also permits to
% define diffusions interpolating between arbitrary distributions. We show that
% our Riemannian extension of DSB enjoys the same benefits.

We validate our methodology by modelling a number of natural disaster occurrence datasets collected by \cite{mathieu2020riemannian}. We compare to three previous baselines, a mixture of Kent distributions \cite{peel2001fitting}, Riemannian Continuous Normalising Flows \cite{mathieu2020riemannian}, and Moser Flows \cite{rozen2021moser}. We also compare to using a standard standard Euclidean SGM by projecting the manifold onto Euclidean space and performing the flow there (e.g. projecting the sphere via the stereographic projection onto the 2D plane). We find in all cases that RSGMs outperform all baselines.

% We validate our methodology \valentin{TO FILL}. (experiments)
% Code is available at ...

The rest of the paper is organized as follows. We introduce the notation needed
in the rest of the paper in \cref{sec:notation}. We recall the basics of
standard Euclidean SGMs in \Cref{sec:eucl-sgm-riem}. In
\Cref{sec:score-appr-manif}, we present RGSMs, our extension of SGMs to compact
Riemannian manifolds. We discuss related works in
\Cref{sec:related-works} and assess the efficiency of our method in
\Cref{sec:experiments}. % Finally, we present an extension of our work to
% Schr\"odinger bridges in \Cref{sec:extension} and
Finally we summarize our contributions in \Cref{sec:conclusion}.

\section{Notation}
\label{sec:notation}

We consider a compact connected Riemannian manifold
$(\M, \langle \cdot, \cdot \rangle_\M)$. We denote by $\XM$ the set of vector
fields on $\M$ and $\XMdeux$ the section
$\Gamma(\M, \sqcup_{x \in \M} \mathcal{L}(\mathrm{T}_x \M))$, where $\mathcal{L}(\mathrm{T}_x \M)$ is the space of linear mappings on
$\mathrm{T}_x \M$. Let $(\bfM_t)_{t \in \ccint{0,T}}$ be a real-valued process
and $(\bfX_t)_{t \in \ccint{0,T}}$ be a $\M$-valued process with distribution
$\Pbb \in \Pens(\rmc(\ccint{0,T}, \M))$.  $(\bfM_t)_{t \in \ccint{0,T}}$ is a
$\Pbb$-martingale if $(\bfM_t)_{t \in \ccint{0,T}}$ is a martingale w.r.t the
filtration $(\mcf_t)_{t \in \ccint{0,T}}$ where for any $t \in \ccint{0,T}$,
$\mcf_t = \sigma(\ensembleLigne{\bfX_s}{s \in \ccint{0,t}})$. In addition, for
any $\Pbb \in \Pens(\rmc(\ccint{0,T}, \M))$, we define $R(\Pbb)$ such that for
any $\msa \in \mcb{\rmc(\ccint{0,T}, \msx)}$ we have
$R(\Pbb)(\msa) = \Pbb(R(\msa))$, where
$R(\msa) = \ensembleLigne{t \mapsto \omega_{T-t}}{\omega \in \msa}$. In other
words, $R(\Pbb)$ is the path measure associated with the reverse process $\Pbb$.
When there is no ambiguity, we use the same notation for distributions and their
densities.

Let $T > 0$ or $T=+\infty$, $b: \ \ccint{0,T} \to \XM$, $\Sigma: \ \ccint{0,T} \to \XMdeux$
such that for any $t \in \ccint{0,T}$ and $x \in \M$, $\Sigma(t,x)$ is
symmetric, non-negative and denote $\sigma(t,x) = \Sigma^{1/2}(t,x)$. Let
$(\bfX_t)_{t \in \ccint{0,T}}$ a continuous process with distribution
$\Pbb \in \Pens(\rmc(\ccint{0,T}, \M))$ such that for any $f \in \rmc^2(\M)$ we
have that $(\bfM_t^{\bfX, f})_{t \in \ccint{0,T}}$ is a $\Pbb$-martingale where
for any $t \in \ccint{0,T}$
  \begin{equation}
    \textstyle{ \bfM_t^{\bfX, f} = f(\bfX_t) - \int_0^t \{ \langle b(s, \bfX_s), \nabla f(\bfX_s) \rangle_\M + (1/2) \langle \Sigma(\bfX_s),  \nabla^2 f(\bfX_s) \rangle_\M \} \rmd s  . }
  \end{equation}
  Then, we say that $(\bfX_t)_{t \in \ccint{0,T}}$ is \emph{associated with} the
  SDE $\rmd \bfX_t = b(t, \bfX_t) \rmd t + \sigma(t, \bfX_t) \rmd \bfB_t^\M$
  with infinitesimal generator
  $\generator: \ \ccint{0,T} \times \rmc^2(\M) \to \rmc(\M)$ given for any
  $t \in \ccint{0,T}$ by
  $\generator_t( f) = \langle b, \nabla f \rangle_\M + (1/2) \langle \Sigma,
  \nabla^2 f \rangle_\M$. Note that if $\Sigma = \Id$ then
  $\langle \Sigma, \nabla^2 f \rangle_\M = \Delta f$, where $\Delta$ is the
  Laplace-Beltrami operator.

%%% Local Variables:
%%% mode: latex
%%% TeX-master: "main"
%%% End:


\section{Euclidean Score-based Generative Modeling}
\label{sec:eucl-sgm-riem}

We recall here briefly the key concepts behind SGMs on the Euclidean space $\rset^d$ for some $d \in \nset$. We refer to \cite{song2020score,song2019generative,debortoli2021neurips} for a more detailed introduction to SGMs. In what follows, let $p_0$ denote the
data distribution. We have practically only access to an empirical approximation of this distribution given by the available data.


We consider a forward
\emph{noising} process $(\bfX_t)_{t \geq 0}$ defined by the following Stochastic
Differential Equation (SDE)
\begin{equation}\label{eq:forward_SDE}
  \rmd \bfX_t = -\bfX_t \rmd t + \sqrt{2} \rmd \bfB_t,\quad \bfX_0 \sim p_0 
\end{equation}
where $(\bfB_t)_{t \geq 0}$ is a $d$-dimensional Brownian motion. As a result
$(\bfX_t)_{t \geq 0}$ is an Ornstein--Ulhenbeck process targeting a multivariate standard Gaussian
distribution. Let $T \geq0$, under
mild conditions on the data distribution $p_0$, the time-reversed process
$(\bfhX_t)_{t \geq 0} = (\bfX_{T-t})_{t \in \ccint{0,T}}$ also satisfies an SDE
\citep{cattiaux2021time,haussmann1986time} given by
\begin{equation} \label{eq:backward_SDE}
  \rmd \bfhX_t = \{ \bfhX_t + 2 \nabla \log p_{T-t}(\bfhX_t)\} \rmd t + \sqrt{2} \rmd \bfB_t,\quad \bfhX_0 \sim p_T 
\end{equation}
where $p_t$ denotes the density of $\bfX_t$. By construction, the law of $\bfhX_{T-t}$ is equal to the law of $\bfX_t$ for $t \in \ccint{0,T}$ and in particular $\bfhX_{T}\sim p_0$. Hence, if one could sample from
$(\bfhX_t)_{t \in \ccint{0,T}}$ then its final distribution would be the target
data distribution $p_0$.  

Unfortunately there are three sources of intractability that prevents us from sampling the process $(\bfhX_t)_{t \in \ccint{0,T}}$. 

\textbf{Problem 1:} Its initial distribution is given by $p_T$ which is intractable.

\textbf{Solution:} The Ornstein--Ulhenbeck process \eqref{eq:forward_SDE} converges exponentially fast towards a standard multivariate Gaussian so one can approximate $p_T$ by this Gaussian for $T$ large enough. 

\textbf{Problem 2:} The scores are intractable so the dynamics \eqref{eq:backward_SDE} cannot be implemented. 

\textbf{Solution:} To approximate the scores, we exploit the following identity 
\begin{equation}\label{eq:scoreidentity}
  \textstyle{\nabla \log p_t(x) = \int_{\rset^d} \nabla \log p_{t|0}(x|x_0) p_{0|t}(x_0|x) \rmd x_0,}
\end{equation}
where $p_{t|0}(x'|x)$ is the transition density of the Ornstein--Ulhenbeck process which is available in closed-form. It follows directly that $\nabla \log p_t$ is the minimizer of the loss function
$\ell_t(s) = \expeLigne{\normLigne{s(\bfX_t) - \nabla \log
    p_{t|0}(\bfX_t|\bfX_0)}^2}$ over function $s$ where the expectation is over the joint distribution of $\bfX_0,\bfX_t$. This result can be exploited as follows. We consider a neural network approximation $\bm{s}_\theta: \ccint{0,T} \times \rset^d \to \rset^d$ which we train by minimizing the loss function $\ell(\theta)=\int_0^T \lambda_t \ell_{t}(\bm{s}_\theta(t,\cdot))\rmd t$ for some weighting function $\lambda_t>0$ . 


\textbf{Problem 3:} The loss function $\ell(\theta)$ and the SDE approximating \eqref{eq:backward_SDE} by replacing the scores $(\nabla \log p_t)_{t \in \ccint{0,T}}$ by $(s_\theta(t,\cdot))_{t \in \ccint{0,T}}$ and $p_T$ by the standard multivariate normal cannot not be simulated exactly on a computer.

\textbf{Solution:} For a discretization step $\gamma$ such that $T=\gamma N$ for integer $N$, the loss function is approximated by $\sum_{n=0}^N \lambda_{n\gamma} \ell_{n \gamma}(\bm{s}_\theta(n \gamma,\cdot))$ and we perform a Euler--Maruyama discretization of the resulting SDEe; i.e. we define $(Y_n)_{n \in \{0, \dots, N\}}$ such that for $Z_n\overset{\textup{i.i.d.}}{\sim} \mathcal{N}(0,I_d)$
% \begin{equation}
%   \label{eq:backward_discreteexactscores}
%   Y_{n+1} = Y_n + \gamma \{Y_n + 2 \nabla \log p_{T -n \gamma}(Y_n) \} + \sqrt{2} Z_{n+1},\quad Y_0\sim \mathcal{N}(0,I_d)
% \end{equation}
% for any $n \in \{0, \dots, N-1\}$ where $\mathcal{N}(0,I_d)$ denotes the multivariate standard normal on $\mathbb{R}^d$,  $\gamma > 0$ such that $T = N \gamma$, $(Z_n)\overset{\textup{i.i.d.}}{\sim} \mathcal{N}(0,I_d)$.
    
    
    
%     Finally, at sampling times we consider the following dynamics
\begin{equation}\label{eq:backward_discrete_final}
  Y_{n+1} = Y_n + \gamma \{Y_n + 2  \bm{s}_\theta(T -n \gamma, Y_n) \} + \sqrt{2} Z_{n+1},\quad Y_0\sim \mathcal{N}(0,I_d).
\end{equation}




% \textbf{Problem 2:} The continuous-time process \eqref{eq:backward_SDE} cannot not be simulated exactly on a computer even if the scores $(\nabla \log p_t)_{t \in \ccint{0,T}}$ were known. 

% \textbf{Solution:} We perform a time-discretization of the resulting SDE using an Euler--Maruyama scheme; i.e. we define $(Y_n)_{n \in \{0, \dots, N\}}$ such that
% \begin{equation}
%   \label{eq:backward_discreteexactscores}
%   Y_{n+1} = Y_n + \gamma \{Y_n + 2 \nabla \log p_{T -n \gamma}(Y_n) \} + \sqrt{2} Z_{n+1},\quad Y_0\sim \mathcal{N}(0,I_d)
% \end{equation}
% for any $n \in \{0, \dots, N-1\}$ where $\mathcal{N}(0,I_d)$ denotes the multivariate standard normal on $\mathbb{R}^d$,  $\gamma > 0$ such that $T = N \gamma$, $(Z_n)\overset{\textup{i.i.d.}}{\sim} \mathcal{N}(0,I_d)$.






% First, its initial distribution is given by $p_T$ which is intractable. Second, the continuous-time process \eqref{eq:backward_SDE} cannot not be simulated exactly on a computer even if the scores $(\nabla \log p_t)_{t \in \ccint{0,T}}$ were known. Third, the scores are also intractable. \emile{repetition with following paragraph} 

% Unfortunately there are three sources of intractability that prevents us from sampling the process $(\bfhX_t)_{t \in \ccint{0,T}}$. First, its initial distribution is given by $p_T$ which is intractable. However, the
% Ornstein--Ulhenbeck process \eqref{eq:forward_SDE} converges exponentially fast towards a standard multivariate Gaussian so one can approximate $p_t$ by this Gaussian for $T$ large enough. Second, the continuous-time process \eqref{eq:backward_SDE} cannot not be simulated exactly on a computer even if the scores $(\nabla \log p_t)_{t \in \ccint{0,T}}$ were known. 
% We then perform a time-discretization of the resulting SDE using an Euler--Maruyama scheme; i.e. we define $(Y_n)_{n \in \{0, \dots, N\}}$ such that
% \begin{equation}
%   \label{eq:backward_discreteexactscores}
%   Y_{n+1} = Y_n + \gamma \{Y_n + 2 \nabla \log p_{T -n \gamma}(Y_n) \} + \sqrt{2} Z_{n+1},\quad Y_0\sim \mathcal{N}(0,I_d)
% \end{equation}
% for any $n \in \{0, \dots, N-1\}$ where $\mathcal{N}(0,I_d)$ denotes the multivariate standard normal on $\mathbb{R}^d$,  $\gamma > 0$ such that $T = N \gamma$, $(Z_n)\overset{\textup{i.i.d.}}{\sim} \mathcal{N}(0,I_d)$.
% Third, the scores are also intractable so the discrete-time process \eqref{eq:backward_discreteexactscores} cannot be implemented either. To approximate the scores, we exploit the following identity 
% \begin{equation}\label{eq:scoreidentity}
%   \textstyle{\nabla \log p_t(x) = \int_{\rset^d} \nabla \log p_{t|0}(x|x_0) p_{0|t}(x_0|x) \rmd x_0,}
% \end{equation}
% where $p_{t|0}(x'|x)$ is the transition density of the Ornstein--Ulhenbeck process which is available in closed-form. It follows directly that $\nabla \log p_t$ is the minimizer of the loss function
% $\ell_t(s) = \expeLigne{\normLigne{s(\bfX_t) - \nabla \log
%     p_{t|0}(\bfX_t|\bfX_0)}^2}$ over function $s$ where the expectation is over the joint distribution of $\bfX_0,\bfX_t$. In practice, we thus consider consider a neural network approximation
% % $\bm{s}_\theta: \big\{0,...,N-1\big\} \times \rset^d \to \rset^d$ 
% $\bm{s}_\theta: [0, T] \times \rset^d \to \rset^d$ 
% which we train by minimizing over $\theta$ the loss 
% % $\sum_{n=0}^{N-1} \lambda_n \ell_{n \gamma}(\bm{s}^\theta(n \gamma, \cdot))$ for some positive weights $\lambda_n>0$ in a preliminary training phase.
% $ \mathbb{E}_{t}\left[\lambda(t) \ell_{t}(\bm{s}_\theta(t, \cdot))\right]$ where $t \sim \mathcal{U}([0,T])$ and $\lambda: [0, T] \rightarrow \R^{+}$ is a positive weighting function.
% Finally, at sampling times we consider the following dynamics
% \begin{equation}\label{eq:backward_discrete_final}
%   Y_{n+1} = Y_n + \gamma \{Y_n + 2  s_\theta(T -n \gamma, Y_n) \} + \sqrt{2} Z_{n+1},\quad Y_0\sim \mathcal{N}(0,I_d).
% \end{equation}

% \michael{Just playing with layout options}
% %% Testing enumerate
% Unfortunately there are three sources of intractability that prevents us from sampling the process $(\bfhX_t)_{t \in \ccint{0,T}}$. 

% \begin{enumerate}
%     \item Its initial distribution is given by $p_T$ which is intractable. 
%     However, the Ornstein--Ulhenbeck process \eqref{eq:forward_SDE} converges exponentially fast towards a standard multivariate Gaussian so one can approximate $p_t$ by this Gaussian for $T$ large enough. 
%     \item The continuous-time process \eqref{eq:backward_SDE} cannot not be simulated exactly on a computer even if the scores $(\nabla \log p_t)_{t \in \ccint{0,T}}$ were known. 
%     We then perform a time-discretization of the resulting SDE using an Euler--Maruyama scheme; i.e. we define $(Y_n)_{n \in \{0, \dots, N\}}$ such that
%     \begin{equation}
%       \label{eq:backward_discreteexactscores}
%       Y_{n+1} = Y_n + \gamma \{Y_n + 2 \nabla \log p_{T -n \gamma}(Y_n) \} + \sqrt{2} Z_{n+1},\quad Y_0\sim \mathcal{N}(0,I_d)
%     \end{equation}
%     for any $n \in \{0, \dots, N-1\}$ where $\mathcal{N}(0,I_d)$ denotes the multivariate standard normal on $\mathbb{R}^d$,  $\gamma > 0$ such that $T = N \gamma$, $(Z_n)\overset{\textup{i.i.d.}}{\sim} \mathcal{N}(0,I_d)$.
%     \item The scores are also intractable so the discrete-time process \eqref{eq:backward_discreteexactscores} cannot be implemented either.
%     To approximate the scores, we exploit the following identity 
%     \begin{equation}\label{eq:scoreidentity}
%       \textstyle{\nabla \log p_t(x) = \int_{\rset^d} \nabla \log p_{t|0}(x|x_0) p_{0|t}(x_0|x) \rmd x_0,}
%     \end{equation}
%     where $p_{t|0}(x'|x)$ is the transition density of the Ornstein--Ulhenbeck process which is available in closed-form. It follows directly that $\nabla \log p_t$ is the minimizer of the loss function
%     $\ell_t(s) = \expeLigne{\normLigne{s(\bfX_t) - \nabla \log
%         p_{t|0}(\bfX_t|\bfX_0)}^2}$ over function $s$ where the expectation is over the joint distribution of $\bfX_0,\bfX_t$. In practice, we thus consider consider a neural network approximation
%     $\bm{s}_\theta: \big\{0,...,N-1\big\} \times \rset^d \to \rset^d$. 
%       We train $\bm{s}_\theta$ by minimizing over $\theta$
%     $\sum_{n=0}^{N-1} \lambda_n \ell_{n \gamma}(\bm{s}^\theta(n \gamma, \cdot))$ for some positive weights $\lambda_n>0$ in a preliminary training phase. Finally, at sampling times we consider the following dynamics
%     \begin{equation}\label{eq:backward_discrete_final}
%       Y_{n+1} = Y_n + \gamma \{Y_n + 2  s^\theta_{T -n \gamma}(Y_n) \} + \sqrt{2} Z_{n+1},\quad Y_0\sim \mathcal{N}(0,I_d).
%     \end{equation}
% \end{enumerate}
% %% Testing enumerate

% %% Testing problem-solution
% Unfortunately there are three sources of intractability that prevents us from sampling the process $(\bfhX_t)_{t \in \ccint{0,T}}$. 

% \textbf{Problem 1:} Its initial distribution is given by $p_T$ which is intractable.

% \textbf{Solution:} The Ornstein--Ulhenbeck process \eqref{eq:forward_SDE} converges exponentially fast towards a standard multivariate Gaussian so one can approximate $p_t$ by this Gaussian for $T$ large enough. 


% \textbf{Problem 2:} The continuous-time process \eqref{eq:backward_SDE} cannot not be simulated exactly on a computer even if the scores $(\nabla \log p_t)_{t \in \ccint{0,T}}$ were known. 

% \textbf{Solution:} We perform a time-discretization of the resulting SDE using an Euler--Maruyama scheme; i.e. we define $(Y_n)_{n \in \{0, \dots, N\}}$ such that
% \begin{equation}
%   \label{eq:backward_discreteexactscores}
%   Y_{n+1} = Y_n + \gamma \{Y_n + 2 \nabla \log p_{T -n \gamma}(Y_n) \} + \sqrt{2} Z_{n+1},\quad Y_0\sim \mathcal{N}(0,I_d)
% \end{equation}
% for any $n \in \{0, \dots, N-1\}$ where $\mathcal{N}(0,I_d)$ denotes the multivariate standard normal on $\mathbb{R}^d$,  $\gamma > 0$ such that $T = N \gamma$, $(Z_n)\overset{\textup{i.i.d.}}{\sim} \mathcal{N}(0,I_d)$.

% \textbf{Problem 3:} The scores are also intractable so the discrete-time process \eqref{eq:backward_discreteexactscores} cannot be implemented either. 

% \textbf{Solution:} To approximate the scores, we exploit the following identity 
% \begin{equation}\label{eq:scoreidentity}
%   \textstyle{\nabla \log p_t(x) = \int_{\rset^d} \nabla \log p_{t|0}(x|x_0) p_{0|t}(x_0|x) \rmd x_0,}
% \end{equation}
% where $p_{t|0}(x'|x)$ is the transition density of the Ornstein--Ulhenbeck process which is available in closed-form. It follows directly that $\nabla \log p_t$ is the minimizer of the loss function
% $\ell_t(s) = \expeLigne{\normLigne{s(\bfX_t) - \nabla \log
%     p_{t|0}(\bfX_t|\bfX_0)}^2}$ over function $s$ where the expectation is over the joint distribution of $\bfX_0,\bfX_t$. In practice, we thus consider consider a neural network approximation
% $\bm{s}_\theta: \big\{0,...,N-1\big\} \times \rset^d \to \rset^d$. 
%   We train $\bm{s}_\theta$ by minimizing over $\theta$
% $\sum_{n=0}^{N-1} \lambda_n \ell_{n \gamma}(\bm{s}^\theta(n \gamma, \cdot))$ for some positive weights $\lambda_n>0$ in a preliminary training phase. Finally, at sampling times we consider the following dynamics
% \begin{equation}\label{eq:backward_discrete_final}
%   Y_{n+1} = Y_n + \gamma \{Y_n + 2  s^\theta_{T -n \gamma}(Y_n) \} + \sqrt{2} Z_{n+1},\quad Y_0\sim \mathcal{N}(0,I_d).
% \end{equation}
%% testing problem-solution

% Hence, we consider
% the process $(\bfY_t)_{t \in \ccint{0,T}}$ such that $\bfY_0$ is a Gaussian
% random variable with zero mean and identity covariance matrix and
% $(\bfY_t)_{t \in \ccint{0,T}}$ satisfies \eqref{eq:backward_SDE}. In order to
% obtain an algorithm which can be implemented in practice we first discretize the
% process \eqref{eq:backward_SDE}, \ie \ we define $(Y_n)_{n \in \{0, \dots, N\}}$
% such that $Y_0$ is a Gaussian random variable with zero mean and identity
% covariance matrix and for any $n \in \{0, \dots, N-1\}$ we have
% \begin{equation}
%   \label{eq:backward_discrete}
%   Y_{n+1} = Y_n + \gamma \{Y_n + 2 \nabla \log p_{T -n \gamma}(Y_n) \} + \sqrt{2} Z_{n+1} \eqsp ,
% \end{equation}
% where $\gamma > 0$ such that $T = N \gamma$, $(Z_n)_{n \in \nset}$ is a sequence
% of i.i.d. Gaussian random variables with zero mean and identity covariance
% matrix. Note that \eqref{eq:backward_discrete} is simply the Euler-Maruyama
% discretization of \eqref{eq:backward_SDE}. One last key step relies in the
% approximation of the dynamics of \eqref{eq:backward_discrete} since the
% logarithmic gradient (or Stein score) $(\nabla \log p_t)_{t \in \ccint{0,T}}$ is
% not tractable. To do so, we consider a neural network approximation
% $\bm{s}_\theta: \ \ccint{0,T} \times \rset^d \to \rset^d$. Since for any
% $t \in \ccint{0,T}$ and $x \in \rset^d$ we have that
% \begin{equation}
%   \textstyle{\nabla \log p_t(x) = \int_{\rset^d} \nabla \log p_{t|0}(x|x_0) p_{0|t}(x_0|x) \rmd x_0 \eqsp , }
% \end{equation}
% we obtain that for any $t \in \ccint{0,T}$, $\nabla \log p_t$ is the minimizer
% of the loss function
% $\ell_t(s) = \expeLigne{\normLigne{s(\bfX_t) - \nabla \log
%     p_{t|0}(\bfX_t|\bfX_0)}^2}$. We train $\bm{s}_\theta$ to minimize
% $\int_0^T \lambda(t) \ell_t(\bm{s}_\theta(t, \cdot)) \rmd t$, where $t \mapsto \lambda(t)$ is some positive weighting function.
% \emile{Would remove from main paper}
% In \citet[Theorem 1]{debortoli2021neurips}, error bounds on
% the total variational norm between the law of $Y_N$ and $p_0$ are established depending on $N$, $\gamma$
% and the approximation error of the neural network $\bm{s}_\theta$. In
% particular, the approximation error can be decomposed into two terms: one corresponding to the mixing of the process  \eqref{eq:backward_discreteexactscores} which is controlled by the mixing of the forward process \eqref{eq:forward_SDE}, another one which corresponds to the approximation of the scores. %Hence the bottleneck of SGMs are not mixing issues does not reside in the mixing time of the backward chain but in the approximation of the score.

We have presented the basics of SGM but we highlight that many recent works
improve on these models;
\citep[see e.g.][]{song2020score,song2020improved,song2020denoising,jolicoeur2020adversarial,jolicoeur2021gotta,nichol2021beatgans}. In
particular, it is worth noting that choosing an adaptive stepsize
$(\gamma_n)_{n \in \nset}$ \citep{bao2022analyticdpm,watson2021learning} drastically improve the
synthesis results as well as using a predictor-corrector scheme
\citep{song2020score} instead of a simple Euler--Maruyama discretization. Finally,
we note that there exist other approaches to introduce SGMs using variational and maximum likelihood
techniques \citep{ho2020denoising,huang2021variational,durkan2021maximum}.


%%% Local Variables:
%%% mode: latex
%%% TeX-master: "main"
%%% End:


\section{Riemannian Score-based Generative Modeling}
\label{sec:score-appr-manif}

Similarly to the Euclidean setting, three ingredients are required to extend SGM
to compact Riemannian manifolds:
\begin{enumerate*}[label=\roman*)]
\item a forward \emph{noising} process on the Riemannian manifold which converges to an easy-to-sample reference distribution, 
\item a time-reversal formula on Riemannian manifolds which defines a backward generative process, 
\item a method to efficiently approximate the drift of the time-reversal process.
\end{enumerate*}
We address all these problems and introduce RGSM. The key differences between
SGMs and RSGMs are summarised in \cref{tab:difference}.


\begin{table}[h]
\small
\centering
\renewcommand*{\arraystretch}{1.2}
\begin{tabular}{lcc}
%   \toprule 
  Ingredient \textbackslash ~Space  &       Euclidean                & Compact manifold \\ \hline
  Forward process & Ornstein--Ulhenbeck & Brownian motion \\
  Easy-to-sample distribution & Gaussian & Uniform \\
  Time reversal  &  \citet[Theorem 4.9]{cattiaux2021time} & \cref{thm:time_reversal_manifold}  \\   
  Sampling of the forward process & Direct & Geodesic Random Walk \\
  Sampling of the backward process & Euler--Maruyama & Geodesic Random Walk \\
%   \bottomrule
\end{tabular}
\caption{\small Differences between SGM on Euclidean spaces and RSGM on compact Riemannian manifolds.}
\label{tab:difference}
\end{table}



\subsection{Brownian motion on compact Riemannian manifolds}
\label{sec:brown-moti-comp}

\paragraph{Brownian motion and uniform distribution}

First, we define a forward noising process on $\M$ targeting an easy-to-sample
reference distribution. In Euclidean spaces, the reference distribution is a
standard normal in the compact manifold setting the uniform distribution
$\piinv$ is the natural choice.  For most manifolds of interest, one can either
sample exactly from $\piinv$ or obtain approximate samples with high accuracy.
For the forward noising dynamics, the Ornstein--Ulhenbeck process
\eqref{eq:forward_SDE} used in Euclidean scenarios is now replaced by the
Brownian motion defined on $\M$ as it converges exponentially fast to
$\piinv$---see \Cref{prop:brownian_conv} below. We refer to
\Cref{sec:brown-moti-manif} for a general introduction to Brownian motions on
manifolds. This Brownian motion is defined as follows.

\begin{definition}[Brownian motion]
  $(\bfB^\M_t)_{t \geq 0}$ is a Brownian motion on $\M$ if
  $(\bfB^\M_t)_{t \geq 0}$ is associated with the SDE with infinitesimal
  generator $\generator(f) = \Delta f$, see \cref{sec:notation}.
\end{definition}

We refer to \cref{sec:brown-moti-manif} or \citet[Chapter 1, Chapter
3]{hsu2002stochastic} for the definition of a $\M$-valued semimartingale and the
Laplace-Beltrami operator. By \citet[Proposition 3.2.1]{hsu2002stochastic}, we
have that for any initial condition $\bfB^\M_0$ with distribution
$\pizero \in \Pens(\M)$, there exists $(\bfB^\M_t)_{t \geq 0}$. The convergence
rates are obtained w.r.t. the total variation distance between the uniform
distribution and the semi-group $(\Pker_t)_{t \geq 0}$\footnote{We define
  $(\Pker_{t,s})_{t, s \geq 0, t \geq s}$ the semi-group such that for any
  $f,g \in \rmc(\M)$ and $t, s \geq 0$ with $t \geq s$ we have
  $\expeLigne{f(\bfB_t^\M)g(\bfB_s^\M)} = \expeLigne{\int_{\M}f(y)
    \Pker_{t|s}(\bfB_s^\M, \rmd y) g(\bfB_s^\M)}$. For the rest of this paper we
  denote $(\Pker_t)_{t \geq 0} = (\Pker_{t|0})_{t \geq 0}$.}
%, see \citet[Proposition 2.6]{urakawa2006convergence}.

\begin{proposition}[{{Convergence of Brownian motion \cite[Proposition 2.6]{urakawa2006convergence}}}]
  \label{prop:brownian_conv}
  For any $t > 0$, $\Pker_t$ admits a density $p_t$ w.r.t $\piinv$ and
  $\piinv \Pker_t = \piinv$, \ie \ $\piinv$ is an invariant measure for
  $(\Pker_t)_{t \geq 0}$. In addition, if there exists $C, \alpha \geq 0$ such
  that for any $t \in \ocint{0,1}$, $p_t(x,x) \leq C t^{-\alpha /2}$ then 
  for any $\pizero \in \Pens(\M)$ and for any $t \geq 1/2$ we have 
  \begin{equation}
    \textstyle{\tvnorm{\pizero \Pker_t - \piinv} \leq C^{1/2} \rme^{\lambda_1 /2} \rme^{-\lambda_1 t}  ,}
  \end{equation}
  where $\lambda_1$ is the first non-negative eigenvalue of $-\Delta_\M$  in $\mathrm{L}^2(\piinv)$.
\end{proposition}

The diagonal upper bound on the heat kernel is satisfied for many manifolds
including the $d$-dimensional torus and sphere \cite[see][Section
3]{saloff1994precise}. Hence, \cref{prop:brownian_conv} ensures that under mild
conditions the Brownian motion converges exponentially fast towards the uniform
distribution on the compact Riemmanian manifold $\M$. Therefore, in the context
of SGM, the Brownian motion on $\M$ is the counterpart to the Ornstein--Ulhenbeck
process and the uniform distribution is the counterpart to the Gaussian one.

We note that in previous works on SGMs, the Brownian motion has also been used
as a forward noising process \citep{song2020score,song2019generative}. However,
in these cases, the Brownian motion is not geometrically ergodic and does not
admit any invariant distribution contrary to our setting. Two issues remain to
be solved. First, we need to be able to sample this forward noising process
$(\bfB_t^\M)_{t \geq 0}$. Second we need to obtain tractable approximations of
the heat kernel, i.e. the transition kernel of this process, in order to define
efficient score approximation schemes.

\paragraph{Sampling from diffusions}
In Euclidean spaces, sampling an Ornstein--Ulhenbeck process is straightforward
whereas obtaining samples from a Brownian motion on a manifold is non-trivial in
general. First, if $\M$ is isometrically embedded into $\rset^p$ (with
$p \geq d$)---i.e.\ $\M \subset \rset^p$---then we have that
$(\bfB^\M_t)_{t \geq 0}$ (seen as a process on the ambient space $\rset^p$)
satisfies the following SDE
\begin{equation}
  \label{eq:brownian_motion_extrinsic}
  \textstyle{\rmd \bfB^\M_t = \sum_{i=1}^p P_i(\bfB^\M_t) \circ \rmd \bfB_t^i,}
\end{equation}
where $\circ$ denotes the Stratanovitch integral \footnote{Manifold valued processes are usually defined using the Stratanovitch integral because it satisfies the chain rule of differential calculus. For more details we refer to \cref{sec:stoch-diff-equat}.},
$(\{\bfB_t^i\}_{i=1}^p)_{t \geq 0}$ is a $p$-dimensional Brownian motion and for
any $i \in \{1, \dots, p\}$ we have $P_i(x) = P(x) e_i$ for any $x \in \M$, where $\{e_i\}_{i=1}^p$ is the canonical basis of $\rset^p$
and $P(x): \ \rset^p \to \mathrm{T}_x \M$ is the orthogonal projection operator,
see \cref{sec:metr-conn-tens}. However, this approach is \emph{extrinsic} and
requires the knowledge of the projection
operator. % It is also limited to the Brownian motion and it is not
% easily extended to other diffusions on $\M$.
Here we consider an \emph{intrisic} 
approach based on Geodesic Random Walks (GRWs), see \cite{jorgensen1975central}
for a review of their properties.

GRWs are not restricted to approximating the Brownian motion on $\M$ but in fact
can approximate \emph{any} well-behaved diffusion on $\M$. This property will be
useful when sampling the backward diffusion process. Hence, we introduce GRWs in
a general framework and we are going to consider a discrete-time process
$(X_n^\gamma)_{n \in \nset}$ which approximates $(\bfX_t)_{t \geq 0}$
is associated with
\begin{equation}
  \label{eq:generic}
 \rmd \bfX_t = b(\bfX_t) \rmd t + \sigma(\bfX_t) \rmd \bfB_t^\M.
\end{equation}
Let
$\{ \nu_x \}_{x \in \M}$ such that for any $x \in \M$,
$\nu_x \in \Pens(\mathrm{T}_x \M)$. Assume that for any $x \in \M$,
$\int_{\M} \normLigne{v}^2 \rmd \nu_x(v)< +\infty$. In addition assume that
there exists $b \in \XM$ and $\Sigma \in \XMdeux$, such that for any $x \in \M$,
$\int_{\M} v \rmd \nu_x(v) = b(x)$ and
$\int_{\M} (v - b(x)) \otimes (v - b(x)) \rmd \nu_x(v) =
\Sigma(x) = \sigma(x) \sigma(x)^\top$. % In addition, we assume
    % that for any $x \in \M$, $\Sigma(x) = \mu^{(2)}(x) - \mu^{(1)}(x) \otimes \mu^{(1)}(x)$ is
    % strictly positive definite and that there exists $\Ltt \geq$ such that for
    % any $x, y \in \M$, $\tvnorm{\nu_x - \nu_y} \leq \Ltt d(x,y)$. Where we have
    % that for any $\nu_1 \in \mathrm{T}_x \M$ and $\nu_2 \in \mathrm{T}_y \M$,
    % \begin{equation}
    %   \tvnorm{\nu_x - \nu_y} = \sup \ensembleLigne{\nu_1[f] - \Gamma_{x}^y(\gamma)_\# \nu_2[f]}{\gamma \in \mathrm{Geo}_{x,y}, \ f \in \rmc(\mathrm{T}_x \M)}  . 
    % \end{equation}
    % Note that if $d(x,y) \leq \vareps$ then for some $\vareps > 0$ we have that $\abs{\mathrm{Geo}_{x,y}}=1$.


\begin{definition}[Geodesic Random Walk]
  Let $X_0$ be a $\M$-valued random variable.  For any $\gamma > 0$, we
  define $(\bfX_t^{\gamma})_{t \geq 0}$ such that $\bfX_0^\gamma = X_0$ and
  for any $n \in \nset$ and $t \in \ccint{0, \gamma}$,
  $\bfX_{n\gamma + t} = \exp_{\bfX_{n \gamma}}\left(t\gamma \{ \mu_n +
  (1/\sqrt{\gamma}) (V_n - \mu_n)\}\right)$\footnote{where $\exp_x: \ \mathrm{T}_x \M \to \M$ is the exponential mapping on the manifold, see \citet[Chapter 20]{lee2013smooth} for details.}, where $(V_n)_{n \in \nset}$ is a
  sequence of random variables in such that for any $n \in \nset$, $V_n$ has
  distribution $\nu_{\bfX_{n \gamma}}$ conditionally to $\bfX_{n
    \gamma}$. We say that
  $(X_n^\gamma)_{n \in \nset} = (\bfX_{n \gamma})_{n \in \nset} \in 
  \M$ is a Geodesic
  Random Walk (GRW).
\end{definition}

Note that for any $n \in \nset$ and $\gamma >0$, $X_n^\gamma \in \M$.  For any
$\gamma>0$, we denote by $(\Qker_n^{\gamma})_{n \in \nset}$ the sequence of
Markov kernels such that for any $n \in \nset$, $x \in \M$ and
$\msa \in \mcb{\M}$ we have that
$\updelta_x \Qker_n^\gamma(\msa) = \Pbb(X_n^\gamma \in \msa)$, with
$X_0^\gamma = x$.  GRWs are appealing because, under mild conditions, when the
stepsize $\gamma \to 0$, they converge towards $(\bfX_t)_{t \geq 0}$ solution of
\eqref{eq:generic} in the following sense:

\begin{theorem}[{{Convergence of geodesic random walk \cite[Theorem 2.1]{jorgensen1975central}}}]
  \label{thm:grw_diffusion}
  Under the conditions of \cref{thm:jorgensen_appendix}, for any $t \geq 0$,
  $f \in \rmc(\M)$  we have that
  $\lim_{\gamma \to 0} \normLigne{ \Qker_{\gamma}^{\ceil{t/\gamma}}[f] -
    \Pker_t[f]}_{\infty} = 0$, where $(\Pker_t)_{t \geq 0}$ is the
  semi-group associated with the infinitesimal generator
  $\generator: \ \rmc^\infty(\M) \to \rmc^\infty(\M)$ given for any
  $f \in \rmc^\infty(\M)$ by
  $\generator(f) = \langle b, \nabla f \rangle_{\M} + (1/2) \langle
  \Sigma, \nabla^2f \rangle_{\M}$.
\end{theorem}   

In particular if $b = 0$ and $\sigma = \Id$, then the random walk
converges towards a Brownian motion on $\M$ in the sense of the convergence
of semi-groups.
% In this case, for any $x \in \M$ in local coordinates we
% have that $\Phi_\# \nu_x$ has zero mean and covariance matrix $G(x)$, where
% $\Phi$ is a local chart around $x$ and
% $G(x) = (g_{i,j}(x))_{1 \leq i,j \leq d}$ the coordinates of the metric in
% that chart.
One advantage of GRW is that they allow to samples from arbitrary diffusions
under mild assumptions. This property will be key to sample from the backward
process. \cref{thm:grw_diffusion} can be extended to approximate time
inhomogeneous diffusions. We leave the proof of this extension for future
work. In \cref{alg:grw}, we remind how to approximately sample from a diffusion
$(\bfX_t)_{t \in \ccint{0,T}}$ using GRWs, where $(\bfX_t)_{t \in \ccint{0,T}}$
associated with the family of infinitesimal generator
$(\generator_t)_{t \in \ccint{0,T}}$ given for any $t \in \ccint{0,T}$ and
$f \in \rmc^2(\M)$ by
$\generator_t(f) = \langle b_t, \nabla f \rangle + \langle \Sigma_t, \nabla^2 f
\rangle$, where $b: \ \ccint{0,T} \to \XM$, $\Sigma_t = \sigma_t \sigma_t^\top$
with $\sigma_t : \ \ccint{0,T} \to \XMdeux$. For simplicity, in \cref{alg:grw},
we assume that $\M$ is embedded in $\rset^p$ and use the projection to define
the noise on the tangent space (such an embedding always exists using the Nash
embedding theorem, see \cite{gunther1991isometric} for example).  In a more
general setting, we compute the noise on the tangent space using local
coordinates.


\begin{algorithm}[!t]
\caption{\small Geodesic Random Walk (GRW)}
\label{alg:grw}
\begin{algorithmic}[1]
 \small
  \Require $T, K, X_0, b, \sigma, P$
  \State $\gamma = T / K$ \Comment Step-size
  \For{$k \in \{0, \dots, K-1\}$}
  \State $\bar{Z}_{k+1} \sim \mathcal{N}(0, I_p)$ \Comment Standard Gaussian in ambient space $\rset^p$
  \State $Z_{k+1} = P(X_k) \bar{Z}_{k+1}$ \Comment Projection in the tangent space $\mathrm{T}_x \M$ 
  \State $V_{k+1} = \gamma b(k \gamma, X_k) + \sqrt{\gamma} \sigma(k \gamma, X_k) Z_{k+1}$ \Comment Euler-Maruyama step on tangent space 
  \State $X_{k+1} = \exp_{X_k}\left(V_{k+1}\right)$ \Comment Geodesic projection onto $\M$
  \EndFor
  \State {\bfseries return} $\{ X_k\}_{k=0}^{K-1}$
\end{algorithmic}
\end{algorithm}

\paragraph{Heat kernel on compact Riemannian manifolds}
The semi-group of the Brownian motion $(\Pker_t)_{t \geq 0}$ (called the heat
kernel) admits a density w.r.t.\ $\piinv$, such that for any $f \in \rmc(\M)$,
$x_0 \in \M$ and $t > 0$ we have
\begin{equation}
  \textstyle{
    \updelta_{x_0} \Pker[f] = \int_{\M} f(x_t)p_{t|0}(x_t|x_0)  \rmd \piinv(x_t).
    }
\end{equation}
In addition, this transition density is positive and
$(t,x,y) \mapsto p_{t|0}(y|x) \in \rmc^\infty(\ooint{0,+\infty} \times \M \times
\M)$ and satisfies the heat equation $\partial_t p_{t|0} = \Delta
p_{t|0}$. However, contrary to the Gaussian transition density of the
Ornstein--Ulhenbeck process, it is typically only available as an infinite
series. In order to circumvent this issue we consider two
techniques: \begin{enumerate*}[label=\roman*)]
\item a truncation approach, 
\item a Taylor expansion around $t=0$, \ie \ a Varadhan asymptotics.
\end{enumerate*}    
%
First, we recall that in the case of compact manifolds we have that for any
$t > 0$ and $x, y \in \M$
\begin{equation}
  \label{eq:infinite_sum}
  \textstyle{p_{t|0}(x,y) = \sum_{j \in \nset} \rme^{-\lambda_j t} \phi_j(x)\phi_j(y),}
\end{equation}
where the convergence occurs in $\mathrm{L}^2(\piinv \otimes \piinv)$,
$(\lambda_j)_{j \in \nset}$ and $(\phi_j)_{j \in \nset}$ are the
eigenvalues, respectively the eigenvectors, of $-\Delta_\M$ in
$\mathrm{L}^2(\piinv)$ \cite[see][Section 2]{saloff1994precise}. When the eigenvalues and eigenvectors are known, we approximate the
logarithmic gradient of $p_{t|0}$ by truncating the sum in
\cref{eq:infinite_sum} with $J \in \nset$ terms to obtain for any
$t > 0$ and $x,y \in \M$
\begin{equation}
  \nabla_x \log p_{t|0}(x,y) \approx \textstyle{S_{J,t}(x,y) = \sum_{j=0}^J \rme^{-\lambda_j t} \nabla \phi_j(x) \phi_j(y) / \sum_{j=0}^J \rme^{-\lambda_j t} \phi_j(x) \phi_j(y). }
\end{equation}    
Note that for any $t \geq 0$, $x, y\in \M$,
$S_{J,t}(x,y) \in \mathrm{T}_x \M$.  Under regularity conditions on $\M$ it can be
shown that for any $x,y \in \M$ and $t \geq 0$,
$\lim_{J \to +\infty} S_{J,t} = \nabla_x \log p_{t|0}(x,y)$ \cite[see][Lemma
1]{jones2008Manifold}. In the case of the $d$-dimensional torus or sphere
the eigenvalues and eigenvectors are known, \cite[see][Section
2]{saloff1994precise} and we can apply this method to approximate $p_{t|0}$
for any $t > 0$. We refer to \cref{sec:eigenf-eigenv-lapl} for more details
about eigenvalues and eigenfunctions of the Laplace-Beltrami operator in the
special case of the $d$-dimensional torus and sphere.  \valentin{experiment:
  quality o the approximation as a function of $J$ and $t$. On the same
  graph put the Varadhan approx}

When the eigenvalues and eigenvectors are not tractable, we
can still derive an approximation of the heat kernel for small times $t$. Using
Varadhan's asymptotics---see \citet[Theorem 3.8]{bismut1984large} or
\citet[Theorem 2.1]{chen2021logarithmic}---for any $x, y \in \M$ with
$y \notin \mathrm{Cut}(x)$ (where $\mathrm{Cut}(x)$ is the cut-locus of $x$
in $\M$) we have that \cite[see][Chapter 10]{lee2018introduction} 
\begin{equation}
  \label{eq:varadhan}
  \textstyle{\lim_{t \to 0} t \nabla_y \log p_{t|0}(x,y) = - \exp^{-1}_x(y) . }
\end{equation}
Note that since the cut-locus has null measure under the uniform distribution
$\piinv$ \citep[Theorem 10.34]{lee2006riemannian}, the previous relation is
valid almost everywhere.  We will see in \cref{sec:riem-score-appr} that an
approximation for any $x, y \in \M$ of $t \nabla_y \log p_t(x,y)$ for small
values of $t \geq 0$ is enough to define a score approximation.


\subsection{A manifold time-reversal formula}
\label{sec:time-revers-form}

After having defined the forward noising process targeting a reference
distribution, a second key ingredient of SGMs is to derive a time-reversal
formula. Namely, if $(\bfX_t)_{t \in \ccint{0,T}}$ is a diffusion process then
$(\bfX_{T-t})_{t \in \ccint{0,T}}$ is also a diffusion process w.r.t.\ the
backward filtration whose coefficients can be computed, see
\cref{sec:time-reversal}.  Our next result is the Riemannian
counterpart to the Euclidean time-reversal formula, see \citet[Theorem
4.9]{cattiaux2021time} and \citet{haussmann1986time} for instance, which states
under mild regularity and integrability conditions if the $\rset^d$-valued
process $(\bfX_t)_{t \in \ccint{0,T}}$ is a (weak) solution to the SDE
\begin{equation}
  \rmd \bfX_t = b(\bfX_t) \rmd t +  \rmd  \bfB_t  , 
\end{equation}
then $(\bfY_t)_{t \in \ccint{0,T}} = (\bfX_{T-t})_{t \in \ccint{0,T}}$ is a
(weak) solution to the SDE
\begin{equation}
  \rmd \bfY_t = \{-b(\bfY_t) + \nabla \log p_{T-t}(\bfY_t) \}\rmd t + \rmd \bfB_t  , 
\end{equation}
In the case where $(\bfX_t)_{t \in \ccint{0,T}}$ is an Ornstein-Ulhenbeck
process, then we recover \cref{eq:backward_SDE}. 

\begin{theorem}[Reverse diffusion]
  \label{thm:time_reversal_manifold}
  Let $T \geq 0$ and $(\bfB_t^\M)_{t \geq 0}$ be a Brownian motion on $\M$ such
  that $\bfB_0^\M$ has distribution $\piinv$.  Let
  $(\bfX_t)_{t \in \ccint{0,T}}$ associated with the SDE
  $\rmd \bfX_t = b(\bfX_t) \rmd t + \rmd \bfB_t^\M$.  Let
  $(\bfY_t)_{t \in \ccint{0,T}} = (\bfX_{T-t})_{t \in \ccint{0,T}}$ and assume
  that $\KLLigne{\Pbb}{\Qbb} < +\infty$, where
  $\Qbb \in \Pens(\rmc(\ccint{0,T}, \M))$ is the distribution of
  $(\bfB_t^\M)_{t \in \ccint{0,T}}$. In addition, assume that for any
  $t \in \ccint{0,T}$, $\Pbb_t$ admits a smooth positive density $p_t$ w.r.t.\
  $\piinv$. Then, we have that $(\bfY_t)_{t \in \ccint{0,T}}$ is associated with
  the SDE
  \begin{equation}
    \label{eq:time_reversal_manifold}
   \rmd \bfY_t = \{-b(\bfY_t) + \nabla \log p_{T-t}(\bfY_t)\} \rmd t + \rmd \bfB_t^\M. 
  \end{equation}
\end{theorem}

\begin{proof}
  The proof is a smooth extension of \citet[Theorem 4.9]{cattiaux2021time} to the
  Riemannian manifold case. We postpone the detailed proof to
  \cref{sec:time-reversal}.
\end{proof}

Note the formula obtained for the drift of the time reversed process are the
same in the Euclidean and the Riemannian settings upon replacing the Euclidean
gradient operator, Laplacian and scalar product by the Riemannian ones.  As a
corollary of \cref{thm:time_reversal_manifold}, we get the following result.

% Recall that in the case where the manifold is isometrically embedded in $\rset^p$
% for some $p \geq d$ then $(\bfB_t^\M)_{t \in \ccint{0,T}}$ satisfies \eqref{eq:brownian_motion_extrinsic}.
% Therefore, if we let $(\bfY_t)_{t \in \ccint{0,T}}= (\bfB_{T-t}^\M)_{t \in \ccint{0,T}} $ we have that for any $j \in \{1, \dots, p\}$
% \begin{equation}
% ....
% \end{equation}
%\valentin{I dont even see now how we can get the extrinsic result!}

\begin{corollary}
  Under the conditions of \cref{thm:time_reversal_manifold}, denote
  $(\Pker_t)_{t \in \ccint{0,T}}$ and
  $(\Qker_{s,t})_{s, t \in \ccint{0,T}, t\geq s}$ the semi-group associated with
  $\Pbb$, respectively $R(\Pbb)$. Then:
  \begin{enumerate}[label= (\alph*),  wide, labelwidth=!, labelindent=0pt]
  \item $(\Pker_t)_{t \in \ccint{0,T}}$ is associated with the generator
          $\generator: \ \rmc^\infty(\M) \to \rmc^\infty(\M)$ given for any
      $f \in \rmc^\infty(\M)$ by
      $\generator(f) = \langle b, \nabla f \rangle_{\M} + (1/2) \Delta_\M f$.
  \item $(\Qker_{t|s})_{s,t \in \ccint{0,T}, t\geq s}$ is associated with the family of  generators $(\generator_u)_{u \in \ccint{0,T}}$ such that for any $u \in \ccint{0,T}$, 
          $\generator_u: \ \rmc^\infty(\M) \to \rmc^\infty(\M)$ is given for any
      $f \in \rmc^\infty(\M)$ by
      $\generator_u(f) = \langle -b + \nabla \log p_{T-u}, \nabla f \rangle_{\M} + (1/2) \Delta_\M f$.
  \end{enumerate}
\end{corollary}

In particular, we can approximate the time-reversed process using a GRW using
\cref{thm:grw_diffusion}.

\subsection{Score approximation on Riemannian manifolds}
\label{sec:riem-score-appr}

The last ingredient in order to define the (compact) Riemannian manifold
extension of SGM is an approximation of the logarithmic gradient appearing in
\cref{eq:time_reversal_manifold}.


\paragraph{Score-matching and loss functions}
We aim to approximate $\nabla \log p_t(x)$ for every $t \in \ocint{0,T}$ and
$x \in \M$. To do so, we first remark that for any $s,t \in \ocint{0,T}$ with $t > s$ and
$x_t \in \M$, $p_t(x_t) = \int_{\M} p_{t|s}(x_t|x_s) \rmd \Pbb_s(x_s)$.  Therefore,
we obtain that for any $s, t \in \ccint{0,T}$ with $t > s$ and $x_t \in \M$
  \begin{equation}
  \textstyle{
    \nabla \log p_t(x_t) = \int_{\M} \nabla_x \log p_{t|s}(x_t|x_s) \Pker_{s|t}(x_t, \rmd x_s)  .
    }    
  \end{equation}
  Hence, for any $s, t \in \ccint{0,T}$ with $t > s$ we have that
  \begin{equation}
    \textstyle{
      \nabla \log p_t = \argmin \ensembleLigne{\ell_{t|s}(s_t)}{s_t \in \rmL^2(\Pbb_t)}  , \quad \ell_{t|s}(s_t) = \int_{\M \times \M} \normLigne{\nabla_x \log p_{t|s}(x_t|x_s) - s_t(x_t)}^2 \rmd \Pbb_{s,t}(x_s,x_t)   .
      }
    \end{equation}
    The loss function $\ell_{t|s}$ is called the Denoising Score Matching (DSM)
    loss. It can also be written in an \emph{implicit} fashion.
    \begin{proposition}
      \label{prop:implicit_der}
      Let $t \in \ocint{0,T}$. If $s_t \in \rmc^\infty(\M)$ then we have that  $\ell_{t|s}(s_t) = 2 \ellim_t(s_t) + \int_{\M \times \M} \normLigne{\nabla \log p_{t|s}(x_t|x_s)}^2 \rmd \Pbb_{s,t}(x_s,x_t)$, where
      \begin{equation}
        \textstyle{
          \ellim_t(s_t) = \int_\M \{ \tfrac{1}{2}\normLigne{s_t(x_t)}^2 + \dive(s_t)(x_t) \}  \rmd \Pbb_t(x_t)  . 
          }
      \end{equation}

    \end{proposition}

    \begin{proof}
      The proof is postponed to \Cref{sec:implicit-losses}.
    \end{proof}

    For any $t \in \ocint{0,T}$ the minimizers of the loss $\ellim_t$ on $\XM$
    are the same as the ones for $\ell_{t|s}$. The loss $\ellim_t$ is called the
    \emph{implicit} score matching (ISM) loss (or sliced score matching (SSM)
    loss if the divergence is approximated using the Hutchinson's trace
    estimator \cite{hutchinson1989stochastic}).  Depending on the assumptions on
    the specific manifold at hand it may be more convenient to use $\ell_{t|s}$
    or $\ellim_t$.  Assume that we have access to
    $\ensembleLigne{\nabla \log p_{t|s}}{s, t \in \ccint{0,T}, \ t > s}$ or an
    approximation of this family, then we can use $\ell_{t|s}$, the
    \emph{explicit} score function to learn
    $\ensembleLigne{s_t}{t \in \ccint{0,t}}$. Using the results of
    \cref{sec:brown-moti-comp}, we highlight to methods to approximate
    $\ell_{t|s}$:
    \begin{enumerate}[label= (\alph*),  wide, labelwidth=!, labelindent=0pt]
    \item If we have access to an approximation of
      $\ensembleLigne{p_{t|0}}{t \in \ocint{0,T}, \ t}$ then $\ell_{t|0}$ can be
      used. Note that this loss is similar to the one used in the Euclidean
      setting, see
      \citep{song2020score,song2020improved,song2020denoising,ho2020denoising}
      for instance. In the case, where the eigenvalues and the eigenfunctions of
      the Laplace-Beltrami operator are known then such an approximation is
      available, see \cref{sec:brown-moti-comp}. However, the quality of the
      approximation deteriorates when $t$ is close to
      $0$. % In particular, in the case of
      % the sphere or the torus, we use this loss function as our baseline, see
      % \cref{sec:experiments}. \valentin{TO MODIFY}
    \item If we do not have access to the eigenvalues and eigenfunctions of the
      Laplace-Beltrami operators then we can still derive an approximation of
      the $\nabla \log p_{t|s}$ for all $s \in \ccint{0,t}$ if $\abs{t-s}$ is
      small enough, using Varadahn type approximations \eqref{eq:varadhan} and
      the inverse of $\exp$\footnote{If $\exp^{-1}$ is not available then it can
        be estimated using approximated logarithmic mappings
        \citep{goto2021approximated,schiela2020sqp} or inverse retractions
        \citep{zhu2020riemannian,sato2019riemannian}.}. In this case we use the
      loss functions $\ell_{t|s}$ for $\absLigne{t-s}$ small enough.
    \end{enumerate}
    We highlight that these two methods can be used in conjunction. For
    instance, one can rely on the truncation techniques to estimate $\ell_{t|0}$
    for large $t$ and the Varadhan asymptotics for small $t$.
    
    Last but not least, the \emph{implicit} score loss $\{\ellim_t\}_{t=0}^T$ is
    used in cases where we do not have access to the approximations of $p_{t|s}$
    for $s,t \in \ccint{0,T}$ with $t > s$. The only requirement to learn the
    implicit score is to be able to (approximately) sample from the forward
    dynamics, i.e. the Brownian motion on the Riemannian manifold. In
    particular, no approximation of the logarithmic derivative of the heat
    kernel is needed. One downside of using such an approach is that it relies
    on the computation of the divergence of the score $s_t$. The exact
    computation of the divergence is too costly in high dimension as it requires
    $d$ Jacobian-vector calls and estimators need to be used
    \cite{hutchinson1989stochastic}. Note that the loss function used in
    \citep{rozen2021moser} also involves computing a divergence. We summarize our
    different loss functions in \cref{tab:sm_losses}.

\begin{table}[h]
\centering
\small
\renewcommand*{\arraystretch}{1.4}
\begin{tabular}{ccl}
Method & Loss function  & Requirements \\
\midrule
$\ell_{t|0}$ (DSM)   &  $\frac{1}{2} \E \left[ \| s(\bfX_t) - \nabla \log p_{t|0}(\bfX_t|\bfX_0) \|^2 \right]$ &  \vtop{\hbox{\strut $\triangleright$ Sampling of $(\bfX_t, \bfX_0)$}\hbox{$\triangleright$ Approximation of $\nabla \log p_{t|0}$}}  \\ %\hline
$\ell_{t|s}$ (DSM)   &  $\frac{1}{2} \E \left[ \| s(\bfX_t) - \nabla \log p_{t|0}(\bfX_t|\bfX_s) \|^2 \right]$ & \vtop{\hbox{\strut $\triangleright$ Sampling of $(\bfX_t, \bfX_s)$ for $\abs{t-s}$ small}\hbox{$\triangleright$ Approximation of $\nabla \log p_{t|s}$ for $\abs{t-s}$ small}} \\ %\hline
$\ellim_t$ (ISM)  &  $\E \left[\frac{1}{2} \| s(\bfX_t) \|^2 + \dive( s)(\bfX_t)  \right]$  & \vtop{\hbox{\strut $\triangleright$ Sampling of $\bfX_t$}\hbox{$\triangleright$ Approximation of $\dive(\bm{s}_\theta)$}}
\end{tabular}
\caption{\small Riemannian score matching losses.}
\label{tab:sm_losses}
\end{table}

\paragraph{Parametric family of vector fields}
% We need to define a parametric family of functions in order to
We approximate $\{\nabla \log p_t\}_{t=0}^T$ by a
family of function $\{\bm{s}_\theta\}_{\theta \in \Theta}$ where $\Theta$ is a
set of parameters and for any $\theta \in \Theta$,
$\bm{s}_\theta: \ \ccint{0,T} \to \XM$. In this work, we consider several
parameterisations of vector fields:
%
\begin{itemize}
\item \textbf{Projected vector field}. We define
  $\bm{s}_\theta(t, x) = \text{proj}_{T_{x}\M}(\tilde{\bm{s}}_\theta(t, x)) = P(x)
  \tilde{\bm{s}}_\theta(t, x) $ for any $t \in \ccint{0,T}$ and $x \in \M$, with
  $\tilde{\bm{s}}_\theta: \ \rset^p \times \ccint{0,T} \to \rset^p$ an ambient vector
  field and $P(x)$ the orthogonal projection over $\mathrm{T}_x\M$ at $x \in M$.
  According to \citet[Lemma 2]{rozen2021moser}, then
  $\dive(s_\theta)(x,t) = \dive_E(s_\theta)(x,t)$ for any $x \in \M$, where
  $\dive_E$ denotes the standard Euclidean divergence.
    % \mjh{We use the same trick in the vec field GP paper} \emile{We also do that in practice in \citep{mathieu2020riemannian}}
    

\item \textbf{Divergence-free vector fields}: For any compact \valentin{I think
    this is needed? Maybe not} Lie group, any basis of the Lie algebra $\mathfrak{g}$
  yields a global frame. Indeed, let $v \in \mathfrak{g}$ and define the flow
  $\Phi: \ \rset \times \M \to \M$ given for any $t \in \rset$ and $x \in M$ by
  $\Phi_t^v(x) = x \exp(t v)$. Then defining
  $\{E_i\}_{i=1}^d = \{\partial_t \Phi_0^{v_i}\}_{i=1}^d$, where
  $\{v_i\}_{i=1}^d$ is a basis of $\mathfrak{g}$, we get that $\{E_i\}_{i=1}^d$
  is a left-invariant global frame. As a result, we have that for any
  $i \in \{1, \dots, d\}$, $\dive(E_i)=0$ (for the classical left invariant
  metric). This result simplifies the computation of $\dive(\bm{s}_\theta)$ where
  $\bm{s}_\theta(t,s) = \sum_{i=1}^d \bm{s}^i_\theta(t,x) E_i(x)$ for any
  $t \in \ccint{0,T}$ and $x \in \M$ \cite[see][]{falorsi2020neural}.
%   We define
%   $s_\theta(x, t) = \sum_{i=1}^d s^i_\theta(x, t) f^i(x)$ for any
%   $t \in \ccint{0,T}$ and $x \in \M$, where for any $i \in \{1, \dots, d\}$,
%   $f_i(x) = \partial_t \exp( \cdot \xi_i)(0) \cdot x$ (with some frame
%   $\{\xi_i\}_{i=1}^d$). For any homogeneous space, this family of vector fields
%   generates the tangent bundle and is divergence-free \valentin{ref}.
%     % \mjh{Is the divergence free nature of the field going to impose some extra constraint on the score function, and the resulting process?}
%     % The $f^i$ are divergence free but the linear combination which yields $s_\theta$ has no constraint.
%   Then for any $t \in \ccint{0,T}$ and $x \in \M$, we have
%   $\dive(s_\theta)(x, t) = \sum_{i=1}^d \dive(s^i_\theta(\cdot, t) f^i)(x)  = \sum_{i=1}^d \langle \nabla
%     s^i_\theta(x, t), f^i(x)\rangle$. \valentin{this can be estimated stochastically. How?}
% %    For more information see notes at \url{https://www.overleaf.com/read/thvfprqwmkjq}.
%     % Not sure whether this can be useful here but this term does appear in standard (non-denoising) score matching \citep{hyvarinenEstimation,song2019Sliced}. }
\item \textbf{Coordinates vector fields}. We define
  $\bm{s}_\theta(t, x) = \sum_{i=1}^d \bm{s}^i_\theta(t,x) E_i(x)$ for any
  $t \in \ccint{0,T}$ and $x \in \M$, with
  $\{E_i\}_{i=1}^d = \{\partial_i \varphi(x)\}_{i=1}^d$ the vector fields
  induced by a choice of local coordinates, where $\varphi$ is a local
  parameterization $\varphi: \ \msu \to \M$ and $z \in \msu \subset
  \rset^d$. Then the divergence can be computed in these local coordinates
  $\dive(\bm{s}_\theta)(t, \varphi(z)) =\absLigne{\det G}^{-1/2} \sum_{i=1}^d
  \partial_i \{ \absLigne{\det G}^{1/2} \bm{s}^i_\theta(t,
  \varphi(\cdot))\}(z)$. In the case of the sphere, one recovers the standard
  divergence in spherical coordinates using this formula.
  % If the manifold is not
  % parallelisable, there does not exist a global frame, which implies that
  % $\{f^i(x)\}$ is not a basis for any $x \in \M$. Think of the Hairy-ball
  % theorem for the (n-)sphere.
  % \mjh{but we don't actually need a basis. We just need a smooth set of basis
  % vector fields that span the tangent space. The fields can be redundant
  % e.g. 3 axis fields on the sphere. We then just mix these smooth fields with
  % smooth scalar coeffs from an nn and mix them to get a smooth field}
  % \emile{Agree that we don't need a basis, we only need a generator of the
  % tangent bundle.}
\end{itemize}
\emile{We do not discuss NN architectural choices for $\{s_\theta^i\}_i$ but can do for the next iteration.}
%
Combining this parameterization with the score-matching losses, the
time-reversal formula \cref{sec:time-revers-form} and the sampling of forward
and backward processes \cref{sec:brown-moti-comp}, we now define our Riemannian
Score-based Generative Modeling algorithm, in \cref{alg:rsgm}.


  \begin{algorithm}[!t]
   \caption{\small Computation of the loss}
   \label{alg:rsgm}
   \begin{algorithmic}[1]
     \small
     \Require $\vareps, T, K, \pizero, \mathrm{loss}, \mathrm{thres}, s$
     \State $\bfX_0 \sim \pizero$
     \State $t \sim U(\ccint{\vareps, T})$ \Comment Uniform sampling between $\vareps$ and $T$
     \State $\bfX_t \sim \Pker_{t|0}(\bfX_0, \cdot)$ \Comment Approximate sampling using \cref{alg:grw} 
     \If{$\mathrm{loss = denoising}$} \Comment Denoising loss function
     \If{$t < \mathrm{thres}$}
     \State $\mathrm{score} = -(1/t) \exp^{-1}_{\bfX_t}(\bfX_0)$ \Comment Varadhan asymptotics
     \Else
     \State $\mathrm{score} = \sum_{j=0}^J \rme^{-\lambda_j t} \nabla \phi_j(x) \phi_j(y) / \sum_{j=0}^J \rme^{-\lambda_j t} \phi_j(x) \phi_j(y).$ \Comment Series truncation
     \EndIf
     \State $\ell(s) = \norm{s(\bfX_t) - \mathrm{score}}^2$
     \Else \Comment Implicit loss function
     \State $\ell(s) = (1/2) \norm{s(\bfX_t)}^2 + \dive(s)(\bfX_t)$
     \EndIf
      \State {\bfseries return} $\ell(s)$
    \end{algorithmic}
    \end{algorithm}


\subsection{Likelihood computation}
\label{sec:likel-comp}

Similarly to \cite{song2020score}, once the score is learned we can use it
in conjunction with an Ordinary Differential Equation (ODE) solver to compute
the likelihood of the model. Let $\{\Phi_t\}_{t=0}^T$ be a family of vector
fields. We define $(\bfX_t)_{t \in \ccint{0,T}}$ such that $\bfX_0$ has
distribution $p_0$ (the data distribution) and satisfying
$\rmd \bfX_t = \Phi_t(\bfX_t) \rmd t$. Assuming that  $p_0$ admits a density
w.r.t.\ $\piinv$ then for any $t \in \ccint{0,T}$, the distribution of $\bfX_t$
admits a density w.r.t.\ $\piinv$ and we denote $p_t$ this density.  We recall that
$\partial_t \log p_t(\bfX_t) = \dive(\Phi_t)(\bfX_t)$, see \citet[Proposition 
2]{mathieu2020riemannian} for instance.

Recall that we consider a Brownian motion on the manifold as a forward process
$(\bfB_t^\M)_{t \in \ccint{0,T}}$ with $\{p_t\}_{t=0}^T$ the associated family
of densities. Thus we have that for any $t \in \ccint{0,T}$ and $x \in \M$
\begin{equation}
  \partial_t p_t(x) = \tfrac{1}{2} \Delta p_t(x) = \dive\left(\tfrac{1}{2} p_t \nabla \log p_t \right)(x)  . 
\end{equation}
Hence, we can define $(\bfX_t)_{t \in \ccint{0,T}}$ satisfying
$\rmd \bfX_t = \tfrac{1}{2} \nabla \log p_t(\bfX_t) \rmd t$ such that $\bfX_0$ has
distribution $p_0$.
Defining
$(\bfhX_t)_{t \in \ccint{0,T}} = (\bfX_{T-t})_{t \in \ccint{0,T}}$, it follows
that $\bfhX_0$ has distribution $\mathcal{L}(\bfX_T)$ and satisfies
\begin{equation}
  \label{eq:backward_flow}
 \rmd \bfhX_t =-\tfrac{1}{2} \nabla \log p_{T-t}(\bfhX_t) \rmd t  . 
\end{equation}
Finally, we introduce $(\bfY_t)_{t \in \ccint{0,T}}$ satisfying
\eqref{eq:backward_flow} but such that $\bfY_0 \sim \piinv$.  Note
that if $T \geq 0$ is large then the two processes
$(\bfY_t)_{t \in \ccint{0,T}}$ and $(\bfhX_t)_{t \in \ccint{0,T}}$ are close
since $\mathcal{L}(\bfX_T)$ is close to $\piinv$.  Therefore, using the score
network and a manifold ODE solver \citep[as in][]{mathieu2020riemannian}, we
are able to approximately solve the following ODE
\begin{equation}
  \partial_t \log q_t(\bfY_t) = -\tfrac{1}{2}\dive(\bm{s}_\theta(t,\cdot))(\bfY_t)  ,
\end{equation}
with $q_t$ the density of $\bfY_t$ w.r.t.\ $\piinv$ and $\log q_0(\bfY_0) =
0$. The likelihood approximation of the model is then given by $\log q_T(\bfY_T)$. In
\cref{sec:diff-betw-ode}, we highlight that this likelihood computation is
slightly different from the one obtained using the SDE.


%%% Local Variables:
%%% mode: latex
%%% TeX-master: "main"
%%% End:


\section{Related work}
\label{sec:related-works}

% The study of distribution on manifolds and their approximation is a central part
% of \emph{directional statistics}, see \citep{mardia2009directional}.
% Traditional methods rely on fitting mixtures of distributions to the target. However, in
% recent years several other methods have been introduced in the context of generative modeling.
In what follows, we discuss previous work on parametrizing family of distributions for manifold-valued data. %supported on manifolds.
% We first briefly discuss
Note that in this work, the manifold structure is considered to be prescribed.
In contrast, another line of work has been focusing on jointly learning the manifold structure and a generative model
% combining manifold learning and generative modeling have also been proposed 
\citep{brehmer2020flows,kalatzis2021multi,caterini2021Rectangular}.
% At a high level, the related work can be divided 


\paragraph{Parametric family of distributions.} Defining flexible easy-to-sample
distributions on manifolds is not a trivial task.
The various parametric families of distributions that have been proposed can broadly be categorised into three main approaches \citep{navarro2017multivariate}: wrapping, projecting and conditioning.
Wrapped distributions consider a parametric distribution on $\mathbb{R}^n$ that is pushed-forward along an invertible map $\psi: \mathbb{R}^n \rightarrow \M$.
% Parametric wrapped distributions are usually defined on $\mathbb{S}^1$, 
A canonical example is the wrapped normal distribution on $\mathbb{S}^1$
\citep{collett1981Discriminating}.  Another example has been proposed by
\cite{mathieu2019continuous,nagano2019wrapped} on the hyperbolic space with the
exponential map \valentin{this is the same thing as push forward of Euclidean
  NF?}.  Given a Euclidean submanifold $\M \subset \mathbb{R}^n$ and a
distribution $p_{\text{amb}} \in \Pens(\mathbb{R}^n)$,
% a distribution can be defined on $\M$ by marginalizing out a 
marginalizing out $p_{\text{amb}}$ along the normal bundle induces a distribution on $\M$.
Samples are obtained by first sampling $p_{\text{amb}}$ and then applying an orthogonal projection on these samples.
Finally, the conditioning method consists into considering the unormalized density defined by the restriction of an ambient density $p_{\text{amb}}$ with $\M$.
% According to \citep{navarro2017multivariate}, there exists three main methods to sample from
% these distributions on manifolds such as spheres and tori: wrapping, projecting and conditioning.
% Wrapped distributions are usually defined on $\mathbb{S}^1$,
% by considering a distribution on $\rset$. Then using that
% $\phi: \mathbb{S}^1 \times \zset \to \rset$ with
% $\phi(\theta, k) = \theta + 2k\uppi$ is a bijection, we define a probability
% distribution on $\mathbb{S}^1$ by marginalizing along the second component.
% Let $\mathcal{M}$ be a submanifold of $\rset^d$ such that
% $\mathcal{M} = \ensembleLigne{x \in \rset^d}{\phi^{-1}(x)_2 = a}$ with
% $\phi: \ \mse_1 \times \mse_2 \to \rset^d$ a diffeomorphism and $a \in
% \mse_2$
% Then, the projecting method consists into considering a probability
% distribution in $\mse_1 \times \mse_2$ and marginalizing w.r.t. $\mse_2$.
% On the
% $(d-1)$-dimensional sphere this amounts to sampling from the probability
% distribution and normalizing the samples.
% Finally, the conditioning method
% consists into considering the disintegration of a probability distribution
% w.r.t. $\phi^{-1}_2$.
Such distributions encompass the von Mises-Fisher
distribution \citep{fisher1953dispersion} and the Kent distribution
\citep{kent1982fisher}.
These distributions are usually unimodal and 
% in order to fit more complex distributions it is necessary to consider mixtures
considering mixtures of thereof is key to increase flexibility
\citep{peel2001fitting,mardia2008multivariate}.
% quid Riemannian normal distributions? (max entropy generalisation)
% also quid power spherical distribution?

% \paragraph{Normalizing flows in latent spaces.}
\paragraph{Push-forward of Euclidean normalizing flows.}
More recently, approaches leveraging the flexibility of normalizing flows
\citep{papamakarios2019normalizing} have been proposed.
Following the wrapping method described above, these methods 
% The simplest approach is to 
parametrize a normalizing flow in the Euclidean space $\mathbb{R}^n$ that is pushed-forward along an invertible map $\psi: \mathbb{R}^n \rightarrow \M$.
However, to globally represent the manifold, the map $\psi$ needs to be a homeomorphism, which can only happen if $\M$ is topologically equivalent to $\mathbb{R}^n$, hence limiting the scope of that approach.
One natural choice for this map if the exponential map $\exp_x: \mathrm{T}_x \M \cong \mathbb{R}^d$. %, leading so called wrapped distributions.
This approach has been taken, for instance, by \cite{falorsi2019reparameterizing} and \cite{bose2020latent}, respectively parametrizing distributions on Lie groups and hyperbolic space.
% \cite{gemici2016normalizing} introduced normalizing flows on the sphere using the stereographic projection.
% One limitation of this approach is
% that the probability distribution is hard to model near the pole which is sent
% to $\infty$ using the stereographic mapping. Indeed, if one tries to model a
% probability distribution with one mode near the pole then most of the mass of
% the distribution pushed by the stereographic mapping is concentrated away from
% the origin. As a result, it is hard to approximate this distribution by learning
% a deformation of an easy-to-sample distribution, like a Gaussian
% distribution.
% For Lie groups, \cite{falorsi2019reparameterizing} proposed to
% perform the inference in the Lie algebra and then push the distribution to the
% whole manifold using that for compact Lie groups the exponential mapping is
% surjective.
% Similarly, \cite{bose2020latent} proposed in hyperbolic spaces two
% approaches to push a normalizing flow defined in tangent spaces of the manifold
% using different wrappings.
%
% a normalizing flow approach based on a recursive construction which is more numerically stable than the one proposed in \citep{gemici2016normalizing}.

\paragraph{Neural ODE on manifolds.}
To avoid artifacts or numerical instabilities due to the manifold embedding, another line
of work uses tools from Riemannian geometry to define flows directly on the
manifold of interest
\citep{falorsi2020neural,mathieu2020riemannian,falorsi2021Continuous}.
Since these methods do not require a specific embedding mapping, they 
% can be considered as \emph{intrinsic}.
are referred as \emph{Riemannian}.
% These methods leverage tools from continuous normalizing flows (CNFs) \citep{grathwohl2019Scalable},
They extend continuous normalizing flows (CNFs) \citep{grathwohl2019Scalable} to the manifold setting, by implicity parametrizing flows as solutions of Ordinary Differential Equations (ODEs).
As such, the parametric flow is a \emph{continuous} function of time.
% extending the evolution equation of CNFs to Riemannian manifolds.
This approach has recently been extended by \cite{rozen2021moser}
introducing Moser flows, whose main appeal being that it circumvents the need to solve an ODE in the training process. % by reparametrizing the vector field with an interpolant function.
% is that they do not require backpropagating
% through and Ordinary Differential Equation (ODE).
% Similarly to Moser flow, RGSM learns an interpolation between the target distribution and an easy-to-sample distribution. 


% \begin{itemize}
% \item limitation 1: divergence to compute high dimensional
% \item limitation 2: the importance sampling (high dimensional as well)
% \item comparison on high dimensional sphere
% \item comparison with bunny as well
% \end{itemize}

\paragraph{Optimal transport on manifolds.}
Another line of work has focused on developing flows on manifolds 
% Finally, another recent method introduces flows on manifolds 
using tools from optimal transport. % \citep{ambrosio2003Optimal}.
\cite{sei2013jacobian} introduced a flow that is given by $f_\theta: x \mapsto \exp_x(\nabla \psi^c_\theta)$ 
% the exponential map applied to the gradient of a $c$-convex function, where $c$ is the squared distance on the Riemannian manifold.
with $\psi^c_\theta$ a $c$-convex function and $c=d^2_\M$, where $d_\M$ is the
geodesic distance.  This approach is motivated by the fact that the
optimal transport map takes such an expression
\citep{ambrosio2003Optimal}.  These methods operate directly on the manifold,
similarly to CNFs, yet in contrast they are \emph{discrete} in time.  The
benefits of this approach depend on the specific choice of parametric family of
$c$-convex functions \citep{rezende2021Implicit,cohen2021riemannian},
trading-off expressively with scalability.
% The optimization of these flows is then
% conducted on the parameters of the chosen family of $c$-convex functions
In the case of tori and spheres, \cite{rezende2020Normalizing} introduced \emph{discrete} Riemannian flows based on Möbius transformations and spherical splines.


% Methods to check:
% \begin{itemize}
% \item \cite{mathieu2019continuous} -- pas vraiment relie. C'est le latent space d'un VAE qui est un espace hyperbolique.
% \item \cite{nagano2019wrapped} -- pareil ?
% \item \cite{rey2019diffusion} -- vae aussi
% \item \cite{falorsi2018explorations} -- aussi
% \item \cite{davidson2018hyperspherical} -- aussi
% \end{itemize}


%%% Local Variables:
%%% mode: latex
%%% TeX-master: "main"
%%% End:


\section{Experiments}
\label{sec:experiments}

We evaluate the model on a collection of datasets, each containing an empirical distribution of occurrences of earth and climate science events on the surface of the earth. These events are: volcanic eruptions \cite{volcanoe_dataset}, earthquakes \cite{earthquake_dataset}, floods \citep{flood_dataset} and wild fires \citep{fire_dataset}. In each case the earth is approximated as a perfect sphere. We compare to previous baseline methods: Riemannian Continuous Normalizing Flows \citep{mathieu2020riemannian}, Moser Flows \citep{rozen2021moser} and a mixture of Kent distributions \citep{peel2001fitting}. The mixture of Kent distributions is optimised using an EM algorithm and the optimal number of components is selected on a validation set.
Additionally, we consider another score-based generative model: a standard SBGM on the 2D place followed by the inverse stereographic projection which induces a density on the sphere \citep{gemici2016normalizing}.
More experimental details can be found in \cref{sec:exp_detail}.
We observe from \cref{tab:geoscience}, that the RSBGM model outperforms all other methods in density estimation, in particular by a large margin on the volcanic eruptions dataset.
% Qualitatively, we see on \cref{fig:geoscience} that 

\begin{table}[h]
    \centering
    \begin{tabular}{lrrrrr}
    % \toprule
     & \textbf{Volcano} & \textbf{Earthquake} & \textbf{Flood} & \textbf{Fire} \\
    \midrule
    Mixture of Kent & $-0.95_{\pm 0.14}$ & $0.14_{\pm 0.13}$ & $0.73_{\pm 0.07}$ & $-1.18_{\pm 0.06}$ \\
    Riemannian CNF & $-0.97_{\pm 0.15}$ & $0.19_{\pm0.04}$ & $0.90_{\pm0.03}$ & $-0.66_{\pm0.05}$ \\
    Moser Flow & $-2.02_{\pm 0.42}$ & $-0.09_{\pm0.02}$ & $0.62_{\pm 0.04}$ & $-1.03_{\pm 0.03}$ \\
    Stereographic Score-Based & ${-4.37}_{\pm ???}$ & ${-0.05}_{\pm ???}$ & ${1.32}_{\pm ???}$ & $0.11_{\pm ???}$ \\
    Riemannian Score-Based & $\bm{-5.56}_{\pm0.26}$ & $\bm{-0.21}_{\pm0.03}$ & $\bm{0.52}_{\pm0.02}$ & $\bm{-1.24}_{\pm 0.07}$\\
    \midrule 
    Dataset size & 827 & 6120 & 4875 & 12809 \\
    \bottomrule
    \end{tabular}
    \caption{
    Negative log-likelihood scores for each method on the earth and climate science datasets.
    Bold indicates statistically significant best method.
    Means and standard deviations are computed over 5 different runs.
    }
    \label{tab:geoscience}
\end{table}

\begin{figure}[t]
% \vspace{-0.8em}
  \centering
\begin{subfigure}{.33\textwidth}
  \includegraphics[width=\linewidth]{{pdf_earthquake_rsbgm}.png}
  \put(-150,40){\rotatebox{90}{Stereographic}}
\end{subfigure}\hfil
\begin{subfigure}{.33\textwidth}
  \includegraphics[width=\linewidth]{{pdf_earthquake_rsbgm}.png}
\end{subfigure}\hfil
\begin{subfigure}{.33\textwidth}
  \includegraphics[width=\linewidth]{{pdf_earthquake_rsbgm}.png}
\end{subfigure}\hfil
\begin{subfigure}{.33\textwidth}
  \includegraphics[width=\linewidth]{{pdf_earthquake_rsbgm}.png}
  \put(-150,40){\rotatebox{90}{Riemannian}}
  \put(-90,-10){Earthquake}
\end{subfigure}\hfil
\begin{subfigure}{.33\textwidth}
  \includegraphics[width=\linewidth]{{pdf_earthquake_rsbgm}.png}
  \put(-80,-10){Flood}
\end{subfigure}\hfil
\begin{subfigure}{.33\textwidth}
  \includegraphics[width=\linewidth]{{pdf_earthquake_rsbgm}.png}
  \put(-70,-10){Fire}
\end{subfigure}
\caption{
    Trained score-based generative models on earth sciences data.
    The learned density is colored green-blue.
    Blue and red dots represent training and testing datapoints, respectively.
  }
  \label{fig:geoscience}
% \vspace{-1.0em}
\end{figure}

\section{Discussion and limitations}
\label{sec:conclusion}

In this paper we introduced Riemannian Score-Based Generative Models (RSGMs), a class of deep generative models that represent target densities supported on compact manifolds, as the reverse diffusion process of a Brownian motion.
The main benefits of our method stems from its scalability to high dimensions, its applicability to a broad class of manifolds due to the diversity of available loss functions and its capacity to model complex datasets.
We proved that RSGMs are universal generative models for densities supported on compact manifolds. \valentin{universal ?}
Empirically, we demonstrated that our method outperforms previous work on density estimation tasks with spherical geoscience datasets.


% \paragraph{Limitation}
% contrary to NF we cannot estimate using the KL
% ie we need to have access to data and not just the unnormalized density
One current limitation---similarly to other score-based generative models---is the requirement of samples from the targeted distribution, as such models cannot directly fit an unnormalized density.
% Limitation to compact manifold, extension to hyperbolic
% Proof of convergence similar to Theorem 1 in our paper
An important future work direction, and a current limitation, is the manifold compactness assumption.
Several important manifolds do not fit into this category, such as the special linear group, symmetric positive definite matrices or the hyperbolic space which underlies special relativity \citep{ungar2005Einstein}.
\emile{Should write a paragraph in the app on the conditional extension}
Another extension of interest is conditional sampling. By amortizing SGMs with respect to an observation it is possible to approximately sample from a given posterior distribution \citep[see for instance][]{kawar2021snips,kawar2021stochastic,lee2021priorgrad,sinha2021d2c,batzolis2021conditional,chung2021come}.
% Conditional Riemannian SGM (CRSGM).
% Indeed, by amortizing SGM with respect to an observation it is possible to approximately sample from a given posterior distribution, and therefore solve inverse problems, provided that we know how to sample from the corruption model \citep[see for instance][]{kawar2021snips,kawar2021stochastic,lee2021priorgrad,sinha2021d2c,batzolis2021conditional,chung2021come}.


%%% Local Variables:
%%% mode: latex
%%% TeX-master: "main"
%%% End:


\bibliographystyle{apalike}
\bibliography{bibliography}

\newpage
\appendixhead
\appendix

\theoremstyle{plain}
\newtheorem{unlemma}{Lemma S}
\newtheorem{unproposition}{Proposition S}
\newtheorem{uncorollary}{Corollary S}
\newtheorem{untheorem}{Theorem S}

\setcounter{equation}{0}
\setcounter{figure}{0}
\setcounter{table}{0}
\setcounter{page}{1}
\makeatletter
\renewcommand{\theequation}{S\arabic{equation}}
\renewcommand{\thefigure}{S\arabic{figure}}
\renewcommand{\thetheorem}{S\arabic{theorem}}
\renewcommand{\thedefinition}{S\arabic{definition}}
\renewcommand{\thelemma}{S\arabic{lemma}}
\renewcommand{\thesection}{S\arabic{section}}
\renewcommand{\theremark}{S\arabic{remark}}
\renewcommand{\theproposition}{S\arabic{proposition}}
\renewcommand{\thecorollary}{S\arabic{corollary}}
\setcounter{tocdepth}{1}


\section{Organization of the supplementary}
\label{sec:organ-suppl}

In this supplementary we gather the proof of \cref{thm:time_reversal_manifold}
as well as additional derivations on score-based generative models and
Riemannian manifolds. In \cref{sec:prel-stoch-riem}, we recall basics on
stochastic Riemannian geometry following \cite{hsu2002stochastic}. In
\cref{sec:diff-betw-ode}, we highlight differences between ODE and SDE 
models for likelihood computation. In \cref{sec:eigenf-eigenv-lapl}, we recall
some basic facts about eigenvalues and eigenfunctions of the Laplace-Beltrami
operator on the $d$-dimensional sphere and torus. In \cref{sec:time-reversal},
we present the extension of the time-reversal formula to manifold and prove
\cref{thm:time_reversal_manifold}.

%%% Local Variables:
%%% mode: latex
%%% TeX-master: "main"
%%% End:


\section{Preliminaries on stochastic Riemannian geometry}
\label{sec:prel-stoch-riem}

In this section, we recall some basic facts on Riemannian geometry and
stochastic Riemannian geometry.  We follow
\cite{hsu2002stochastic,lee2018introduction,lee2006riemannian} and refer to
\cite{lee2010introduction,lee2013smooth} for a general introduction to
topological and smooth manifolds. Throughout this section $\M$ is a
$d$-dimensional smooth manifold, $\TM$ its tangent bundle and $\TMstar$ it
cotangent bundle. We denote $\rmc^\infty(\M)$ the set of real-valued smooth
functions on $\M$ and $\XM$ the set of vector fields on $\M$.

\subsection{Tensor field, metric, connection and transport}
\label{sec:metr-conn-tens}

\paragraph{Tensor field and Riemannian metric}

For a vector space $V$ let
$\mathrm{T}^{k, \ell}(V) = V^{\otimes k} \otimes (V^\star)^{\otimes \ell}$ with
$k, \ell \in \nset$. For any $k, \ell \in \nset$ we define the space of
$(k,\ell)$-tensors as
$\mathrm{T}^{k,\ell} \M = \sqcup_{p \in \M}
\mathrm{T}^{k,\ell}(\mathrm{T}_p\M)$. Note that
$\Gamma(\M, \mathrm{T}^{0,0}\M) = \mathrm{C}^\infty(\M)$,
$\XM = \Gamma(\M, \mathrm{T}^{1,0} \M)$ and that the space of $1$-form on $\M$
is given by $\Gamma(\M, \mathrm{T}^{0,1} \M)$, where $\Gamma(\M, V(\M))$ is a
section of a vector bundle $V(\M)$ \citep[see][Chapter 10]{lee2013smooth}.  For
any $k \in \nset$, we denote
$\mathrm{T}^{\abs{k}} \M = \sqcup_{j=0}^k \mathrm{T}^{j,k-j} \M$.
% \valentin{maybe talk about pushforward and pullback here ?}
$\M$ is said to be
a Riemannian manifold if there exists $g \in \Gamma(\M, \mathrm{T}^{0,2} \M)$ such that for
any $x \in \M$, $g(x)$ is positive definite. $g$ is called the Riemannian metric
of $\M$. Every smooth manifold can be equipped with a Riemannian metric
\cite[see][Proposition 2.4]{lee2018introduction}. In local coordinates we define
$G = \{g_{i,j}\}_{1 \leq i,j \leq d} = \{g(X_i, X_j)\}_{1 \leq i,j \leq d}$,
where $\{X_i\}_{i=1}^d$ is a basis of the tangent space. In what follows we
consider that $\M$ is equipped with a metric $g$ and for any $X, Y \in \XM$ we
denote $\langle X,Y \rangle_{\M} = g(X,Y)$.

\paragraph{Connection}
A connection $\nabla$ is a mapping which allows one to differentiate vector
fields w.r.t other vector fields. $\nabla$ is a linear map
$\nabla: \ \XM \times \XM \to \XM$. In addition, we assume that
\begin{enumerate*}[label=\roman*)]
\item for any $f \in \rmc^\infty(\M)$, $X, Y \in \XM$, $\nabla_{f X}(Y) = f \nabla_X Y$, 
\item for any $f \in \rmc^\infty(\M)$, $X, Y \in \XM$, $\nabla_{X}(fY) = f \nabla_X Y + X(f) Y$.
\end{enumerate*}
Given a system of local coordinates, the Christoffel symbols
$\{\Gamma_{i,j}^k\}_{1 \leq i,j,k\leq d}$ are given for any
$i,j \in \{1, \dots, d\}$ by
$\nabla_{X_i}X_j = \sum_{k=1}^d \Gamma_{i,j}^k X_k$. We
also define the Levi-Civita connection $\nabla$ by considering the additional
two conditions: 
\begin{enumerate*}[label=\roman*)]
\item $\nabla$ is torsion-free, \ie \ for any $X, Y \in \XM$ we have
  $\nabla_X Y - \nabla_Y X = [X,Y]$, where $[X,Y]$ is the Lie bracket between
  $X$ and $Y$,
\item $\nabla$ is compatible with the metric $g$, \ie \ for any $X,Y,Z \in \XM$,
  $X (\langle Y,Z \rangle_\M) = \langle\nabla_X Y, Z\rangle_\M + \langle Y, \nabla_X Z \rangle_\M$.
\end{enumerate*}
We recall that the Levi-Civita connection is uniquely defined since for any
$X,Y,Z \in \XM$ we have
\begin{align}
  2 \prodM{\nabla_X Y}{Z} &= X(\prodM{Y}{Z}) + Y(\prodM{Z}{X}) - Z(\prodM{X}{Y}) + \prodM{[X,Y]}{Z} - \prodM{[Z,X]}{Y} - \prodM{[Y,Z]}{X}  . 
\end{align}
In this case, we have that the Christoffel symbols are given for any
$i,j,k \in \{1, \dots, d\}$ by
\begin{equation}
  \textstyle{\Gamma_{i,j}^k = (1/2) \sum_{m=1}^d g^{km} (\partial_j g_{m,i} + \partial_i g_{m,j} - \partial_m g_{i,j}) ,}
\end{equation}
where $\{g^{i,j}\}_{1 \leq i,j \leq d} = G^{-1}$. Note that if $\M$ is Euclidean
then for any $i,j,k \in \{1, \dots, d\}$, $\Gamma_{i,j}^k = 0$. We also extend
the connection so that for any $X \in \XM$ and $f \in \rmc^\infty(M)$ we have
$\nabla_X f = X(f)$. In particular, we have that
$\nabla_X f \in \rmc^\infty(\M)$. In addition, we extend the connection such
that for any $\alpha \in \Gamma(\M, \mathrm{T}^{0,1} \M)$, $X,Y \in \XM$ we have
$\nabla_X \alpha (Y) = \alpha(\nabla_X Y) - X(\alpha(Y))$. In particular, we
have that $\nabla_X \alpha \in \Gamma(\M, \mathrm{T}^{1,0} \M)$. Note that for any
$X \in \XM$ and $\alpha, \beta \in \mathrm{T}^{\abs{1}} \M$ we have
$\nabla_X (\alpha \otimes \beta) = \nabla_X \alpha \otimes \beta + \alpha
\otimes \nabla_X \beta$. Similarly, we can define recursively $\nabla_X \alpha$
for any $\alpha \in \Gamma(\M, \mathrm{T}^{k,\ell}\M)$ with $k, \ell \in \nset$. Such an
extension is called a covariant derivative.

\paragraph{Parallel transport, geodesics and exponential mapping} Given a
connection, we can define the notion of parallel transport, which transports
vector fields along a curve. Let $\gamma: \ \ccint{0,1} \to \M$ be a smooth
curve. We define the covariant derivative along the curve $\gamma$ by
$D_{\dot{\gamma}}: \ \Xgamma \to \Xgamma$ similarly to the connection, where
$\Xgamma = \Gamma(\gamma(\ccint{0,1}), \TM)$. In particular if $\dot{\gamma}$
and $X \in \Xgamma$ can be extended to $\XM$ then we define
$D_{\dot{\gamma}}(X) = \nabla_{\dot{\gamma}}X \in \XM$. In what follows, we
denote $D = \nabla$ for simplicity. We say that $X \in \Xgamma$ is parallel to
$\gamma$ if for any $t \in \ccint{0,1}$, $\nabla_{\dot{\gamma}}X(t) = 0$. In
local coordinates, let $X \in \Xgamma$ be given for any $t \in \ccint{0,1}$ by 
$X = \sum_{i=1}^d a_i(t) E_i(t)$ (assuming that $\gamma([0,1])$ is entirely
contained in a local chart), then we have that for any $t \in \ccint{0,1}$ and
$k \in \{1, \dots, d\}$
\begin{equation}
  \label{eq:parallel_transport}
  \textstyle{\dot{a}_k(t) + \sum_{i,j=1}^d \Gamma_{i,j}^k(x(t)) \dot{x}_i(t) a_j(t) = 0  .}
\end{equation}
A curve $\gamma$ on $\M$ is said to be a geodesics if $\dot{\gamma}$ is parallel
to $\gamma$. Using \cref{eq:parallel_transport} we get that
\begin{equation}
  \label{eq:geodesics}
  \textstyle{\ddot{x}_k(t) + \sum_{i,j=1}^d \Gamma_{i,j}^k(x(t)) \dot{x}_i(t) \dot{x}_j(t) = 0  .}
\end{equation}
For more details on geodesics and parallel transport, we refer to \citet[Chapter
4]{lee2018introduction}. Parallel transport will be key to define the frame
bundle and the orthonormal frame bundle in \cref{sec:frame-bundle-orth}. In
addition, we have that parallel transport provides a linear isomorphism between
tangent spaces. Indeed, let $v \in \mathrm{T}_x \M$ and
$\gamma: \ \ccint{0,1} \to \M$ with $\gamma(0) = x$ a smooth curve. Then, there
exists a unique vector field $X^v \in \Xgamma$ such that $X^v(x) = v$ and $X^v$ is
parallel to $\gamma$. For any $t \in \ccint{0,1}$, we denote
$\Gamma_0^t: \mathrm{T}_{x} \M \to \mathrm{T}_{\gamma(t)} \M$ the linear
isomorphism such that $\Gamma_0^t(v) = X^v(\gamma(t))$.

For any $x \in \M$ and $v \in \mathrm{T}_x \M$ we denote
$\gamma^{x,v}: \ \ccint{0,\vareps^{x,v}}$ the geodesics (defined on the maximal
interval $\ccint{0, \vareps^{x,v}}$) on $\M$ such that $\gamma(0) = x$ and
$\dot \gamma(0) = v$. We denote
$\msu^x = \ensembleLigne{v \in \mathrm{T}_x \M}{\vareps^{x,v} \geq 1}$. Note
that $0 \in \msu^x$. For any $x \in \M$, we define the exponential mapping
$\exp_x: \ \msu^x \to \M$ such that for any $v \in \msu^x$,
$\exp_x(v) = \gamma^{x,v}(1)$. If for any $x \in \M$,
$\msu^x = \mathrm{T}_x \M$, the manifold is called \emph{geodesically
  complete}. Note that any connected compact manifold is geodesically
complete. As a consequence we have that there exists a geodesic between any two
points $x, y \in \M$ \cite[see][Lemma 6.18]{lee2018introduction}. For any
$x, y \in \M$, we denote $\mathrm{Geo}_{x,y}$ the sets of geodesics $\gamma$
such that $\gamma(0) = x$ and $\gamma(y) = 1$. For any $x, y \in \M$ we denote
$\Gamma_x^y(\gamma) : \ \mathrm{T}_x \M \to \mathrm{T}_y \M$ the linear
isomorphism such that for any $v \in \mathrm{T}_x \M$,
$\Gamma_x^y(v) = X^v(\gamma(1))$, where $\gamma \in \mathrm{Geo}_{x,y}$. Note
that for any $x \in \M$ there exists $\msv^x \subset \M$ such that
$x \in \msv^x$ and for any $y \in \msv^x$ we have that
$\absLigne{\mathrm{Geo}_{x,y}}=1$.  In this case, we denote
$\Gamma_x^y = \Gamma_x^y(\gamma)$ with $\gamma \in \mathrm{Geo}_{x,y}$.

\paragraph{Orthogonal projection} We will make repeated use of orthonormal
projections on manifolds. Recall that since $\M$ is a closed Riemannian manifold
we can use the Nash embedding theorem \citep{gunther1991isometric}. In the rest
of this paragraph, we assume that $\M$ is a Riemannian submanifold of $\rset^p$
for some $p \in \nset$ such that its metric is induced by the Euclidean
metric. In order to define the projection we introduce
\begin{equation}  
  \mathrm{unpp}(\M) = \ensembleLigne{x \in \rset^d}{\text{there exists a unique $\xi_x$ such that $\normLigne{x - \xi_x} = d(x, \M)$}}  . 
\end{equation}
Let $\mathcal{E}(\M) = \interior(\mathrm{unpp}(\M))$. By \citet[Theorem
1]{leobacher2021existence}, we have that $\M \subset \mathcal{E}(\M)$. We define
$\tilde{p}: \ \mathcal{E}(\M) \to \M$ such that for any $x \in \mathcal{E}(\M)$,
$\tilde{p}(x) = \xi_x$. Using \citet[Theorem 2]{leobacher2021existence}, we have
that $\tilde{p} \in \rmc^\infty(\rset^p, \M)$ and that for any $x \in \M$,
$\tilde{P}(x) = \rmd \tilde{p}(x)$ is the orthogonal projection on
$\mathrm{T}_x\M$. Since $\rset^p$ is normal and $\M$ and
$\mathcal{E}(\M)^\complementary$ are closed, there exists $\msf$ open such that
$\M \subset \msf \subset \mathcal{E}(\M)$. Let
$p \in \rmc^\infty(\rset^p, \rset^p)$ such that for any $x \in \msf$,
$p(x) = \tilde{p}(x)$ (given by Whitney extension theorem for
instance). Finally, we define $P: \ \rset^p \to \rset^p$ such that for any
$x \in \rset^p$, $P(x) = \rmd p(x)$. Note that for any $x \in \M$, $P(x)$ is the
orthogonal projection $\mathrm{T}_x \M$ and that
$P \in \rmc^\infty(\rset^p, \rset^p)$.


\subsection{Stochastic Differential Equations on manifolds}
\label{sec:stoch-diff-equat}


\paragraph{Stratanovitch integral} For reasons that will become clear in the
next paragraph it is easier to define Stochastic Differential Equations (SDEs)
on manifolds w.r.t the Stratanovitch integral \cite[Part II, Chapter
3]{kloeden:platen:2011}. We consider a filtered probability space
$(\Omega, (\mcf_t)_{t \geq 0}, \Pbb)$. Let $(\bfX_t)_{t \geq 0}$ and
$(\bfY_t)_{t \geq 0}$ be two real continuous semimartingales. We define the
quadratic covariation $([\bfX,\bfY]_t)_{t \geq 0}$ such that for any $t \geq 0$
\begin{equation}
  \textstyle{[\bfX,\bfY]_t = \bfX_t \bfY_t - \bfX_0\bfY_0 - \int_0^t \bfX_s \rmd \bfY_s - \int_0^t \bfY_s \rmd \bfX_s  . }
\end{equation}
We refer to \citet[Chapter IV]{revuz1999continuous} for more details on
semimartingales and quadratic variations. We denote $[\bfX] = [\bfX, \bfX]$. In
particular, we have that $([\bfX, \bfY]_t)_{t \geq 0}$ is an adapted continuous
process with finite-variation and therefore $[[\bfX, \bfY]] = 0$. Let
$(\bfX_t)_{t \geq 0}$ and $(\bfY_t)_{t \geq 0}$ be two real continuous
semimartingales, then we define the Stratanovitch integral as follows for any
$t \geq 0$
\begin{equation}
  \textstyle{ \int_0^t \bfX_s \circ \rmd \bfY_s = \int_0^t \bfX_s \rmd \bfY_s + (1/2) [\bfX, \bfY]_t  . }
\end{equation}
In particular, denoting $(\bfZ_t^1)_{t \geq 0}$ and $(\bfZ_t^2)_{t \geq 0}$ the
processes such that for any $t \geq 0$,
$\bfZ_t^1 = \int_0^t \bfX_s \circ \rmd \bfY_s$ and
$\bfZ_t^2 = \int_0^t \bfX_s \rmd \bfY_s$, we have that $[\bfZ^1] = [\bfZ^2]$. We
refer to \cite{kurtz1995stratonovich} for more details on Stratanovitch
integrals. Note that if for any $t \geq 0$,
$\bfX_t = \int_0^t f(\bfX_s) \circ \rmd \bfY_s$ with $\rmc^1(\rset, \rset)$,
then $[\bfX, \bfY]_t = \int_0^t f(\bfX_s) f'(\bfX_s) \rmd \bfY_s$. Assuming that
$f \in \rmc^3(\rset, \rset)$ we have that \cite[Chapter IV, Exercise
3.15]{revuz1999continuous}
\begin{equation}
  \label{eq:stratanovitch_lemma}
  \textstyle{ f(\bfX_t) = f(\bfX_0) + \int_0^t f'(\bfX_s) \circ \rmd \bfX_s  .}
\end{equation}
The proof relies on the fact that for any $t \geq 0$,
$\rmd [\bfX, f'(\bfX)]_t = f''(\bfX_t) \rmd [\bfX]_t$.  This result should be
compared with It\^o's lemma. In particular, Stratanovitch calculus satisfies the
ordinary chain rule making it a useful tool in differential geometry which
makes a heavy use of diffeomorphism.

\paragraph{SDEs on manifolds}
We define semimartingales and SDEs on manifold through the lens of their actions
on functions. A continuous $\M$-valued stochastic process $(\bfX_t)_{t \geq 0}$
is called a $\M$-valued semimartingale if for any $f \in \rmc^\infty(\M)$ we
have that $(f(\bfX_t))_{t \geq 0}$ is a real valued semimartingale. Let
$\ell \in \nset$, $V^{1:\ell} = \{ V_i\}_{i=1}^\ell \in \XM^\ell$ and
$Z^{1:\ell} = \{Z^i\}_{i=1}^\ell$ a collection of $\ell$ real-valued
semimartingales. A $\M$-valued semimartingale $(\bfX_t)_{t \geq 0}$ is said to
be the solution of $\SDE(V^{1:\ell}, Z^{1:\ell}, \bfX_0)$ up to a stopping
$\tau$ with $\bfX_0$ a $\M$-valued random variable if for all
$f \in \rmc^\infty(\M)$ and $t \in \ccint{0, \tau}$ we have 
\begin{equation}
  \textstyle{f(\bfX_t) = f(\bfX_0) + \sum_{i=1}^\ell \int_0^t V_i(f)(\bfX_s) \circ \rmd \bfZ^i_s  . } 
\end{equation}
Since the previous SDE is defined w.r.t the Stratanovitch integral we have that
if $(\bfX_t)_{t \geq 0}$ is a solution of $\SDE(V^{1:\ell}, Z^{1:\ell}, \bfX_0)$
and $\Phibf: \M \to \mathcal{N}$ is a diffeomorphism then $(\Phibf(\bfX_t))_{t \geq 0}$
is a solution of $\SDE(\Phibf_\star V^{1:\ell}, Z^{1:\ell}, \Phibf(\bfX_0))$,
where $\Phibf_\star$ is the pushforward operation \cite[see][Proposition
1.2.4]{hsu2002stochastic}. Because the vector fields $\{V_i\}_{i=1}^\ell$ are
smooth we have that for any $\ell \in \nset$,
$V^{1:\ell} = \{ V_i\}_{i=1}^\ell \in \XM^\ell$ and
$Z^{1:\ell} = \{Z^i\}_{i=1}^\ell$ a collection of $\ell$ real-valued
semimartingales, there exists a unique solution to
$\SDE(V^{1:\ell}, Z^{1:\ell}, \bfX_0)$ \cite[see][Theorem
1.2.9]{hsu2002stochastic}.


\subsection{Frame bundle and orthonormal frame bundle}
\label{sec:frame-bundle-orth}

We now introduce the concepts of frame bundle and orthonormal bundle over the
manifold $\M$. These concepts are useful to define stochastic processes on $\M$
using Euclidean stochastic processes. In particular, we will see that a Brownian
motion on the manifold can be linked to the Euclidean Brownian motion using the
orthonormal bundle. For any $x \in \M$, a frame at $x$ is an isomorphism
$f: \ \rset^d \to \mathrm{T}_x \M$. Note that $f$ is equivalent to the choice of
a basis in $\mathrm{T}_x \M$. We denote $\mathrm{F}_x \M$ the set of frames at
$p$. The frame bundle denoted $\FM$ is given by
$\FM = \sqcup_{x \in \M} \mathrm{F}_x \M$. The frame bundle can be given a
smooth structure and is therefore a $d + d^2$-dimensional manifold. Similarly,
for any $x \in \M$, an orthonormal frame at $x$ is a linear isometry
$f: \ \rset^d \to \mathrm{T}_x \M$. Note that $f$ is equivalent to the choice of
an orthonormal basis in $\mathrm{T}_x \M$. We denote $\mathrm{O}_p \M$ the set
of orthonormal frames at $p$. The orthonormal frame bundle denoted $\OM$ is
given by $\OM = \sqcup_{x \in \M} \mathrm{O}_x \M$. The orthonormal frame bundle
can be given a smooth structure and is therefore a $d + d(d-1)/2$-dimensional
manifold. We denote $\pi: \ \FM \to \M$ the smooth projection such that for any
$u = (x,f) \in \FM$, $\pi(u) = x$. Note that the restriction of $\pi$ to the
orthonormal bundle is also smooth.  Frame bundles and orthonormal bundles are
primary examples of principal bundles and we refer to \cite{kolar2013natural}
for more details.

One key element of frame bundles and orthonormal bundles is their link with the
connections on $\M$. Let $u = (x, f) \in \FM$ and
$U \in \mathrm{T}_u \mathrm{F}M$. $U$ is said to be vertical if there exists a
smooth curve $u: \ \ccint{0,1} \to \FM$ such that for any $t \in \ccint{0,1}$,
$\pi(u(t)) = x$ and $\dot u(0) = U$. We say that $U$ is tangent to the fibre
$\mathrm{F}_{\pi(u)}\M$. The space of vertical tangent vectors is called the
vertical space and is denoted $\mathrm{V}_u \FM$. We have that
$\mathrm{dim}(\mathrm{V}_u \mathrm{F}\M) = d^2$. We now define the horizontal
space as follows. Let $u: \ \ccint{0,1} \to \FM$ be a smooth curve. We say that
$u = (f,x)$ is horizontal if for any $t \in \ccint{0,1}$ and
$i \in \{1, \dots, d\}$, $\nabla_{\dot x} (f e_i)(t) = 0$, where
$\{e_i\}_{i=1}^d$ is the canonical basis of $\rset^d$. In other words, the
horizontal curve corresponds to the parallel transport of a frame along a smooth
curve in $\M$. Let $u = (x, f) \in \FM$ and $U \in \mathrm{T}_u
\mathrm{F}M$. $U$ is said to be horizontal if there exists a smooth horizontal
curve $u: \ \ccint{0,1} \to \FM$ such that $\dot u(0) = U$. The space of
horizontal tangent vectors is called the horizontal space and is denoted
$\mathrm{H}_u \FM$. Let $v \in \rset^d$, we define the vector field
$H_v \in \mathcal{X}(\FM)$ such that for any $u \in \FM$, $H_v(u) = \dot u(0)$
with $\gamma=(x,f): \ \ccint{0,1} \to \FM$ a smooth curve on $\FM$ such that
$\dot x(0) = e(0)v$ and $\gamma(0) = u$. The existence of $H_v$ for any
$v \in \rset^d$ is discussed in \citet[p.69-70]{kobayashi1963foundations} and
$H_v$ is called the horizontal lift of $v$. For any $i \in \{1, \dots, d\}$ we
denote $H_i = H_{e_i}$ where $\{e_i\}_{i=1}^d$ is the canonical basis of
$\rset^d$. In particular, since any horizontal curve is entirely specified by
$\gamma(0) = (x(0), f(0))$ and $\dot{x}(0)$, we get that
$\mathrm{dim}(\mathrm{H}_u \FM) = d$ for any $u \in \FM$.

Consider a connection $\nabla$ on $\M$. Note that for any $u = (x,f) \in \FM$, we have
$\mathrm{T}_u \FM = \mathrm{T}_u \M \oplus \mathrm{V}_u \FM$. In local
coordinates $\{x_i\}_{i=1}^d$, we denote $\{X_i\}_{i=1}^d$ a basis of
$\mathrm{T}_x \M$. For any $j \in \{1, \dots, j\}$, there exist
$\{f_{i,j}\}_{i=1}^d$ such that $f e_j = \sum_{i=1}^d f_{i,j} X_i$ (note that
$\{f_{i,j}\}_{1 \leq i,j \leq d}$ can be interpreted as the matrix transforming
a vector of $\rset^d$ into a vector of $\mathrm{T}_x\M$ expressed in the basis
$\{X_i\}_{i=1}^d$). In particular, we have that
$\{x_k, f_{i,j}\}_{1 \leq i,j, k \leq d}$ are local coordinates for $\FM$. We
denote by $\{X_k, X_{i,j}\}_{1 \leq i,j,k \leq d}$ the associated basis in
$\mathrm{T}_u \FM$ for any $u \in \msu$, where $\msu$ is an open subset of $\FM$
on which the local coordinates are well-defined. Leveraging properties of
parallel transport, we have that for any $j \in \{1, \dots, d\}$ and $u \in \msu$
\begin{equation}
  \label{eq:horizontal_lift}
  \textstyle{ H_j(u) = \sum_{i=1}^d f_{i,j} X_i - \sum_{\ell, m=1}^d \{ \sum_{i, k=1}^d f_{i,j} f_{k,m} \Gamma_{i,k}^\ell\} X_{\ell,m}  ,}
\end{equation}
where we recall that $\{\Gamma_{i,j}^k\}_{1 \leq i,j,k \leq d}$ are the
Christoffel symbols of the connection in local coordinates.  In particular, it
is clear that for any $u \in \FM$, $\{H_i(u)\}_{i=1}^d$ is a basis of
$\mathrm{H}_u \FM$ and that $\mathrm{H}_u \FM \cap \mathrm{V}_u \FM = \{0\}$,
hence $\mathrm{T}_u \FM = \mathrm{H}_u \FM \oplus \mathrm{V}_u \FM$. Using
\cref{eq:horizontal_lift} we have that the horizontal space is entirely defined
by the connection $\nabla$. Reciprocally, any smooth linear complement of the
vertical space gives rise to a connection \cite[see][Section
11.11]{kolar2013natural}.

We now illustrate how we can go from a smooth curve on $\M$ (equipped with a
connection $\nabla$) to a smooth curve on $\rset^d$. First, let
$x: \ \ccint{0,1} \to \M$ be a smooth curve on manifold. Define
$f(0) \in \mathrm{F}_{x(0)} \M$ and consider $u: \ \ccint{0,1} \to \FM$ the
smooth horizontal curve associated with $x$ and starting frame $f(0)$. Now
consider the antidevelopment of $u$ given by the smooth curve
$z: \ \ccint{0,1} \to \rset^d$ such that for any $t \in \ccint{0,t}$
\begin{equation}
  \textstyle{ z(t) = \int_0^t f(s)^{-1} \dot x(s) \rmd s   . }
\end{equation}
We now show how a smooth curve on $\rset^d$ gives rise to a smooth curve in
$\M$. First, note that for any $t \in \ccint{0,1}$, we have that
$\dot u (t) = \sum_{i=1}^d H_i(u(t)) \dot z_i(t)$. Hence, specifying $u(0)$ any
smooth curve $z$ on $\rset^d$ is associated to a smooth curve on $\FM$. We
obtain a smooth curve on $\M$ by considering $x = \pi(u)$. In the next section,
we present similar ideas when smooth curves are replaced by semimartingales.

\subsection{Horizontal lift and stochastic development}
\label{sec:horiz-lift-stoch}

We are now ready to present the notion of horizontal semimartingale, which is
key to draw the link between semimartingales on $\M$ and semimartingales on
$\rset^d$. We follow the presentation of \citet[Section
2.3]{hsu2002stochastic}. Again, we consider a filtered probability space
$(\Omega, (\mcf_t)_{t \geq 0}, \Pbb)$. All the semimartingales we consider are
defined w.r.t this filtered probability space. We assume that the manifold $\M$
is equipped with a connection $\nabla$.

\begin{definition}[Stochastic development]
  Let $(\bfZ^{1:d}_t)_{t \geq 0} = \{(\bfZ_t^i)_{t \geq 0}\}_{i=1}^d$ be a
  collection of real-valued semimartingales.  Let $(\bfU_t)_{t \geq 0}$ be the
  $\FM$ semimartingale solution of $\SDE(H^{1:d}, \bfZ^{1:d}, \bfU_0)$ with
  $H^{1:d} = \{H_i\}_{i=1}^d$. $(\bfU_t)_{t \geq 0}$ is called the \emph{stochastic
    development} of $\bfZ^{1:d}$ on $\FM$. Similarly, the $\M$-valued
  semimartingale $(\bfX_t)_{t \geq 0} = (\pi(\bfU_t))_{t \geq 0}$ is called the
  \emph{stochastic development} of $\bfZ^{1:d}$ on $\M$.
\end{definition}

The previous definition allows to transfer a semimartingale on $\rset^d$ to a
semimartingale on $\M$ in an \emph{intrinsic} manner. Reciprocally, we also aim
at transferring a semimartingale on $\M$ to a semimartingale on $\rset^d$.

\begin{definition}[Horizontal lift and antivelopment]
  Let $(\bfX_t)_{t \geq 0}$ be a $\M$-valued semimartingale. If there exist a
  $\FM$-valued semimartingale $(\bfU_t)_{t \geq 0}$ and
  $(\bfZ^{1:d}_t)_{t \geq 0} = \{(\bfZ_t^i)_{t \geq 0}\}_{i=1}^d$ a collection
  of real-valued semimartingales such that
  $(\bfX_t)_{t \geq 0} = (\pi(\bfU_t))_{t \geq 0}$ and $(\bfU_t)_{t \geq 0}$ is
  solution of $\SDE(H^{1:d}, \bfZ^{1:d}, \bfU_0)$ with
  $H^{1:d} = \{H_i\}_{i=1}^d$ then $(\bfU_t)_{t \geq 0}$ is called the
  \emph{horizontal lift} of $(\bfX_t)_{t \geq 0}$ and
  $(\bfZ^{1:d}_t)_{t \geq 0}$ the \emph{antidevelopment} of
  $(\bfX_t)_{t \geq 0}$.
\end{definition}

The existence of an horizontal lift and an antidevelopment is not
trivial. Considering the Nash embedding theorem 
\citep[see for example][]{gunther1991isometric}, it is possible to show the existence
and uniqueness of these processes (up to initialization). Without loss of
generality, we can then assume that $\M \subset \rset^p$ and for any $x \in \M$,
$\mathrm{T}_x \M \subset \rset^p$ with $p \geq d(d+1)/2$ (and
$p \leq \max(d(d+5)/2, d(d+3)/2+5)$). For any $x \in \M$, we denote
$P(x): \ \rset^p \to \mathrm{T}_x \M$ the projection operator. In addition for
any $x \in \M$, we denote $\{P_i(x)\}_{i=1}^p = \{P(x) e_i\}_{i=1}^p$, where
$\{e_i\}_{i=1}^p$ is the canonical basis of $\rset^p$. Note that
$\{P_i\}_{i=1}^p \in \XM^p$. In addition for any $x \in \M$ we denote
$\{x^i\}_{i=1}^p$ its coordinates in $\rset^p$, \ie \ for any
$i \in \{1, \dots, p\}$, $x^i = \langle x, e_i \rangle$. In particular, if
$(\bfX_t)_{t \geq 0}$ is a $\M$-valued process then for any
$i \in \{1, \dots, p\}$,
$(\bfX_t^i)_{t \geq 0} = (\langle \bfX_t, e_i \rangle)_{t \geq 0}$ is a
real-valued process. If $(\bfX_t)_{t \geq 0}$ is a $\M$-valued semimartingale
then it is the solution of $\SDE(\{P_i\}_{i=1}^p, \{\bfX^i\}_{i=1}^p, \bfX_0)$
\cite[see][Lemma 2.3.3]{hsu2002stochastic}. Then, a candidate for the horizontal
lift of $(\bfX_t)_{t \geq 0}$ is given by
$(\bfU_t)_{t \geq 0}=(\bfX_t, \bfE_t)_{t \geq 0}$ solution of
$\SDE(\{P_i^\star\}_{i=1}^p, \{\bfX^i\}_{i=1}^p, \bfU_0)$, where for any
$i \in \{1,\dots,p\}$, $P_i^\star(u) = H_{f^{-1}P_i(\pi(u))}(u)$ and
$\bfX_0 = \pi(\bfU_0)$. We have that $(\bfU_t)_{t \geq 0}$ is the stochastic
development of $\{(\bfZ_t^i)_{t \geq 0}\}_{i=1}^d$ where for any $t \geq 0$,
$\bfZ_t = \sum_{i=1}^p \int_0^t \bfE_s^{-1} P_i(\bfX_s) \circ \rmd \bfX_s^i$
 \cite[see][Theorem 2.3.4]{hsu2002stochastic}. Finally, we have that given
$\bfU_0$, $(\bfU_t)_{t \geq 0}$ is the unique horizontal lift of
$(\bfX_t)_{t \geq 0}$ and $(\bfZ_t)_{t \geq 0}$ is the unique antidevelopment of
$(\bfX_t)_{t \geq 0}$ \cite[see][Theorem 2.3.5]{hsu2002stochastic}.

\subsection{Brownian motion on manifolds}
\label{sec:brown-moti-manif}

In this section, we introduce the notion of Brownian motion on manifolds. We
derive some of its basic convergence properties and provide alternative
definitions (stochastic development, isometric embedding, random walk
limit). These alternative definitions are the basis for our alternative
methodologies to sample from the time-reversal. To simplify our discussion, we
assume that $\M$ is a connected compact Riemannian manifold equipped with the
Levi-Civita connection $\nabla$. We denote $p_{\textup{ref}}b$ the Haussdorff measure of
the manifold (which coincides with the measure associated with the Riemannian
volume form \citep[see][Theorem 2.10.10]{federer2014geometric} and
$p_{\textup{ref}} = p_{\textup{ref}}b / p_{\textup{ref}}(\M)$ the associated probability measure.

\paragraph{Gradient, divergence and Laplace operators}
Let $f \in \rmc^{\infty}(\M)$. We define $\nabla f \in \XM$ such that for any
$X \in \XM$ we have $\langle X, \nabla f \rangle_{\M} = X(f)$. Let
$\{X_i\}_{i=1}^d \in \XM^d$ such that for any $x \in \M$, $\{X_i(x)\}_{i=1}^d$
is an orthonormal basis of $\mathrm{T}_x \M$. Then, we define
$\dive: \ \XM \to \rmc^\infty(\M)$ (linear) 
such that for any $X \in \XM$,
$\dive(X) = \sum_{i=1}^d \prodM{\nabla_{X_i}X}{X_i}$. The following Stokes
formula (also called divergence theorem, see \citet[p.51]{lee2018introduction})
holds for any $f \in \rmc^\infty(\M)$ and $X \in \XM$,
$\int_{M} \dive(X)(x) f(x) \rmd p_{\textup{ref}}(x) = - \int_M X(f)(x) \rmd
p_{\textup{ref}}(x)$. Let $X = \sum_{i=1}^d a_i X_i$ in local coordinates.  Using the
Stokes formula and the definition of the gradient we get that in local
coordinates
\begin{equation}
\textstyle{  \nabla f = \sum_{i,j=1}^d g^{i,j} \partial_i f X_j  ,  \qquad \dive(X) = \det(G)^{-1/2} \sum_{i=1}^d \partial_i(\det(G)^{1/2} a_i)  . }
\end{equation}
The Laplace-Beltrami operator is given by 
$\Delta_{\M} : \ \rmc^\infty(M) \to \rmc^\infty(M)$ and for any
$f \in \rmc^\infty(M)$ by $\Delta_{\M}(f) = \dive(\grad(f))$. In local
coordinates we obtain 
$\Delta_{\M}(f) = \det(G)^{-1/2} \sum_{i=1}^d \partial_i (\det(G)^{1/2}
\sum_{j=1}^d g^{i,j} \partial_j f)$. Using the Nash isometric embedding theorem
\citep{gunther1991isometric} we will see that $\Delta_{\M}$ can always be
written as a sum of squared operators. However, this result requires an
\emph{extrinsic} point of view as it relies on the existence of projection
operators. In contrast, if we consider the orthonormal bundle $\OM$ we can
define the Laplace-Bochner operator
$\Delta_{\OM}: \ \rmc^\infty(\OM) \to \rmc^\infty(\OM)$ as
$\Delta_{\OM} = \sum_{i=1}^d H_i^2$, where we recall that for any
$i \in \{1, \dots, d\}$, $H_i$ is the horizontal lift of $e_i$. In this case,
$\Delta_{\OM}$ is a sum of squared operators and we have that for any
$f \in \rmc^\infty(\M)$, $\Delta_{\OM}(f \circ \pi) = \Delta_{\M}(f)$
\cite[see][Proposition 3.1.2]{hsu2002stochastic}. Being able to express the various
Laplace operators as a sum of squared operators is key to express the associated
diffusion process as the solution of an SDE.

\paragraph{Alternatives definitions of Brownian motion}

We are now ready to define a Brownian motion on the manifold $\M$. Using the
Laplace-Beltrami operator, we can introduce the Brownian motion through the lens
of diffusion processes.

\begin{definition}[Brownian motion]
  Let $(\bfB_t^\M)_{t \geq 0}$ be a $\M$-valued semimartingale.
  $(\bfB_t^\M)_{t \geq 0}$ is a Brownian motion on $\M$ if for any
  $f \in \rmc^\infty(\M)$, $(\bfM_t^f)_{t \geq 0}$ is a local martingale where
  for any $t \geq 0$
  \begin{equation}
    \textstyle{\bfM_t^f = f(\bfB_t^\M) - f(\bfB_0^\M) - (1/2)\int_0^t \Delta_{\M}f(\bfB_s^\M) \rmd s  .}
  \end{equation}
\end{definition}

Note that this definition is in accordance with the definition of the Brownian
motion as a diffusion process in the Euclidean space $\rset^d$, since in this
case $\Delta_{\M} = \Delta$. As emphasized in the previous section any
semimartingale on $\M$ can be associated to a process on $\FM$ (or $\OM$) and a
process on $\rset^d$. The proof of the following result can be found in
\citet[Propositions 3.2.1 and 3.2.2]{hsu2002stochastic}.

\begin{proposition}[Intrinsic view of Brownian motion]
  \label{prop:intrinsic_brownian}
  Let $(\bfB_t^\M)_{t \geq 0}$ be a $\M$-valued semimartingales. Then
  $(\bfB_t^\M)_{t \geq 0}$ is a Brownian motion on $\M$ if and only on the
  following conditions hold:
  \begin{enumerate}[label=\alph*)]
  \item The horizontal lift $(\bfU_t)_{t \geq 0}$ is a $\Delta_{\OM}/2$
    diffusion process, \ie \ for any $f \in \rmc^\infty(\OM)$, we have that
    $(\bfM_t^f)_{t \geq 0}$ is a local martingale where for any $t \geq 0$
  \begin{equation}
    \textstyle{\bfM_t^f = f(\bfU_t) - f(\bfU_0) - (1/2)\int_0^t \Delta_{\OM}f(\bfU_s) \rmd s  .}
  \end{equation}    
\item The stochastic antidevelopment of $(\bfB_t^\M)_{t \geq 0}$ is a
  $\rset^d$-valued Brownian motion $(\bfB_t)_{t \geq 0}$.
  \end{enumerate}
\end{proposition}

In particular the previous proposition provides us with an \emph{intrisic} way
to sample the Brownian motion on $\M$ with initial condition $\bfB_0^\M$. First
sample $(\bfU_t)_{t \geq 0}$ solution of $\SDE(H^{1:d}, \bfB^{1:d}, \bfU_0)$
with $H^{1:d} = \{H_i\}_{i=1}^d$ and $\pi(\bfU_0) = \bfB_0^\M$ and $\bfB^{1:d}$ the
Euclidean $d$-dimensional Brownian motion. Then, we recover the $\M$-valued
Brownian motion $(\bfB_t^\M)_{t \geq 0}$ upon letting
$(\bfB_t^\M)_{t \geq 0} = (\pi(\bfU_t))_{t \geq 0}$.
% habermann

We now consider an \emph{extrinsic} approach to the sampling of Brownian motions
on $\M$. Using the Nash embedding theorem \citep{gunther1991isometric}, there
exists $p \in \nset$ such that without loss of generality we can assume that
$\M \subset \rset^p$. For any $x \in \M$, we denote
$P(x): \ \rset^p \to \mathrm{T}_x \M$ the projection operator. In addition for
any $x \in \M$, we denote $\{P_i(x)\}_{i=1}^p = \{P(x) e_i\}_{i=1}^p$, where
$\{e_i\}_{i=1}^p$ is the canonical basis of $\rset^p$. For any
$i \in \{1, \dots, p\}$, we smoothly extend $P_i$ to $\rset^p$. In this case, we
have the following proposition \cite[Theorem 3.1.4]{hsu2002stochastic}:

\begin{proposition}[Extrinsic view of Brownian motion]
  \label{prop:extrinsic_brownian}
  For any $f \in \rmc^{\infty}(\M)$ we have that
  $\Delta_M(f) = \sum_{i=1}^p P_i(P_i(f))$. Hence, we have that
  $(\bfB_t^\M)_{t \geq 0}$ solution of
  $\SDE(\{P_i\}_{i=1}^{p}, \bfB^{1:p}, \bfB_0^\M)$ with $\bfB_0^\M$ a $\M$-valued
  random variable and $\bfB^{1:p}$ a $\rset^p$-valued Brownian motion.
\end{proposition}

The second part of this proposition, stems from the fact that any solution of
$\SDE(\{V_i\}_{i=1}^{\ell}, \bfB^{1:\ell}, \bfX_0)$, where $\bfX_0$ is a
$\M$-valued random variable and $\bfB^{1:\ell}$ a $\rset^\ell$-valued Brownian
motion is a diffusion process with generator $\generator$ such that for any
$f \in \rmc^\infty(\M)$, $\generator(f) = \sum_{i=1}^\ell V_i(V_i(f))$. The
\emph{extrinsic} approach is particularly convenient since the SDE appearing in 
\cref{prop:extrinsic_brownian} can be seen as an SDE on the Euclidean space
$\rset^p$. 

We finish this paragraph, by investigating the behavior of the Brownian motion
in local coordinates. For simplicity, we assume here that we have access to a
system of global coordinates. In the case where the coordinates are strictly
local then we refer to \citet[Chapter 5, Theorem 1]{ikeda1989sto} for a
construction of a global solution by patching local solutions. We denote
$\{X_k, X_{i,j}\}_{1 \leq i,j,k \leq d}$ such that for any $u \in \FM$,
$\{X_k(u), X_{i,j}(u)\}_{1 \leq i,j,k \leq d}$ is a basis of $\mathrm{T}_u \FM$,
similarly as in the previous section. Using \cref{eq:horizontal_lift} we get
that $(\bfU_t)_{t \geq 0} = (\{\bfX^k_t, \bfE_t^{i,j}\}_{1 \leq i,j,k \leq d})$
obtained in \cref{prop:intrinsic_brownian} is given in the global coordinates for
any $i,j,k \in \{1, \dots, d\}$ by
\begin{equation}
  \textstyle{
    \rmd \bfX_t^k = \sum_{j=1}^d \bfE_t^{k,j} \circ \rmd \bfB_t^k  , \qquad \rmd \bfE_t^{i,j} = - \sum_{n=1}^d \{\sum_{\ell, m=1}^d \bfE_t^{\ell,n}\bfE_t^{m,j} \Gamma_{\ell,m}^{i}(\bfX_t)\} \circ \rmd \bfB_t^n  . 
    }
  \end{equation}
  By definition of the Stratanovitch integral we have that for any $k \in \{1, \dots, d\}$
  \begin{equation}
    \textstyle{
      \rmd \bfX_t^k = \sum_{j=1}^d \{ \bfE_t^{k,j} \rmd \bfB_t^k +(1/2) \rmd [\bfE_t^{k,j}, \bfB_t^j]_t \}  .
      }
    \end{equation}
    Let $(\bfM_t)_{t \geq 0} = (\{\bfM_t^k\}_{k=1}^d)_{t \geq 0}$ such that for
    any $t \geq 0$ and $k \in \{1, \dots, d\}$
    $\bfM_t^k = \sum_{j=1}^d \int_0^t \bfE_t^{k,j} \rmd \bfB_t^k$. We obtain
    that $\rmd \bfM_t = G(\bfX_t)^{-1/2} \rmd \bfB_t$ for some $d$-dimensional
    Brownian motion $(\bfB_t)_{t \geq 0}$, using L\'evy's characterization of
    Brownian motion. In addition, we have that for any
    $k, j \in \{1, \dots, d\}$
    \begin{equation}
      \textstyle{[\bfE^{k,j}, \bfB^j]_t = -\sum_{\ell, m=1}^d \int_0^t \bfE_t^{\ell, j} \bfE_t^{m,j} \Gamma_{\ell, m}^k(\bfX_t) \rmd t }
    \end{equation}
    Hence, using this result and the fact that
    $\sum_{j=1}^d \bfE_t^{\ell, j} \bfE_t^{m,j} = g^{\ell,m}(\bfX_t)$, we get
    that for any $k \in \{1, \dots, d\}$
    \begin{equation}
      \textstyle{\rmd \bfX_t^k =-  (1/2) \sum_{\ell, m=1}^d g^{\ell,m}(\bfX_t) \Gamma_{\ell, m}^k(\bfX_t) \rmd t + (G(\bfX_t)^{-1/2} \rmd \bfB_t)^k  . }
    \end{equation}
    Note that this result could also have been obtained using the expression of
    the Laplace-Beltrami in local coordinates.


    \paragraph{Brownian motion and random walks}

    In the previous paragraph we consider three SDEs to obtain a Brownian motion
    on $\M$ (stochastic development, isometric embedding and local
    coordinates). In this section, we summarize results from
    \cite{jorgensen1975central} establishing the limiting behavior of Geodesic
    Random Walks (GRWs) when the stepsize of the random walk goes to $0$. This will be
    of particular interest when considering the time-reversal process. We start
    by defining the geodesic random walk on $\M$, following \citet[Section
    2]{jorgensen1975central}.

    Let $\{ \nu_x \}_{x \in \M}$ such that for any $x \in \M$,
    $\nu_x: \mcb{\mathrm{T}_x \M} \to \ccint{0,1}$ with
    $\nu_x(\mathrm{T}_x \M) =1$, \ie \ for any $x \in \M$, $\nu_x$ is a
    probability measure on $\mathrm{T}_x \M$. Assume that for any $x \in \M$,
    $\int_{\M} \normLigne{v}^3 \rmd \nu_x(v)< +\infty$. In addition assume that
    there exists $\mu^{(1)} \in \XM$ and $\mu^{(2)} \in \XMdeux$, where
    $\XMdeux$ is the section
    $\Gamma(\M, \sqcup_{x \in \M} \mathcal{L}(\mathrm{T}_x \M))$, such that for
    any $x \in \M$, $\int_{\M} v \rmd \nu_x(v) = \mu^{(1)}(x)$ and
    $\int_{\M} v \otimes v \rmd \nu_x(v) = \mu^{(2)}(x)$. In addition, we assume
    that for any $x \in \M$,
    $\Sigma(x) = \mu^{(2)}(x) - \mu^{(1)}(x) \otimes \mu^{(1)}(x)$ is strictly
    positive definite and that there exists $\Ltt \geq$ such that for any
    $x, y \in \M$, $\tvnorm{\nu_x - \nu_y} \leq \Ltt d(x,y)$. Where we have that
    for any $\nu_1 \in \Pens(\mathrm{T}_x \M)$ and $\nu_2 \in \Pens(\mathrm{T}_y \M)$,
    \begin{equation}
      \tvnorm{\nu_x - \nu_y} = \sup \ensembleLigne{\nu_1[f] - \Gamma_{x}^y(\gamma)_\# \nu_2[f]}{\gamma \in \mathrm{Geo}_{x,y}, \ f \in \rmc(\mathrm{T}_x \M)}  . 
    \end{equation}
    Note that if $d(x,y) \leq \vareps$ then for some $\vareps > 0$ we have that $\abs{\mathrm{Geo}_{x,y}}=1$.


    \begin{definition}[Geodesic random walk]
      Let $X_0$ be a $\M$-valued random variable.  For any $\gamma > 0$, we
      define $(\bfX_t^{\gamma})_{t \geq 0}$ such that $\bfX_0^\gamma = X_0$ and
      for any $n \in \nset$ and $t \in \ccint{0, \gamma}$,
      $\bfX_{n\gamma + t} = \exp_{\bfX_{n \gamma}}[t\gamma \{ \mu_n +
      (1/\sqrt{\gamma}) (V_n - \mu_n)\}]$, where $(V_n)_{n \in \nset}$ is a sequence
      of random variables in such that for any $n \in \nset$, $V_n$
      has distribution $\nu_{\bfX_{n \gamma}}$ conditionally to $\bfX_{n \gamma}$.
    \end{definition}

    For any $\gamma > 0$, the process
    $(X_n^\gamma)_{n \in \nset} = (\bfX_{n \gamma}^\gamma)_{n \in \nset}$ is
    called a geodesic random walk. In particular, for any $\gamma>0$ we denote
    $(\Rker_n^{\gamma})_{n \in \nset}$ the sequence of Markov kernels such that
    for any $n \in \nset$, $x \in \M$ and $\msa \in \mcb{\M}$ we have that
    $\updelta_x \Rker(\msa) = \Pbb(X_n^\gamma \in \msa)$, with $X_0^\gamma =
    x$. The following theorem establishes that the limiting dynamics of a
    geodesic random walk is associated with a diffusion process on $\M$ whose
    coefficients only depends on the properties of $\nu$ \cite[see][Theorem
    2.1]{jorgensen1975central}.

    \begin{theorem}[Convergence of geodesic random walks]
      \label{thm:jorgensen_appendix}
      For any $t \geq 0$, $f \in \rmc(\M)$ and $x \in \M$ we have that
      $\lim_{\gamma \to 0} \normLigne{ \Rker_{\gamma}^{\ceil{t/\gamma}}[f] -
        \Pker_t[f]}_{\infty} = 0$, where $(\Pker_t)_{t \geq 0}$ is the
      semi-group associated with the infinitesimal generator
      $\generator: \ \rmc^\infty(\M) \to \rmc^\infty(\M)$ given for any
      $f \in \rmc^\infty(\M)$ by
      $\generator(f) = \langle p_{\textup{ref}}^1, \nabla f \rangle_{\M} + (1/2) \langle
      \Sigma, \nabla^2f \rangle_{\M}$.
    \end{theorem}   

    In particular if $\mu^{(1)} = 0$ and $\mu^{(2)} = \Id$ then the random walk
    converges towards a Brownian motion on $\M$ in the sense of the convergence
    of semi-groups. For any $x \in \M$ in local coordinates we have that
    $\Phi_\# \nu_x$ has zero mean and covariance matrix $G(x)$, where $\Phi$ is
    a local chart around $x$ and $G(x) = (g_{i,j}(x))_{1 \leq i,j \leq d}$ the
    coordinates of the metric in that chart.

    
\paragraph{Convergence of Brownian motion}

We finish this section with a few considerations regarding the convergence of
the Brownian motion on $\M$. Since we have assumed that $\M$ is compact we have
that there exist $(\Phi_k)_{k \in \nset}$ an orthonormal basis of $\Delta_\M$ in
$\mathrm{L}^2(p_{\textup{ref}})$, $(\lambda_k)_{k \in \nset}$ such that for any
$i, j \in \nset$, $i \leq j$, $\lambda_i \leq \lambda_j$ and $\lambda_0 = 0$, $\Phi_0=1$ and
for any $k \in \nset$, $\Delta_\M \Phi_k = -\lambda_k \Phi_k$. For any $t \geq 0$
and $x,y \in \M$,
$p_t(x,y) = \sum_{k \in \nset} \exp[-\lambda_k t] \Phi_k(x) \Phi_k(y)$ where for
any $f \in \rmc^\infty$ we have
\begin{equation}
  \textstyle{\expeLigne{f(\bfB_t^{\M,x})} = \int_\M p_t(x,y) f(y) \rmd p_{\textup{ref}}(y)  , }
\end{equation}
where $(\bfB_t^{\M,x})_{t \geq 0}$ is the Brownian motion on $\M$ with $\bfB_0^{\M,x} = x$
and $p_{\textup{ref}}$ is the probability measure associated with the Haussdorff measure on
$\M$. we also have the following result \cite[see][Proposition
2.6]{urakawa2006convergence}.

\begin{proposition}[Concergence of Brownian motion]
\label{prop:brownian_conv_repeat}
  For any $t > 0$, $\Pker_t$ admits a density $p_t$ w.r.t $p_{\textup{ref}}$ and
  $p_{\textup{ref}} \Pker_t = p_{\textup{ref}}$, \ie \ $p_{\textup{ref}}$ is an invariant measure for
  $(\Pker_t)_{t \geq 0}$. In addition, if there exists $C, \alpha \geq 0$ such
  that for any $t \in \ocint{0,1}$, $p_t(x,x) \leq C t^{-\alpha /2}$ then 
  for any $\pizero \in \Pens(\M)$ and for any $t \geq 1/2$ we have 
  \begin{equation}
    \textstyle{\tvnorm{\pizero \Pker_t - p_{\textup{ref}}} \leq C^{1/2} \rme^{\lambda_1 /2} \rme^{-\lambda_1 t}  ,}
  \end{equation}
  where $\lambda_1$ is the first non-negative eigenvalue of $-\Delta_\M$ in
  $\mathrm{L}^2(p_{\textup{ref}})$ and we recall that $(\Pker_t)_{t \geq 0}$ is the
  semi-group of the Brownian motion.
\end{proposition}
A review on lower bounds on the first positive eigenvalue
of the Laplace-Beltrami operator can be found in \citep{he2013lower}. These lower
bounds usually depend on the Ricci curvature of the manifold or its diameter. We
conclude this section by noting that in the non-compact case \citep{li1986large}
establishes similar estimates in the case of a manifold with non-negative Ricci
curvature and maximal volume growth.


%%% Local Variables:
%%% mode: latex
%%% TeX-master: "main"
%%% End:


\section{Difference between ODE and SDE likelihood computations}
\label{sec:diff-betw-ode}

In this section, we show that the likelihood computation from
\cite{song2020score} does not coincide with the likelihood computation
obtained with the SDE model. We present our findings in the Riemannian setting
but our conclusions can be adapted to the Euclidean setting with arbitrary
forward dynamics. Recall that we consider a Brownian motion on the manifold as a forward process
$(\bfB_t^\M)_{t \in \ccint{0,T}}$ with $\{p_t\}_{t=0}^T$ the associated family
of densities. We have that for any $t \in \ccint{0,T}$ and $x \in \M$
\begin{equation}
  \label{eq:forward}
  \partial_t p_t(x) = \tfrac{1}{2} \Delta p_t(x) = \dive(\tfrac{1}{2}p_t \nabla \log p_t )(x)  . 
\end{equation}


\paragraph{ODE model}
In the case of the ODE model we define $(\bfX_t)_{t \in \ccint{0,T}}$ such that
$\bfX_0$ has distribution $\pi$ and satisfies
$\rmd \bfX_t = \tfrac{1}{2}  \nabla \log p_t(\bfX_t) \rmd t$. Note that the family of
densities $\{q_t\}_{t=0}^T$ associated with $(\bfX_t)_{t \in \ccint{0,T}}$ also
satisfies \cref{eq:forward}. Now, we consider
$(\bfhX_t)_{t \in \ccint{0,T}} = (\bfX_{T-t})_{t \in \ccint{0,T}}$ and note that it satisfies
\begin{equation}
  \label{eq:backward_flow_appendix}
 \rmd \bfhX_t = -\tfrac{1}{2}  \nabla \log p_{T-t}(\bfhX_t) \rmd t  .
\end{equation}
Finally, we consider $(\bfY_t^{\mathrm{ODE}})_{t \in \ccint{0,T}}$ which also satisfies
\cref{eq:backward_flow_appendix} and such that the distribution of $\bfY_0^{\mathrm{ODE}}$ is
$\piinv$. Denoting $\{q_t^{\mathrm{ODE}}\}_{t=0}^T$ the densities of
$(\bfY_t^{\mathrm{ODE}})_{t \in \ccint{0,T}}$ w.r.t. $\piinv$ we have for any $t \in \ccint{0,T}$ and $x \in \M$
\begin{equation}
  \label{eq:proba_flow_ode}
 \partial_t q_t^{\mathrm{ODE}}(x) =  \dive(q_t^{\mathrm{ODE}} -\tfrac{1}{2} \nabla\log p_{T-t} )(x)  . 
\end{equation}

\paragraph{SDE model}
When sampling we consider a process $(\bfY^{\mathrm{SDE}}_t)_{t \in \ccint{0,T}}$ such that
$\bfY^{\mathrm{SDE}}_0$ has distribution $\piinv$ and whose family of densities
$\{q_t^{\mathrm{SDE}}\}_{t=0}^T$ satisfies for any $t \in \ccint{0,T}$ and $x \in \M$
\begin{equation}
  \label{eq:proba_flow_sde}
  \partial_t q_t^{\mathrm{SDE}}(x) = -\dive(\log p_{T-t} q_t^{\mathrm{SDE}}(x)) +\tfrac{1}{2}\Delta q_t^{\mathrm{SDE}}(x) = \dive(q_t^{\mathrm{SDE}}\{-\nabla\log p_{T-t} + \tfrac{1}{2}\nabla\log q_t^{\mathrm{SDE}}\})(x)  . 
\end{equation}
Hence, \cref{eq:proba_flow_ode} and \cref{eq:proba_flow_sde} do not agree,
except if $q_t^{\mathrm{SDE}} = q_t^{\mathrm{ODE}} = p_{T-t}$ which is the case if and only if $\bfY^{\mathrm{SDE}}_0$ and
$\bfY_0^{\mathrm{ODE}}$ have the same distribution as $\bfX_T$. Note that it is possible to
evaluate the likelihood of the SDE model using that
\begin{equation}
  \partial_t \log q_t^{\mathrm{SDE}}(\bfY^{\mathrm{SDE}}_t) = \dive(-\nabla\log p_{T-t}(\bfY^{\mathrm{SDE}}_t) +\tfrac{1}{2}\nabla\log q_t^{\mathrm{SDE}}(\bfY^{\mathrm{SDE}}_t)) \rmd t  . 
\end{equation}
We can use the score approximation $\bm{s}_\theta(t,x)$ to approximate
$\nabla \log p_t(x)$ for any $t \in \ccint{0,T}$ and $x \in \M$. In order to
approximate $\nabla \log q_t^{\mathrm{SDE}}$, one can consider another neural network
$\bm{t}_\theta(t,x)$ approximating $\nabla \log q_t^{\mathrm{SDE}}(x)$ for any $t \in \ccint{0,T}$
and $x \in \M$. This approximation can be obtained using the implicit score loss
presented in \Cref{sec:riem-score-appr}.


%%% Local Variables:
%%% mode: latex
%%% TeX-master: "main"
%%% End:


\section{Eigenfunctions, eigenvalues of the Laplace-Beltrami operator}
\label{sec:eigenf-eigenv-lapl}


In this section, we recall the eigenfunctions and eigenvalues of the
Laplace-Beltrami operator in two specific cases: the $d$-dimensional torus and
the $d$-dimensional sphere.

\paragraph{The case of the torus}
Let $\{b_i\}_{i=1}^d$ be a basis of $\rset^d$.  We consider the associated
lattice on $\rset^d$, i.e.
$\Gamma = \ensembleLigne{\sum_{i=1}^d \alpha_i b_i}{\{\alpha_i\}_{i=1}^d \in
  \zset^d}$. Finally, the associated $d$-dimensional torus is defined as
$\tset_\Gamma = \rset^d / \Gamma$. Denote
$\rmB = (b_1, \dots, b_d) \in \rset^{d \times d}$. Let
$\{\bar{b}_i\}_{i=1}^d \in (\rset^d)^d$ such that
$(\rmB^{-1})^\top = (\bar{b}_1, \dots, \bar{b}_d)$. We define
$\Gamma^\star = \ensembleLigne{\sum_{i=1}^d \alpha_i
  \bar{b}_i}{\{\alpha_i\}_{i=1}^d \in \zset^d}$, the dual lattice. Note that for
any $x \in \Gamma$ and $y \in \Gamma^\star$ we have that
$\langle x, y \rangle \in \zset$ and that if $\{b_i\}_{i=1}^d$ is an orthonormal
basis then $\Gamma = \Gamma^\star$. The torus $\rset^d/\Gamma$ is a (flat)
compact Riemannian manifold. The set of eigenvalues of the Laplace-Beltrami
operator is given by
$\ensembleLigne{-4 \uppi^2 \normLigne{y}^2}{y \in \Gamma^\star}$. The
eigenfunctions of the Laplace-Beltrami operator are given by
$\ensembleLigne{x \mapsto \sin(2 \uppi \langle x, y \rangle)}{y \in
  \Gamma^\star}$ and
$\ensembleLigne{x \mapsto \cos(2 \uppi \langle x, y \rangle)}{y \in
  \Gamma^\star}$. 


\paragraph{The case of the sphere} Next, we investigate the case of the
$d$-dimensional sphere \citep[see][]{saloff1994precise}. The set of eigenvalues of
the Laplace-Beltrami operator is given by
$\ensembleLigne{-k(k+d-1)}{k \in \nset}$. Note that $\lambda_k = k(k+d-1)$ has
multiplicity $d_k = (k+d-2)!/\{(d-1)!k\}(2k+d-1)$. The eigenfunctions of the
Laplace-Beltrami operator are known as the spherical harmonics and can be
defined in terms of Legendre polynomials. When investigating the heat kernel on
the $d$-dimensional sphere, we are interested in the product
$(x,y) \mapsto \sum_{\phi \in \Phi_n} \phi(x)\phi(y)$, where $\Phi_n$ is the set
of eigenfunctions associated with the eigenvalue $\lambda_n$ for $n \in
\nset$. This function can be described using the Gegenbauer polynomials
\cite[see][Theorem 2.9]{atkinson2012spherical}. More precisely, we have that for any
$n \in \nset$ and $x,y \in \mathbb{S}^d$
\begin{align}
  G_n(x,y) &= \textstyle{ \sum_{\phi \in \Phi_n} \phi(x) \phi(y)} \\
  &= \textstyle{n! \Gamma((d-1)/2) \sum_{k=0}^{\floor{n/2}} (-1)^k (1- \langle x,y \rangle^2)\langle x,y \rangle^{n-2k} / (4^k k! (n -2k)! \Gamma(k + (d-1)/2) ) ,}
\end{align}
where here $\Gamma: \ \rset_+ \to \rset$ is given for any $v > 0$ by
$\Gamma(v) = \int_0^{+\infty} t^{v-1} \rmd t^{-t} \rmd t$.  In the special case
where $d=1$, then the heat kernel coincide with the wrapped Gaussian density and
can be easily evaluated.

% with $\{\lambda_n\}_n$ and $\{\psi_n\}_n$ respectively the eigenvalues and eigenfunctions of the Laplace-Beltrami operator $\Delta_\mathcal{M}$.
% For instance with $\mathbb{S}^d$, we know \citep{borovitskiy2020Matern,devito2019Reproducing,zhao2018Exact} that $\lambda_n = n(n + d - 1)$ and $$\psi_n(x) \psi_n(y) = \frac{2n+d-1}{d-1} \frac{1}{A_{\mathbb{S}^n}} \mathcal{C}_n^{(d-1)/2}(x \cdot y)$$  where $\mathcal{C}_n^{(d-1)/2}$ are Gegenbauer polynomials.
% An exact sampling scheme exists for $\mathbb{S}^d$ \cite{mijatovic2020note} but it is non trivial to implement \footnote{https://github.com/konkam/ExactWrightFisher.jl}.

% When $d=2$, then the eigenfunctions are the spherical harmonics and the Gegenbauer polynomials are the Legendre polynomials $P_n$, we thus get \citep{jammalamadaka2019Harmonic,mardia2000Directional}: 
% $$p_t(x, y) = \sum^\infty_{n=0} e^{- n(n+1) \cdot t } ~\frac{2n + 1}{4 \pi} P_n(x \cdot y).$$
% When $d=1$, the heat kernel and Wrapped normal density coincide which means one can easily sample $X_t|X_0$.
% Additionally, around $t \approx 0$, \cref{eq:heat_kernel} can be expended as
% $$p_t(x, y) = (4\pi t)^{-d/2} G(r)^{-1/2} \exp \left(-\frac{r^2}{4t}\right) + \mathcal{O}(1)$$
% with $r=d_\mathcal{M}(x,y)$.  Higher order expansions can be obtained
% \cite{rey2019diffusion,zhao2018Exact}.  One could get an unbiased estimator of
% \cref{eq:heat_kernel} via the Russian roulette estimator
% $\sum_n \Delta_n = \mathbb{E}_{N \sim p} \left[ \sum^N_n
%   \frac{\Delta_n}{\mathbb{P}(N \ge n)} \right]$, although what we care in
% practice about $\nabla_x \log p_t(x, y)$ where the $\log$ would bias the
% estimator.



%%% Local Variables:
%%% mode: latex
%%% TeX-master: "main"
%%% End:


\section{Time-reversal formula: extension to compact Riemannian manifolds}
\label{sec:time-reversal}

In this section, we  provide the proof of
\cref{thm:time_reversal_manifold}.  The proof follows the arguments of
\citet[Theorem 4.9]{cattiaux2021time}. We could have also applied the abstract
results of \citet[Theorem 5.7]{cattiaux2021time} to obtain our results. Note that
the time-reversal on manifold could also be obtained by readily extending
arguments from \citet{haussmann1986time}, however the entropic conditions found
by \citet{cattiaux2021time} are more natural when it comes to the study of the
Schr\"odinger Bridge problem. For the interested reader we provide an informal
derivation of the time-reversal formula obtained by \citet{haussmann1986time} in
\cref{sec:informal-derivation}. The proof of \cref{thm:time_reversal_manifold}
is given in \cref{sec:proof-crefthm:t}. Finally, we emphasize that
\citet{garcia2021brenier} develops a Girsanov theory for stochastic processes
defined on compact manifolds with boundary in order to study the
Brenier-Schr\"odinger problem.

\subsection{Informal derivation}
\label{sec:informal-derivation}

In this section, we provide a non-rigorous derivation of
\cref{thm:time_reversal_manifold} following the approach of
\citet{haussmann1986time}. Let $(\bfX_t)_{t \in \ccint{0,T}}$ be a continuous
process such that for any $f \in \rmc^2(\M)$ we have that
$(\bfM_t^{\bfX, f})_{t \in \ccint{0,T}}$ is a $\bfX$-martingale where for any
$t \in \ccint{0,T}$
  \begin{equation}
    \label{eq:martingale_forward}
    \textstyle{ \bfM_t^{\bfX, f} = f(\bfX_t) - \int_0^t \{ \langle b(\bfX_s), \nabla f(\bfX_s) \rangle_\M + (1/2) \Delta f(\bfX_s) \} \rmd s  . }
  \end{equation}
  Let $(\bfY_t)_{t \in \ccint{0,T}} = (\bfX_{T-t})_{t \in \ccint{0,T}}$. Our goal is to show that for any $f \in \rmc^2(\M)$, 
  $(\bfM_t^{\bfY, f})_{t \in \ccint{0,T}}$ is a $\bfY$-martingale where for any
  $t \in \ccint{0,T}$
  \begin{equation}
    \textstyle{ \bfM_t^{\bfY, f} = f(\bfY_t) - \int_0^t \{ \langle b(\bfY_s) + \nabla \log p_{T-s}(\bfY_s), \nabla f(\bfY_s) \rangle_\M + (1/2) \Delta f(\bfY_s) \} \rmd s  . }
  \end{equation}
  Note that here we implicitly assume that for any $t \in \ccint{0,T}$, $\bfX_t$
  admits a smooth positive density w.r.t. $\piinv$ denoted $p_t$.  In other
  words, we want to show that for any $g \in \rmc^2(\M)$ and
  $s, t \in \ccint{0,T}$ with $t \geq s$ we have
  \begin{equation}
    \label{eq:time_reversal_manifold_haussman}    
    \textstyle{\expeLigne{g(\bfY_s)(f(\bfY_t) - f(\bfY_s))} = \expeLigne{g(\bfY_s)\int_s^t \{ \langle b(\bfY_u) + \nabla \log p_{T-u}(\bfY_u), \nabla f(\bfY_u) \rangle_\M + (1/2) \Delta f(\bfY_u) \} \rmd u}  . }
  \end{equation}
  We introduce the infinitesimal generator
  $\generator: \  \rmc^2(\M) \to \rmc(\M)$ given for any $f \in \rmc^2(\M)$ and $x \in \M$ by
  \begin{equation}
    \generator (f)(x) = \langle b(x) , \nabla f(x) \rangle_\M + (1/2) \Delta f(x)  . 
  \end{equation}
  Similarly, we introduce the infinitesimal generator
  $\generatort: \  \ccint{0,T} \times \rmc^2(\M) \to \rmc(\M)$ given for any $f \in \rmc^2(\M)$, $t \in \ccint{0,T}$ and $x \in \M$ by
  \begin{equation}
    \generatort (t, f)(x) = \langle b(x) + \nabla \log p_{T-t}(x), \nabla f(x) \rangle_\M + (1/2) \Delta f(x)  . 
  \end{equation}
  With these notations, \eqref{eq:time_reversal_manifold_haussman} can be written as follows:  we want to show that for any $g \in \rmc^2(\M)$ and
  $s, t \in \ccint{0,T}$ with $t \geq s$ we have 
  \begin{equation}
    \label{eq:time_reversal_manifold_haussman}    
    \textstyle{\expeLigne{g(\bfY_s)(f(\bfY_t) - f(\bfY_s))} = \expeLigne{g(\bfY_s)\int_s^t \generatort(u, \bfY_u) \rmd u}  . }
  \end{equation}
  The rest of this section follows the first part of the proof of \citet[Theorem 2.1]{haussmann1986time}.
  Let $t, s \in \ccint{0,T}$ with $t \geq s$. We have
  \begin{align}
    \textstyle{\expeLigne{g(\bfY_s)(f(\bfY_t) - f(\bfY_s))}} &= \textstyle{\expeLigne{g(\bfX_{T-s})(f(\bfX_{T-t}) - f(\bfX_{T-t}))}} \\
                                                             &= \textstyle{\expeLigne{\CPELigne{g(\bfX_{T-s})}{\bfX_{T-t}}f(\bfX_{T-t})} - \expeLigne{g(\bfX_{T-s})f(\bfX_{T-s})}} \\
                                                             &= \textstyle{\expeLigne{v(T-t,\bfX_{T-t})f(\bfX_{T-t})} - \expeLigne{v(T-s,\bfX_{T-s})f(\bfX_{T-s})}}  ,
                                                               \label{eq:first_der}
  \end{align}
  with $v: \ \ccint{0,T-s} \times \M \to \rset$ given for any $u \in \ccint{0,T-s}$
  and $x \in \M$ by $v(u,x) = \CPELigne{g(\bfX_{T-s})}{\bfX_u=x}$. We have that $v$
  satisfies the backward Kolmogorov equation, i.e. we have for any
  $u \in \ccint{0,T-s}$ and $x \in \M$
  \begin{equation}
    \label{eq:backward_kolmogorov}
    \partial_u v(u,x) = -\generator v(u,x) . 
  \end{equation}
  Note that it is not trivial to show that $v$ is regular enough to satisfy the
  backward Kolmogorov equation. In this informal derivation, we assume that $v$
  is regular enough and will provide a different rigorous proof of the
  time-reversal formula in \cref{sec:proof-crefthm:t}. However, note that it is
  possible to show that $v$ indeed satisfies the backward Kolmogorov equation by
  adapting arguments from \citet{haussmann1986time} to the manifold framework.

  Let $h: \ \ccint{0,T-s} \times \M \to \rset$ given for any
  $u \in \ccint{0,T-s}$ and $x \in \M$ by $h(u,x) = v(u,x) f(x)$. Using
  \eqref{eq:backward_kolmogorov}, we have for any $u \in \ccint{0,T-s}$ and
  $x \in \M$
  \begin{align}
    \label{eq:def_h}
    \partial_u h(u,x) + \generator h(u, x) &= f(x) \partial_u v(u,x)  + f(x) \generator v(u,x) + v(u,x) \generator f(x) +  \langle \nabla f(x), \nabla v(u,x)\rangle \\
    &=  v(u,x) \generator f(x) + \langle \nabla f(x), \nabla v(u,x)\rangle_\M  . 
  \end{align}
  In addition, using the
      divergence theorem \citep[see][p.51]{lee2018introduction}, we have for any $u \in \ccint{0,T-s}$
  \begin{align}
    &\expeLigne{\langle \nabla f(\bfX_u), \nabla v(u,\bfX_u)\rangle_\M} = \textstyle{\int_{\M} \langle \nabla f(x_u), \nabla v(u,x_u) p_u(x_u) \rangle_\M \rmd \piinv(x_u) } \\
                                                                    & \qquad \qquad = - \textstyle{\int_{\M} v(u,x_u) \dive(p_u \nabla f) (x_u) \rmd \piinv(x_u) } \\
                                                                    & \qquad \qquad = - \textstyle{\int_{\M} v(u,x_u) \Delta f(x_u) p_u(x_u) \rmd \piinv(x_u) - \int_{\M} v(u,x_u) \langle \nabla f(x_u), \nabla \log p_u(x_u) \rangle_\M p_u(x_u) \rmd \piinv(x_u) } \\
                                                                    & \qquad \qquad = - \textstyle{\expeLigne{ v(u,\bfX_u) \Delta f(\bfX_u)}  - \expeLigne{ v(u,\bfX_u) \langle \nabla f(\bfX_u), \nabla \log p_u(\bfX_u) \rangle_\M} }  .
  \end{align}
  Therefore, using this result and \eqref{eq:def_h} we get that for any
  $u \in \ccint{0,T-s}$
  \begin{align}
    \expeLigne{\partial_u h(u,\bfX_u) + \generator h(u, \bfX_u)} &= \expeLigne{v(u,\bfX_u)\{ \langle b(\bfX_u) - \nabla \log p_u(\bfX_u), \nabla f(\bfX_u) \rangle_\M -(1/2) \Delta f(\bfX_u)\}} \\
    &= -\expeLigne{v(u,\bfX_u)\generatort(T-u,f)(\bfX_u)}  . 
  \end{align}
  Combining this result and \eqref{eq:martingale_forward} and that for any
  $u \in \ccint{0,T-s}$ and $x \in \M$,
  $v(u,x) = \CPELigne{g(\bfX_{T-s})}{\bfX_u=x}$ we get
  \begin{align}
    \expeLigne{v(T-t,\bfX_{T-t})f(\bfX_{T-t})} - \expeLigne{v(T-s,\bfX_{T-s})f(\bfX_{T-s})} &= \expeLigne{h(T-t, \bfX_{T-t}) - h(T-s, \bfX_{T-s})} \\
                                                                                            &= \textstyle{\int_{T-t}^{T-s} \expeLigne{v(u,\bfX_u)\generatort(T-u, \bfX_u)} \rmd u } \\
    &= \textstyle{\expeLigne{g(\bfX_{T-s})\int_{T-t}^{T-s} \generatort(T-u, \bfX_u) \rmd u } .}\\
  \end{align}
  Using this result, \eqref{eq:first_der} and the change of variable $u \mapsto T-u$ we obtain 
  \begin{equation}
    \expeLigne{g(\bfY_s)(f(\bfY_t) - f(\bfY_s))} = \textstyle{\expeLigne{g(\bfX_{T-s})\int_{T-t}^{T-s} \generatort(u, \bfX_u) \rmd u } } = \textstyle{\expeLigne{g(\bfY_{s})\int_{s}^{t} \generatort(u, \bfY_u) \rmd u } }  .
  \end{equation}
  Hence, \eqref{eq:time_reversal_manifold_haussman} holds and we have proved
  \cref{thm:time_reversal_manifold}. Again, we emphasize that in order to make
  the proof completely rigourous one needs to derive regularity properties of $v$.

  
\subsection{Proof of \cref{thm:time_reversal_manifold}}
\label{sec:proof-crefthm:t}

In this section, we follow another approach to prove the time-reversal
formula. We are going to use the integration by part formula of \citet[Theorem
3.17]{cattiaux2021time} in a similar spirit as \citet[Theorem
4.9]{cattiaux2021time} in the Euclidean setting. In order to adapt arguments
from \citet{cattiaux2021time} to our Riemannian setting, we use the Nash
embedding theorem in order to embed our processes in a Euclidean space and
leverage tools from Girsanov theory. The rest of the section is organized as
follows. First in \cref{sec:diff-proc-stoch}, we recall basic properties of
infinitesimal generators and recall the integration by part formula of
\citet[Theorem 3.17]{cattiaux2021time}. Then in \cref{sec:girs-theory-comp}, we
extend some Girsanov theory to compact Riemannian manifolds using the Nash
embedding theorem. We conclude the proof in \cref{sec:concluding-proof}.

\subsubsection{Diffusion processes and integration by part formula}
\label{sec:diff-proc-stoch}

In this section, we state a simplified version of \citet[Theorem
3.17]{cattiaux2021time} for Markov continuous path (probability) measure on
Polish spaces. Let $(\msx, \mcx)$ be a Polish space. We say that $\Pbb$ is a
path measure if $\Pbb \in \Pens(\rmc(\ccint{0,T}, \msx))$. Let
$(\bfX_t)_{t \in \ccint{0,T}}$ with distribution $\Pbb$. We denote
$(\mcf_t)_{t \in \ccint{0,T}}$ the filtration such that for any
$t \in \ccint{0,T}$, $\mcf_t = \sigma(\bfX_s, \ s \in \ccint{0,t})$. Let
$(\bfM_t)_{t \in \ccint{0,T}}$ be a Polish-valued stochastic process. We say that
$(\bfM_t)_{t \in \ccint{0,T}}$ is a $\Pbb$-local martingale if it is a local
martingale w.r.t. the filtration $(\mcf_t)_{t \in \ccint{0,T}}$. A function
$u: \ \ccint{0,T} \times \msx \to \rset$ is said to be in the domain of the
extended generator of $\Pbb$ if there exists a process
$(\generatorb_\Pbb u(t, \bfX_{\ccint{0,t}}))_{t \in \ccint{0,T}}$ such that:
\begin{enumerate}[label= (\alph*),  wide, labelwidth=!, labelindent=0pt]
\item $(\generatorb_\Pbb u(t, \bfX_{\ccint{0,t}}))_{t \in \ccint{0,T}}$ is adapted w.r.t. $(\mcf_t)_{t \in \ccint{0,T}}$.
\item $\int_0^T \absLigne{\generatorb_\Pbb u(t, \bfX_{\ccint{0,t}})} \rmd t < +\infty$, $\Pbb$-a.s.
\item The process $(\bfM_t)_{t \in \ccint{0,T}}$ is a $\Pbb$-local martingale,
  where for any $t \in \ccint{0,T}$
  \begin{equation}
    \textstyle{\bfM_t = u(t,\bfX_t) - u(0, \bfX_0) - \int_0^t \generatorb_\Pbb u(s, \bfX_{\ccint{0,s}}) \rmd s   .}
  \end{equation}
\end{enumerate}
The domain of the extended generator is denoted $\dom(\generatorb_\Pbb)$. We say
that $(u,v)$ with $u,v : \ \ccint{0,T} \times \msx \to \rset$ is in the domain
of the carr\'e du champ if $u,v, uv \in \dom(\generatorb_\Pbb)$. In this case, we
define the carr\'e du champ $\carrechampb_\Pbb$ as
\begin{equation}
  \carrechampb_\Pbb(u,v) = \generatorb_\Pbb(uv) - \generatorb_\Pbb(u)v - \generatorb_\Pbb(v)u  . 
\end{equation}
Note that if $\msx = \M$ is a Riemannian manifold,
$\rmc^2(\M) \subset \dom(\generatorb_\Pbb)$ and for any $u \in \rmc^2(\M)$
$\generatorb_\Pbb(u) = \langle \nabla u, X \rangle_\M + (1/2)\Delta u$ with
$X \in \Gamma(\TM)$  then we have that $\rmc^2(\M) \times \rmc^2(\M) \subset \dom(\carrechampb_\Pbb)$
and for any $u, v \in \rmc^2(\M)$,
$\carrechampb_\Pbb(u,v) = \langle \nabla u, \nabla v \rangle_\M$. Assume that there exists
$\mathcal{U}_\Pbb \subset \dom(\generatorb_\Pbb) \cap \rmc_b(\msx)$ such that
$\mathcal{U}_\Pbb$ is an algebra. We denote $\mathcal{U}_{\Pbb,2}$ such that
\begin{equation}
  \mathcal{U}_{\Pbb,2} = \ensembleLigne{u \in \mathcal{U}_\Pbb}{\generatorb_\Pbb u \in \mathrm{L}^2(\Pbb), \ \carrechampb_\Pbb(u,u) \in \mathrm{L}^1(\Pbb)}  . 
\end{equation}
Finally we denote $R(\Pbb)$ the time-reverse path measure, i.e. for any
$\msa \in \mcb{\rmc(\ccint{0,T}, \msx)}$ we have
$R(\Pbb)(\msa) = \Pbb(R(\msa))$, where
$R(\msa) = \ensembleLigne{t \mapsto \omega_{T-t}}{\omega \in \msa}$.  In what
follows, we assume $\Pbb$ is Markov. It is well-known, see \citep[Theorem
1.2]{leonard2014reciprocal} for instance, that in this case $R(\Pbb)$ is also
Markov. In addition, since $\Pbb$ is Markov, for any $u \in \mathrm{dom}(\generatorb_\Pbb)$ and
$t \in \ccint{0,T}$ there exists $\generator_\Pbb$ such that
$\generatorb_\Pbb u(t, \bfX_{\ccint{0,t}}) = \generator_\Pbb u(t, \bfX_t)$ with
$\generator_\Pbb u: \ \ccint{0,T} \times \msx \to \rset$. Similarly, we define
$\carrechamp_\Pbb(u,v): \ \ccint{0,T} \times \msx \to \rset$ from $\carrechampb_\Pbb(u,v)$.

We are now ready to state the integration by part formula,
\citep[Theorem 3.17]{cattiaux2021time}. 

\begin{theorem}
  \label{thm:ibp_cattiaux}
  Let $u, v \in \mathcal{U}_{\Pbb, 2}$. The following hold:
  \begin{enumerate}[label= (\alph*),  wide, labelwidth=!, labelindent=0pt]    
  \item If
  $u \in \dom(\generator_{R(\Pbb)})$ and
  $\generator_{R(\Pbb)}u \in \mathrm{L}^1(\Pbb)$ then for almost any $t \in \ccint{0,T}$
  \begin{equation}
    \expeLigne{\{\generator_\Pbb u(t, \bfX_t) + \generator_{R(\Pbb)} u (T-t, \bfX_t)\}v(\bfX_t) + \carrechamp_\Pbb(u,v)(t, \bfX_t)} = 0  .       
  \end{equation}  
\item If the following hold:
  \begin{enumerate}[label=\roman*)]
  \item $\carrechamp_\Pbb(u,v) \in \rmc(\ccint{0,T} \times \msx, \rset)$.
  \item $\mathcal{U}_{2, \Pbb}$ determines the weak convergence of Borel measure.
  \item $\mu$ defines a finite measure on $\ccint{0,T} \times \msx$ where for
    any $\omega \in \bar{\mathcal{U}}_{2, \Pbb}$ we have
    \begin{equation}
      \textstyle{\mu[\omega] = \expeLigne{\int_0^T \carrechamp_\Pbb(u,\omega_t)(t, \bfX_t) \rmd t  ,}}
    \end{equation}
    where
    $\bar{\mathcal{U}}_{2, \Pbb} = \ensembleLigne{\omega \in \rmc(\ccint{0,T}
      \times \msx, \rset)}{\omega(t, \cdot) \in \mathcal{U}_{2, \Pbb}\ \
      \text{for any $t \in \ccint{0,T}$}}$.
  \end{enumerate}
  Then $u \in \dom(\generator_{R(\Pbb)})$ and
  $\generator_{R(\Pbb)}u \in \mathrm{L}^1(\Pbb)$.
  \end{enumerate}
\end{theorem}

Note that this theorem is a simplified version of \citet[Theorem
3.17]{cattiaux2021time} where we restrict ourselves to the case of Markov path
measures. In what follows, we wish to apply \cref{thm:ibp_cattiaux} to diffusion
processes on manifolds. To do so, we will verify that under a finite entropy
assumption, the conditions $u \in \dom(\generator_{R(\Pbb)})$ and
$\generator_{R(\Pbb)}u \in \mathrm{L}^1(\Pbb)$ are fullfilled for a class of
regular functions $u$. These integrability results are obtained using Girsanov
theory.

\subsubsection{Girsanov theory on compact Riemannian manifolds}
\label{sec:girs-theory-comp}

In this section, we will consider two types of martingale problems: one on
Euclidean spaces and one on the compact Riemannian manifold $\M$. Let
$\Pbb \in \Pens(\rmc(\ccint{0,T}, \rset^p))$. We say that $\Pbb$ satisfies the
(Euclidean) martingale problem with infinitesimal generator
$\generator: \ \ccint{0,T} \times \rmc^2(\rset^p) \times \rset^p \to \rset$ if
for any $u \in \rmc_c^2(\rset^p)$, $(\bfM_t)_{t \in \ccint{0,T}}$ is a
$\Pbb$-martingale where for any $t \in \ccint{0,T}$ we have
\begin{equation}
  \textstyle{
    \bfM_t = \bfM_0 + \int_0^t \generator(t, u)(\bfX_s) \rmd s  ,
    }
  \end{equation}
  where $(\bfX_t)_{t \in \ccint{0,T}}$ has distribution $\Pbb$ and
  $\int_0^T \absLigne{\generator(t, u)(\bfX_s) \rmd t} <+\infty$, $\Pbb$-a.s.
  Let $\Pbb \in \Pens(\rmc(\ccint{0,T}, \M))$. We say that $\Pbb$ satisfies the
  (Riemannian) martingale problem with infinitesimal generator
  $\generatort: \ \ccint{0,T} \times \rmc^2(\M) \times \M \to \rset$ if for any
  $u \in \rmc^2(\M)$, $(\bfM_t)_{t \in \ccint{0,T}}$ is a $\Pbb$-martingale
  where for any $t \in \ccint{0,T}$ we have
\begin{equation}
  \textstyle{
    \bfM_t = \bfM_0 + \int_0^t \generatort(t, u)(\bfX_s) \rmd s  ,
    }
  \end{equation}
  where $(\bfX_t)_{t \in \ccint{0,T}}$ has distribution $\Pbb$ and 
  $\int_0^T \absLigne{\generatort(t, u)(\bfX_s) \rmd t} <+\infty$, $\Pbb$-a.s.
  We now prove the following theorem.

  \begin{proposition}
    \label{prop:girsanov_manifold}
    Let $\Qbb$ be the path measure of a Brownian motion on $\M$. Let $\Pbb$ be a
    Markov path measure on $\rmc(\ccint{0,T}, \M)$ such that $\KL{\Pbb}{\Qbb} < +\infty$. Then there exists
    $\beta$ such that for any $t \in \ccint{0,T}$ and
    $x \in \M$, $\beta(t,x) \in \mathrm{T}_x \M$. In addition, we have that
    $\Pbb$ satisfies the martingale problem with infinitesimal generator
    $\generator$ where for any $t \in \ccint{0,T}$, $u \in \rmc^2(\M)$ and
    $x \in \M$ we have
    \begin{equation}
      \generator(t,u)(x) = \langle \beta(t,x), \nabla u(x) \rangle_\M + (1/2) \Delta u(x)  . 
    \end{equation}
    In addition, we have that
    \begin{equation}
      \textstyle{\KL{\Pbb}{\Qbb} = \KL{\Pbb_0}{\Qbb_0} + (1/2) \int_0^T \expeLigne{\norm{\beta(t, \bfX_t)}^2} \rmd t  ,}
    \end{equation}
    where $(\bfX_t)_{t \in \ccint{0,T}}$ has distribution $\Pbb$.
  \end{proposition}


  \begin{proof}
    First, we extend $(\bfB_t^\M)_{t \in \ccint{0,T}}$ to $\rset^p$ using the
    Nash embedding theorem \citep[see][]{gunther1991isometric}.
    $(\bfB_t^\M)_{t \in \ccint{0,T}}$ can be seen as a process on $\rset^p$ (for
    some $p \in \nset$) which satisfies in a weak sense
    \begin{equation}
      \textstyle{
        \rmd \bfB_t^\M = \sum_{i=1}^p P_i(\bfB_t^\M) \circ \rmd \bfB_t^i  = P(\bfB_t^\M) \circ \rmd \bfB_t  ,
        }
    \end{equation}
    where $(\bfB_t)_{t \in \ccint{0,T}}$ is a $p$-dimensional Brownian motion
    and $P \in \rmc^\infty(\rset^p, \rset^{p\times p})$ is such that for any
    $x \in \M$, $P(x)$ is the projection onto $\mathrm{T}_x \M$ and for any
    $i \in \{1, \dots, p\}$, $P_i \in \rmc^\infty(\rset^p, \rset^p)$ with
    $P_i = P e_i$ where $\{e_j\}_{j=1}^d$ is the canonical basis of $\rset^p$.
    Using the link between Stratanovitch and It\^o integral, there exists
    $\bar{b} \in \rmc^\infty(\rset^p, \rset^p)$ such that
    $(\bfB_t^\M)_{t \in \ccint{0,T}}$ can be seen as a process on $\rset^p$
    which satisfies in a weak sense
    \begin{equation}
      \textstyle{
        \rmd \bfB_t^\M = \bar{b}(\bfB_t^\M) \rmd t +  P(\bfB_t^\M)  \rmd \bfB_t  .
        }
      \end{equation}
      For any $u \in \rmc^2(\M)$, we consider $\bar{u}$ an extension to $\rmc^2_c(\rset^p)$ and we have for any $s, t \in \ccint{0,T}$
      \begin{align}
        &\textstyle{\expeLigne{\bar{v}(\bfB_s^\M) \int_s^t (1/2) \Delta u(\bfB_u^\M) \rmd u}} \\
        & \qquad =  \textstyle{\expeLigne{\bar{v}(\bfB_s^\M) \int_s^t \{ \langle \nabla \bar{u}(\bfB_u^\M), \bar{b}(\bfB_u^\M) \rangle + (1/2) \langle P(\bfB_u^\M), \nabla^2 \bar{u}(\bfB_u^\M) \rangle \} \rmd u}  . }
      \end{align}
      In particular, we get that for any $x \in \M$,
      $\Delta u(x) = 2 \langle \bar{u}(x), \bar{b}(x) \rangle + \Delta
      \bar{u}(x)$ \valentin{prove that for the projection this is okay}. Note
      that $(\bfB_t^\M)_{t \in \ccint{0,T}}$ (seen as a process on $\rset^p$)
      satisfies the condition $\mathrm{(U)}$ in
      \cite{leonard2012girsanov}. Therefore applying \cite[Theorem
      2.1]{leonard2012girsanov}, \citep[Claim 4.5]{cattiaux2021time}, there
      exists $\bar{\beta}: \ \ccint{0,T} \times \rset^p \to \rset^p$ such that
      \begin{equation}
        \label{eq:KL_ineq}
      \textstyle{\KL{\Pbb}{\Qbb} = \KL{\Pbb_0}{\Qbb_0} + (1/2) \int_0^T \expeLigne{\normLigne{P(\bfX_t) \bar{\beta}(t, \bfX_t)}^2} \rmd t  .}
    \end{equation}
    In addition, $\Pbb$ (seen as a process on $\rset^p$) satisfies a martingale
    problem with infinitesimal generator
    $\generatorb: \ \ccint{0,T} \times \rmc^2_c(\rset^p) \times \rset^p \to \rset$ such that
    for any $t \in \ccint{0,T}$, $\bar{u} \in \rmc^2(\rset^p)$ and $x \in \rset^p$
    \begin{equation}
      \generatorb(t,\bar{u})(x) = \langle \bar{b}(x) + P(x)\bar{\beta}(t,x), \nabla \bar{u}(x) \rangle + (1/2) \Delta \bar{u}(x)  . 
    \end{equation}
    Let $\beta: \ \ccint{0,T} \times \M$ such that for any $t \in \ccint{0,T}$
    and $x \in \M$ we have $\beta(t,x) = P(x) \bar{\beta}(t,x)$. In particular,
    we have that for any $x \in \M$, $\beta(t,x) \in \mathrm{T}_x\M$. Let
    $u \in \rmc^2(\M)$ \valentin{dire que c'est okay pour le delta et pour le
      gradient si on prend u bar = u circ p} and consider an extension $\bar{u}$
    to $\rmc^2(\rset^p)$. For any $t \in \ccint{0,T}$ and $x \in \M$ we have
    \begin{align}
      \generatorb(t,\bar{u})(x) &= \langle \bar{b}(x) + P(x)\bar{\beta}(t,x), \nabla \bar{u}(x) \rangle + (1/2) \Delta \bar{u}(x) \\
                               &= \langle  \beta(t,x), \nabla \bar{u}(x) \rangle + (1/2) \Delta u(x) \\
                               &= \langle P(x) \beta(t,x), P(x) \nabla \bar{u}(x) \rangle + (1/2) \Delta u(x) = \langle \beta(t,x), \nabla u(x) \rangle + (1/2) \Delta u(x)  . 
    \end{align}
    In particular, we have that $\Pbb$ (seen as a process on $\M$) satisfies a
    martingale with infinitesimal generator
    $\generatorb: \ \ccint{0,T} \times \rmc^2_c(\M) \times \M \to \rset$ such that
    for any $t \in \ccint{0,T}$, $u \in \rmc^2(\rset^p)$ and $x \in \M$
    \begin{equation}
      \generator(t,\bar{u})(x) = \langle \beta(t,x), \nabla u(x) \rangle_\M + (1/2) \Delta u(x)  . 
    \end{equation}
    In addition, rewriting \eqref{eq:KL_ineq} we have
      \begin{equation}
        \label{eq:KL_ineq}
      \textstyle{\KL{\Pbb}{\Qbb} = \KL{\Pbb_0}{\Qbb_0} + (1/2) \int_0^T \expeLigne{\normLigne{\beta(t, \bfX_t)}^2} \rmd t  ,}
    \end{equation}
    which concludes the proof.
  \end{proof}
  
  Once this proposition is established, we can obtain the following
  straightforward extension of \citet[Proposition 4.6]{cattiaux2021time}.

  \begin{proposition}
    \label{prop:hyp_317}
    Let $\Qbb$ be a Brownian motion with $\Qbb_0 = \piinv$ and $\Pbb$ a path
    measure on $\rmc(\ccint{0,T}, \M)$ such that $\KL{\Pbb}{\Qbb} <
    +\infty$. Then, there exist $\beta_\Pbb, \beta_{R(\Pbb)}: \ \ccint{0,T} \times \M \to $
    such that for any $t \in \ccint{0,T}$ and $x \in \M$,
    $\beta_\Pbb(t,x), \beta_{R(\Pbb)}(t,x) \in \mathrm{T}_x \M$. In addition, we have that
    $\Pbb$ and $R(\Pbb)$ satisfy martingale problems with infinitesimal generator
    $\generator_{\Pbb}$, respectively $\generator_{R(\Pbb)}$ where for any $t \in \ccint{0,T}$, $u \in \rmc^2(\M)$ and
    $x \in \M$ we have
    \begin{align}
      &\generator_{\Pbb}(t,u)(x) = \langle \beta_\Pbb(t,x), \nabla u(x) \rangle_\M + (1/2) \Delta u(x)  , \\
      &\generator_{R(\Pbb)}(t,u)(x) = \langle \beta_{R(\Pbb)}(t,x), \nabla u(x) \rangle_\M + (1/2) \Delta u(x)  . 
    \end{align}
    Finally, we have that
    \begin{equation}
      \textstyle{
        \int_0^T \expeLigne{\norm{\beta_\Pbb(t, \bfX_t)}^2} \rmd t + \int_0^T \expeLigne{\norm{\beta_{R(\Pbb)}(t, \bfX_{T-t})}^2} \rmd t < +\infty  ,
        }
    \end{equation}
    where $(\bfX_t)_{t \in \ccint{0,T}}$ has distribution $\Pbb$.
  \end{proposition}

  \begin{proof}
    The proof is straightforward upon combining \cref{prop:girsanov_manifold}
    and the fact that
    $\KL{\Pbb}{\Qbb} = \KL{R(\Pbb)}{R(\Qbb)} = \KL{R(\Pbb)}{\Qbb} < +\infty$,
    using that $\Qbb$ is stationary.
  \end{proof}

  We conclude this section, with the following application of \cref{thm:ibp_cattiaux}.

  \begin{proposition}
    \label{prop:cattiaux_spec}
    For any $u, v \in \rmc^\infty(\M)$, we have that for almost any $t \in \ccint{0,T}$
    \begin{equation}
      \label{eq:equalitu}
      \expeLigne{v(\bfX_t) \langle \beta_\Pbb(t, \bfX_t) + \beta_{R(\Pbb)}(T-t, \bfX_t), \nabla u(\bfX_t) \rangle_\M + \langle \nabla u(\bfX_t), \nabla v(\bfX_t) \rangle} = 0  . 
    \end{equation}
  \end{proposition}

  \begin{proof}
  Remark that $\rmc^2(\M) \subset \dom(\carrechamp_\Pbb)$ and
  $\rmc^2(\M) \subset \dom(\carrechamp_{R(\Pbb)})$. In addition, we have that for any
  $u,v \in \rmc^2(\M)$,
  $\carrechamp_\Pbb(u,v) = \carrechamp_{R(\Pbb)}(u,v) = \langle u, v \rangle$. Note that
  by \cref{prop:hyp_317} and \cref{thm:ibp_cattiaux} we immediately have that
  for any $u, v \in \rmc^\infty(\M)$, \eqref{eq:equalitu} holds.    
  \end{proof}
\subsubsection{Concluding the proof}
\label{sec:concluding-proof}

Using \cref{prop:cattiaux_spec} we can now conclude the proof of \cref{thm:time_reversal_manifold}.
First, remark that we can identify $\beta_\Pbb = b$. Let $u, v \in \rmc^\infty(\M)$, we have that 
    \begin{equation}
      \label{eq:equality_fin}
      \expeLigne{v(\bfX_t) \langle b(\bfX_t) + \beta_{R(\Pbb)}(T-t, \bfX_t), \nabla u(\bfX_t) \rangle + \Delta u(\bfX_t) v(\bfX_t)+ \langle \nabla u(\bfX_t), \nabla v(\bfX_t) \rangle} = 0  . 
    \end{equation}
    Using that for any $t \in \ccint{0,T}$, $\Pbb_t$ admits a smooth positive
    density w.r.t. $\piinv$ denoted $p_t$ and the divergence theorem, see
    \citep[p.51]{lee2018introduction}, we have that for any $t \in \ccint{0,T}$,
\begin{align}
  &    \textstyle{\int_{\M} \{ \langle \beta_{R(\Pbb)}(T-t, x), \nabla u(x) \rangle + \langle b(x), \nabla u(x) \rangle \} v(x) p_t(x) \rmd \piinv(x)} \\
    & \qquad \qquad \qquad \qquad = \textstyle{\int_\M \langle \nabla u(x) p_t(x), \nabla v(x) \rangle \rmd \piinv(x) } \\
   & \qquad \qquad \qquad \qquad = - \textstyle{\int_\M \{ \Delta u (x) + \langle \nabla \log p_t(x), \nabla u(x) \rangle \} v(x) p_t(x)\rmd \piinv(x) }  . 
\end{align}
Therefore, we get that for any $t \in \ccint{0,T}$ and $x \in \M$,
$\langle \beta_{R(\Pbb)}(T-t, x), \nabla u(x) \rangle = \langle 
-b(x) + \nabla\log p_t(x), \nabla u(x) \rangle$, which concludes the proof.

    
%%% Local Variables:
%%% mode: latex
%%% TeX-master: "main"
%%% End:


\section{Schr\"odinger Bridges on Manifolds}
\label{sec:extension}

%\valentin{no compelling examples for conditional sampling... Maybe in physics?}

For Euclidean SGMs, the generative model is given by an approximation
of the time-reversal of the noising dynamics $(\bfX_t)_{t \in \ccint{0,T}}$ while the backward dynamics
$(\bfY_t)_{t \in \ccint{0,T}}$ is initialized with the invariant distribution of
the noising dynamics (the uniform distribution $\piinv$ in case of
RSGM). However, in order for the method to yield good results we need
$\mathcal{L}(\bfY_0) \approx \mathcal{L}(\bfX_T)$ \cite[see][Theorem
1]{debortoli2021neurips}. Usually, this requires the number of steps in the
backward process to be large in order to keep $T$ large and $\gamma$ small
(where $\gamma > 0$ is the stepsize in the Geodesic Random Walk). Another
limitation of SGMs is that existing methods target
an easy-to-sample reference distribution. Hence, classical SGMs
cannot interpolate between two distributions defined by datasets. To
circumvent this problem, one can consider a process whose initial and terminal
distribution are pinned down using Schr\"odinger bridges
\citep{schrodinger1932theorie,leonard2012schrodinger,chen2016entropic,debortoli2021neurips}.

\paragraph{Dynamical Schr\"odinger bridges}
We briefly recall the notion of dynamical Schr\"odinger bridge
\citep{leonard2012schrodinger,chen2016entropic,vargas2021solving,debortoli2021neurips,chen2021likelihood}. We
consider a reference path probability measure
$\Pbb \in \Pens(\rmc(\ccint{0,T}, \M))$. In practice, we set $\Pbb$ to be the
distribution of the Brownian motion $(\bfB_t^\M)_{t \in \ccint{0,T}}$ such that
$\bfB_0^\M$ has distribution $\pizero$, the target data distribution. Then, we consider
the \emph{dynamical Schr\"odinger bridge problem}
\begin{equation}
  \Qbb^\star = \argmin \ensembleLigne{\KL{\Qbb}{\Pbb}}{\Qbb \in \Pens(\ccint{0,T}, \M), \ \Qbb_0 = \pizero, \ \Qbb_T = \piinv} . 
\end{equation}
The solution $\Qbb^\star$ is called the Schr\"odinger Bridge (SB).  Note that if
$\Qbb^\star$ is associated with a backward process
$(\bfY_t^\star)_{t \in \ccint{0,T}}$, then we can obtain a generative model as
follows. First sample from $\piinv = \mathcal{L}(\bfY^\star_T)$ and then follow
the (backward) dynamics of $(\bfY^\star_t)_{t \in \ccint{0,T}}$. By definition, we obtain
that $\mathcal{L}(\bfY^\star_0) = \pizero$, the target distribution.

In practice however, the solution of the SB problem is approximated using the
Iterative Proportional Fitting (IPF) algorithm. Note that in discrete space the
IPF is also known as the Sinkhorn algorithm \citep{sinkhorn1967diagonal,peyre2019computational}. The
IPF defines a sequence of path probability measures
$(\Qbb^n)_{n \in \nset} \in (\Pens(\rmc(\ccint{0,T}, \M)))^\nset$, such that
$\Qbb^0 = \Pbb$ and for any $n \in \nset$
\begin{align}
  &\Qbb^{2n+1} = \argmin \ensembleLigne{\KL{\Qbb}{\Qbb^{2n}}}{\Qbb \in \Pens(\rmc(\ccint{0,T}, \M)), \Qbb_T = \piinv}  , \\
  &\Qbb^{2n+2} = \argmin \ensembleLigne{\KL{\Qbb}{\Qbb^{2n+1}}}{\Qbb \in \Pens(\rmc(\ccint{0,T}, \M)), \Qbb_0 = \pizero} .
\end{align}
Under mild assumptions on $\Pbb$, $\pizero$ and $\piinv$, we have that
$(\Qbb^n)_{n \in \nset}$ converges towards $\Qbb^\star$ \cite[see][]{nutz2022stability}.
In what follows, we propose an algorithm to
approximately sample from $(\Qbb^n)_{n \in \nset}$. In Euclidean state spaces,
\cite{debortoli2021neurips,vargas2021solving,chen2021likelihood} have proposed
an algorithm based on time-reversal to compute the IPF. We now extend these
techniques to the case of Riemannian manifolds.

\paragraph{Riemannian Diffusion Schr\"odinger Bridge}

We propose Riemannian Diffusion Schr\"odinger Bridge (RDSB) an extension of
Diffusion Schr\"odinger Bridge \cite{debortoli2021neurips} to approximate
solutions of SB problems. First, we connect the iterates
$(\Qbb^n)_{n \in \nset}$ with diffusion processes on $\M$.

\begin{proposition}
  \label{prop:continuous_schro}
  Let $\Pbb$ be the path measure of the Brownian motion initialized at $\piinv$.
  Assume that for any $n \in \nset$, $\KL{\Qbb^n}{\Pbb}< +\infty$ and that for
  any $t \in \ccint{0,T}$ and $n \in \nset$, $\Qbb^n_t$ admits a smooth positive
  density w.r.t.\ $\piinv$. Then, for any $n \in \nset$ we have:
  \begin{enumerate}[wide, labelwidth=!, labelindent=0pt, label=(\alph*)]
  \item $R(\Qbb^{2n+1})$ solves the martingale problem with generator $\generator^{2n+1}(t,u) = \langle \nabla u, b_{T-t}^n \rangle + (1/2) \Delta u$;
  \item $\Qbb^{2n+2}$ solves the martingale problem with generator $\generator^{2n+2}(t,u) = \langle \nabla u, f_{t}^{n+1} \rangle + (1/2) \Delta u$;    
  \end{enumerate}
  where for any $n \in \nset$, $t \in \ccint{0,T}$ and 
  $x \in \rset^d$, $b^{n}_t( x) = -f^{n}_t(x) + \nabla \log p^{n}_t(x)$, 
  $f^{n+1}_t(x) = -b^n_t(x) + \nabla \log q^n_t(x)$, with $f^0_t(x) = 0$, and $p^n_t$, $q_t^n$
  the densities of $\Qbb^{2n}_t$ and  $\Qbb_t^{2n+1}$.
\end{proposition}

\begin{proof}
  The proof is similar to \citet[Proposition 6]{debortoli2021neurips} using
  \Cref{thm:time_reversal_manifold} instead of \citet[Theorem
  4.19]{cattiaux2021time}
\end{proof}

In particular, we have that $\Qbb^1$ is the diffusion process associated with
RSGM, \ie \ the time-reversal of the Brownian motion initialized at
$\piinv$. Hence, $\Qbb^{2n+1}$ for $n \in \nset$ with $n \geq 1$ can be seen as
a refinement of $\Qbb^1$. In the next proposition, we show that the drift term
of the diffusion processes associated with $(\Qbb^n)_{n \in \nset}$ can be
approximated leveraging score-based techniques.

\begin{proposition}
  \label{prop:loss_implicit_explicit}
  Let $(\bfX_t)_{t \in \ccint{0,T}}$ be a $\M$-valued process with distribution
  $\Pbb \in \Pens(\rmc(\ccint{0,T}, \M))$ such that for any $t \in \ccint{0,T}$,
  $\bfX_t$ admits a positive density $p_t \in \rmc^\infty(\M)$
  w.r.t.\ $\piinv$. Let $s: \ \ccint{0,T} \to \XM$. For any $t \in \ccint{0,T}$
  and $x \in \M$, let
  \begin{equation}
    r(t,x) = -s(t,x) + \nabla \log p_t(x) . 
  \end{equation}
  Then, for any $t \in \ccint{0,T}$, we have that
  \begin{equation}
    r(t, \cdot) = \argmin \ensembleLigne{\expeLigne{(1/2)\normLigne{s(t, \bfX_t) + r(\bfX_t)}^2 + \dive(r)(\bfX_t)}}{r \in \mathrm{L}^2(\Pbb_t)} . 
  \end{equation}
\end{proposition}

\begin{proof}
  Let $t \in \ccint{0,T}$. First, we have for any $x \in \M$
  \begin{align}
    &\normLigne{r(t,x) - \{-s(t,x) + \nabla \log p_t(x) \}}^2\\
    & \qquad = \normLigne{r(t,x) + s(t, x)}^2 -2 \langle r(t,x), \nabla \log p_t(x) \rangle + \normLigne{\nabla \log p_t(x)}^2 - 2 \langle s(t,x), \nabla \log p_t(x) \rangle . 
  \end{align}
  Hence, we get that
  $r(t, \cdot) = \argmin \ensembleLigne{\expeLigne{\norm{s(t, \bfX_t) +
        r(\bfX_t)}^2 - 2 \langle r(\bfX_t), \nabla \log p_t(\bfX_t) \rangle}}{r
    \in \XM}$.
Using the
      divergence theorem \cite[see][p.51]{lee2018introduction}, we have for any $r \in \XM$
      \begin{align}
        \expeLigne{\langle r(\bfX_t), \nabla \log p_t(\bfX_t) \rangle} &= \textstyle{\int_\M \langle r(x_t), \nabla \log p_t(x_t) \rangle p_t(x_t) \rmd \piinv(x_t)} \\
                                                                       &= - \textstyle{\int_\M \dive(r)(x_t)  p_t(x_t) \rmd \piinv(x_t) = -\expeLigne{\dive(r)(\bfX_t)}}  ,
      \end{align}
which concludes the proof.  
\end{proof}



% Once we have defined general score-based generative moedls on compact Riemannian
% manifolds, these models can be used as the basis for several extensions. We list
% two of them: conditional sampling and Schr\"odinger bridge. 

% \paragraph{Conditional sampling} We first consider inverse problems on the
% manifold $\M$. Namely, given an observation $y$, we aim at recovering the
% initial signal $x \in \M$. Inverse problems on manifolds are ubiquitous in




%%% Local Variables:
%%% mode: latex
%%% TeX-master: "main"
%%% End:


\section{Proof of \cref{prop:implicit_der}}
\label{sec:implicit-losses}



\begin{proof}
      Let $t \in \ocint{0,T}$ and $s_t \in \rmc^\infty(\M)$. Using the
      divergence theorem \cite[see][p.51]{lee2018introduction}, we have
      \begin{align}
        \ell_{t|s}(s_t) &\textstyle{= \int_{\M \times \M} \normLigne{\nabla \log p_{t|s}(x_t|x_s)}^2 \rmd \Pbb_{s,t}(x_s,x_t) + \int_\M \normLigne{s_t(x_t)}^2 \rmd \Pbb_{t}(x_t)} \\
        & \qquad \qquad \textstyle{- 2 \int_{\M \times \M} \langle \nabla \log p_{t|s}(x_t|x_s), s_t(x_t) \rangle \rmd \Pbb_{s,t}(x_s,x_t)} \\
                    &=\textstyle{\int_{\M \times \M} \normLigne{\nabla \log p_{t|s}(x_t|x_s)}^2 \rmd \Pbb_{s,t}(x_s,x_t) + \int_\M \normLigne{s_t(x_t)}^2 \rmd \Pbb_{t}(x_t)} \\
                    & \qquad \qquad \textstyle{- 2 \int_{\M \times \M}  \langle \nabla \log p_{t|s}(x_t|x_s), s_t(x_t) \rangle p_{t|s}(x_t|x_s)p_s(x_s) \rmd (\piinv \otimes \piinv) (x_s, x_t)  } \\
        &=\textstyle{\int_{\M \times \M} \normLigne{\nabla \log p_{t|s}(x_t|x_s)}^2 \rmd \Pbb_{s,t}(x_s,x_t) + \int_\M \normLigne{s_t(x_t)}^2 \rmd \Pbb_{t}(x_t)} \\
                    & \qquad \qquad \textstyle{- 2 \int_{\M } \{\int_\M  \langle \nabla p_{t|s}(x_t|x_s), s_t(x_t) \rangle \rmd \piinv(x_t)\} p_s(x_s) \rmd \piinv(x_s)  } \\
        &=\textstyle{\int_{\M \times \M} \normLigne{\nabla \log p_{t|s}(x_t|x_s)}^2 \rmd \Pbb_{s,t}(x_s,x_t) + \int_\M \normLigne{s_t(x_t)}^2 \rmd \Pbb_{t}(x_t)} \\
                    & \qquad \qquad \textstyle{ +2 \int_{\M } \{\int_\M   \dive(s_t)(x_t)p_{t|s}(x_t|x_s)  \rmd \piinv(x_t)\} p_s(x_s) \rmd \piinv(x_s)  }  ,
      \end{align}


      which concludes the proof.
    \end{proof}

%%% Local Variables:
%%% mode: latex
%%% TeX-master: "main"
%%% End:


\section{Experimental detail}
\label{sec:exp_detail}

In what follows we describe the experimental settings used to generate results introduced in \cref{sec:experiments}.

% say we use Jax and geomstats?
% plan on open sourcing?

\paragraph{Architecture}
The architecture of the score network $s_\theta$ is given by a multilayer perceptron with 5 hidden layers with $512$ units each.
We use on sinusoidal activation functions.
% set of divergence free for generating the vector field.

% \paragraph{Loss}
% slide score matching with 1 sample for the Hutchinson estimator.

\paragraph{Optimization}
All models are trained by the stochastic optimizer Adam \citep{kingma2015Adam}
with parameters $\beta_1=0.9$, $\beta_2=0.999$, batch-size of $512$ data-points and a learning rate set to $2e-4$.
% number of iterations
% annealing

\paragraph{Likelihood evaluation}
We rely on the Dormand-Prince solver \citep{dormand1980family}, an adaptive Runge-Kutta 4(5) solver, with absolute and relative tolerance of $1e-5$ to compute approximate numerical solutions of the ODE.
% Models are trained on a cluster of GeForce RTX 2080 Ti GPU cards.

\end{document}

%%% Local Variables:
%%% mode: latex
%%% TeX-master: t
%%% End:
