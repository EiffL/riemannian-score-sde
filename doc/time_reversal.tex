\section{Time-reversal formula: extension to compact Riemannian manifolds}
\label{sec:time-reversal}

In this section, we  provide the proof of
\cref{thm:time_reversal_manifold}.  The proof follows the arguments of
\citet[Theorem 4.9]{cattiaux2021time}. We could have also applied the abstract
results of \citet[Theorem 5.7]{cattiaux2021time} to obtain our results. Note that
the time-reversal on manifold could also be obtained by readily extending
arguments from \citet{haussmann1986time}, however the entropic conditions found
by \citet{cattiaux2021time} are more natural when it comes to the study of the
Schr\"odinger Bridge problem. For the interested reader we provide an informal
derivation of the time-reversal formula obtained by \citet{haussmann1986time} in
\cref{sec:informal-derivation}. The proof of \cref{thm:time_reversal_manifold}
is given in \cref{sec:proof-crefthm:t}. Finally, we emphasize that
\citet{garcia2021brenier} develops a Girsanov theory for stochastic processes
defined on compact manifolds with boundary in order to study the
Brenier-Schr\"odinger problem.

\subsection{Informal derivation}
\label{sec:informal-derivation}

In this section, we provide a non-rigorous derivation of
\cref{thm:time_reversal_manifold} following the approach of
\citet{haussmann1986time}. Let $(\bfX_t)_{t \in \ccint{0,T}}$ be a continuous
process such that for any $f \in \rmc^2(\M)$ we have that
$(\bfM_t^{\bfX, f})_{t \in \ccint{0,T}}$ is a $\bfX$-martingale where for any
$t \in \ccint{0,T}$
  \begin{equation}
    \label{eq:martingale_forward}
    \textstyle{ \bfM_t^{\bfX, f} = f(\bfX_t) - \int_0^t \{ \langle b(\bfX_s), \nabla f(\bfX_s) \rangle_\M + (1/2) \Delta f(\bfX_s) \} \rmd s  . }
  \end{equation}
  Let $(\bfY_t)_{t \in \ccint{0,T}} = (\bfX_{T-t})_{t \in \ccint{0,T}}$. Our goal is to show that for any $f \in \rmc^2(\M)$, 
  $(\bfM_t^{\bfY, f})_{t \in \ccint{0,T}}$ is a $\bfY$-martingale where for any
  $t \in \ccint{0,T}$
  \begin{equation}
    \textstyle{ \bfM_t^{\bfY, f} = f(\bfY_t) - \int_0^t \{ \langle b(\bfY_s) + \nabla \log p_{T-s}(\bfY_s), \nabla f(\bfY_s) \rangle_\M + (1/2) \Delta f(\bfY_s) \} \rmd s  . }
  \end{equation}
  Note that here we implicitly assume that for any $t \in \ccint{0,T}$, $\bfX_t$
  admits a smooth positive density w.r.t. $\piinv$ denoted $p_t$.  In other
  words, we want to show that for any $g \in \rmc^2(\M)$ and
  $s, t \in \ccint{0,T}$ with $t \geq s$ we have
  \begin{equation}
    \label{eq:time_reversal_manifold_haussman}    
    \textstyle{\expeLigne{g(\bfY_s)(f(\bfY_t) - f(\bfY_s))} = \expeLigne{g(\bfY_s)\int_s^t \{ \langle b(\bfY_u) + \nabla \log p_{T-u}(\bfY_u), \nabla f(\bfY_u) \rangle_\M + (1/2) \Delta f(\bfY_u) \} \rmd u}  . }
  \end{equation}
  We introduce the infinitesimal generator
  $\generator: \  \rmc^2(\M) \to \rmc(\M)$ given for any $f \in \rmc^2(\M)$ and $x \in \M$ by
  \begin{equation}
    \generator (f)(x) = \langle b(x) , \nabla f(x) \rangle_\M + (1/2) \Delta f(x)  . 
  \end{equation}
  Similarly, we introduce the infinitesimal generator
  $\generatort: \  \ccint{0,T} \times \rmc^2(\M) \to \rmc(\M)$ given for any $f \in \rmc^2(\M)$, $t \in \ccint{0,T}$ and $x \in \M$ by
  \begin{equation}
    \generatort (t, f)(x) = \langle b(x) + \nabla \log p_{T-t}(x), \nabla f(x) \rangle_\M + (1/2) \Delta f(x)  . 
  \end{equation}
  With these notations, \eqref{eq:time_reversal_manifold_haussman} can be written as follows:  we want to show that for any $g \in \rmc^2(\M)$ and
  $s, t \in \ccint{0,T}$ with $t \geq s$ we have 
  \begin{equation}
    \label{eq:time_reversal_manifold_haussman}    
    \textstyle{\expeLigne{g(\bfY_s)(f(\bfY_t) - f(\bfY_s))} = \expeLigne{g(\bfY_s)\int_s^t \generatort(u, \bfY_u) \rmd u}  . }
  \end{equation}
  The rest of this section follows the first part of the proof of \citet[Theorem 2.1]{haussmann1986time}.
  Let $t, s \in \ccint{0,T}$ with $t \geq s$. We have
  \begin{align}
    \textstyle{\expeLigne{g(\bfY_s)(f(\bfY_t) - f(\bfY_s))}} &= \textstyle{\expeLigne{g(\bfX_{T-s})(f(\bfX_{T-t}) - f(\bfX_{T-t}))}} \\
                                                             &= \textstyle{\expeLigne{\CPELigne{g(\bfX_{T-s})}{\bfX_{T-t}}f(\bfX_{T-t})} - \expeLigne{g(\bfX_{T-s})f(\bfX_{T-s})}} \\
                                                             &= \textstyle{\expeLigne{v(T-t,\bfX_{T-t})f(\bfX_{T-t})} - \expeLigne{v(T-s,\bfX_{T-s})f(\bfX_{T-s})}}  ,
                                                               \label{eq:first_der}
  \end{align}
  with $v: \ \ccint{0,T-s} \times \M \to \rset$ given for any $u \in \ccint{0,T-s}$
  and $x \in \M$ by $v(u,x) = \CPELigne{g(\bfX_{T-s})}{\bfX_u=x}$. We have that $v$
  satisfies the backward Kolmogorov equation, i.e. we have for any
  $u \in \ccint{0,T-s}$ and $x \in \M$
  \begin{equation}
    \label{eq:backward_kolmogorov}
    \partial_u v(u,x) = -\generator v(u,x) . 
  \end{equation}
  Note that it is not trivial to show that $v$ is regular enough to satisfy the
  backward Kolmogorov equation. In this informal derivation, we assume that $v$
  is regular enough and will provide a different rigorous proof of the
  time-reversal formula in \cref{sec:proof-crefthm:t}. However, note that it is
  possible to show that $v$ indeed satisfies the backward Kolmogorov equation by
  adapting arguments from \citet{haussmann1986time} to the manifold framework.

  Let $h: \ \ccint{0,T-s} \times \M \to \rset$ given for any
  $u \in \ccint{0,T-s}$ and $x \in \M$ by $h(u,x) = v(u,x) f(x)$. Using
  \eqref{eq:backward_kolmogorov}, we have for any $u \in \ccint{0,T-s}$ and
  $x \in \M$
  \begin{align}
    \label{eq:def_h}
    \partial_u h(u,x) + \generator h(u, x) &= f(x) \partial_u v(u,x)  + f(x) \generator v(u,x) + v(u,x) \generator f(x) +  \langle \nabla f(x), \nabla v(u,x)\rangle \\
    &=  v(u,x) \generator f(x) + \langle \nabla f(x), \nabla v(u,x)\rangle_\M  . 
  \end{align}
  In addition, using the
      divergence theorem \citep[see][p.51]{lee2018introduction}, we have for any $u \in \ccint{0,T-s}$
  \begin{align}
    &\expeLigne{\langle \nabla f(\bfX_u), \nabla v(u,\bfX_u)\rangle_\M} = \textstyle{\int_{\M} \langle \nabla f(x_u), \nabla v(u,x_u) p_u(x_u) \rangle_\M \rmd \piinv(x_u) } \\
                                                                    & \qquad \qquad = - \textstyle{\int_{\M} v(u,x_u) \dive(p_u \nabla f) (x_u) \rmd \piinv(x_u) } \\
                                                                    & \qquad \qquad = - \textstyle{\int_{\M} v(u,x_u) \Delta f(x_u) p_u(x_u) \rmd \piinv(x_u) - \int_{\M} v(u,x_u) \langle \nabla f(x_u), \nabla \log p_u(x_u) \rangle_\M p_u(x_u) \rmd \piinv(x_u) } \\
                                                                    & \qquad \qquad = - \textstyle{\expeLigne{ v(u,\bfX_u) \Delta f(\bfX_u)}  - \expeLigne{ v(u,\bfX_u) \langle \nabla f(\bfX_u), \nabla \log p_u(\bfX_u) \rangle_\M} }  .
  \end{align}
  Therefore, using this result and \eqref{eq:def_h} we get that for any
  $u \in \ccint{0,T-s}$
  \begin{align}
    \expeLigne{\partial_u h(u,\bfX_u) + \generator h(u, \bfX_u)} &= \expeLigne{v(u,\bfX_u)\{ \langle b(\bfX_u) - \nabla \log p_u(\bfX_u), \nabla f(\bfX_u) \rangle_\M -(1/2) \Delta f(\bfX_u)\}} \\
    &= -\expeLigne{v(u,\bfX_u)\generatort(T-u,f)(\bfX_u)}  . 
  \end{align}
  Combining this result and \eqref{eq:martingale_forward} and that for any
  $u \in \ccint{0,T-s}$ and $x \in \M$,
  $v(u,x) = \CPELigne{g(\bfX_{T-s})}{\bfX_u=x}$ we get
  \begin{align}
    \expeLigne{v(T-t,\bfX_{T-t})f(\bfX_{T-t})} - \expeLigne{v(T-s,\bfX_{T-s})f(\bfX_{T-s})} &= \expeLigne{h(T-t, \bfX_{T-t}) - h(T-s, \bfX_{T-s})} \\
                                                                                            &= \textstyle{\int_{T-t}^{T-s} \expeLigne{v(u,\bfX_u)\generatort(T-u, \bfX_u)} \rmd u } \\
    &= \textstyle{\expeLigne{g(\bfX_{T-s})\int_{T-t}^{T-s} \generatort(T-u, \bfX_u) \rmd u } .}\\
  \end{align}
  Using this result, \eqref{eq:first_der} and the change of variable $u \mapsto T-u$ we obtain 
  \begin{equation}
    \expeLigne{g(\bfY_s)(f(\bfY_t) - f(\bfY_s))} = \textstyle{\expeLigne{g(\bfX_{T-s})\int_{T-t}^{T-s} \generatort(u, \bfX_u) \rmd u } } = \textstyle{\expeLigne{g(\bfY_{s})\int_{s}^{t} \generatort(u, \bfY_u) \rmd u } }  .
  \end{equation}
  Hence, \eqref{eq:time_reversal_manifold_haussman} holds and we have proved
  \cref{thm:time_reversal_manifold}. Again, we emphasize that in order to make
  the proof completely rigourous one needs to derive regularity properties of $v$.

  
\subsection{Proof of \cref{thm:time_reversal_manifold}}
\label{sec:proof-crefthm:t}

In this section, we follow another approach to prove the time-reversal
formula. We are going to use the integration by part formula of \citet[Theorem
3.17]{cattiaux2021time} in a similar spirit as \citet[Theorem
4.9]{cattiaux2021time} in the Euclidean setting. In order to adapt arguments
from \citet{cattiaux2021time} to our Riemannian setting, we use the Nash
embedding theorem in order to embed our processes in a Euclidean space and
leverage tools from Girsanov theory. The rest of the section is organized as
follows. First in \cref{sec:diff-proc-stoch}, we recall basic properties of
infinitesimal generators and recall the integration by part formula of
\citet[Theorem 3.17]{cattiaux2021time}. Then in \cref{sec:girs-theory-comp}, we
extend some Girsanov theory to compact Riemannian manifolds using the Nash
embedding theorem. We conclude the proof in \cref{sec:concluding-proof}.

\subsubsection{Diffusion processes and integration by part formula}
\label{sec:diff-proc-stoch}

In this section, we state a simplified version of \citet[Theorem
3.17]{cattiaux2021time} for Markov continuous path (probability) measure on
Polish spaces. Let $(\msx, \mcx)$ be a Polish space. We say that $\Pbb$ is a
path measure if $\Pbb \in \Pens(\rmc(\ccint{0,T}, \msx))$. Let
$(\bfX_t)_{t \in \ccint{0,T}}$ with distribution $\Pbb$. We denote
$(\mcf_t)_{t \in \ccint{0,T}}$ the filtration such that for any
$t \in \ccint{0,T}$, $\mcf_t = \sigma(\bfX_s, \ s \in \ccint{0,t})$. Let
$(\bfM_t)_{t \in \ccint{0,T}}$ be a Polish-valued stochastic process. We say that
$(\bfM_t)_{t \in \ccint{0,T}}$ is a $\Pbb$-local martingale if it is a local
martingale w.r.t. the filtration $(\mcf_t)_{t \in \ccint{0,T}}$. A function
$u: \ \ccint{0,T} \times \msx \to \rset$ is said to be in the domain of the
extended generator of $\Pbb$ if there exists a process
$(\generatorb_\Pbb u(t, \bfX_{\ccint{0,t}}))_{t \in \ccint{0,T}}$ such that:
\begin{enumerate}[label= (\alph*),  wide, labelwidth=!, labelindent=0pt]
\item $(\generatorb_\Pbb u(t, \bfX_{\ccint{0,t}}))_{t \in \ccint{0,T}}$ is adapted w.r.t. $(\mcf_t)_{t \in \ccint{0,T}}$.
\item $\int_0^T \absLigne{\generatorb_\Pbb u(t, \bfX_{\ccint{0,t}})} \rmd t < +\infty$, $\Pbb$-a.s.
\item The process $(\bfM_t)_{t \in \ccint{0,T}}$ is a $\Pbb$-local martingale,
  where for any $t \in \ccint{0,T}$
  \begin{equation}
    \textstyle{\bfM_t = u(t,\bfX_t) - u(0, \bfX_0) - \int_0^t \generatorb_\Pbb u(s, \bfX_{\ccint{0,s}}) \rmd s   .}
  \end{equation}
\end{enumerate}
The domain of the extended generator is denoted $\dom(\generatorb_\Pbb)$. We say
that $(u,v)$ with $u,v : \ \ccint{0,T} \times \msx \to \rset$ is in the domain
of the carr\'e du champ if $u,v, uv \in \dom(\generatorb_\Pbb)$. In this case, we
define the carr\'e du champ $\carrechampb_\Pbb$ as
\begin{equation}
  \carrechampb_\Pbb(u,v) = \generatorb_\Pbb(uv) - \generatorb_\Pbb(u)v - \generatorb_\Pbb(v)u  . 
\end{equation}
Note that if $\msx = \M$ is a Riemannian manifold,
$\rmc^2(\M) \subset \dom(\generatorb_\Pbb)$ and for any $u \in \rmc^2(\M)$
$\generatorb_\Pbb(u) = \langle \nabla u, X \rangle_\M + (1/2)\Delta u$ with
$X \in \Gamma(\TM)$  then we have that $\rmc^2(\M) \times \rmc^2(\M) \subset \dom(\carrechampb_\Pbb)$
and for any $u, v \in \rmc^2(\M)$,
$\carrechampb_\Pbb(u,v) = \langle \nabla u, \nabla v \rangle_\M$. Assume that there exists
$\mathcal{U}_\Pbb \subset \dom(\generatorb_\Pbb) \cap \rmc_b(\msx)$ such that
$\mathcal{U}_\Pbb$ is an algebra. We denote $\mathcal{U}_{\Pbb,2}$ such that
\begin{equation}
  \mathcal{U}_{\Pbb,2} = \ensembleLigne{u \in \mathcal{U}_\Pbb}{\generatorb_\Pbb u \in \mathrm{L}^2(\Pbb), \ \carrechampb_\Pbb(u,u) \in \mathrm{L}^1(\Pbb)}  . 
\end{equation}
Finally we denote $R(\Pbb)$ the time-reverse path measure, i.e. for any
$\msa \in \mcb{\rmc(\ccint{0,T}, \msx)}$ we have
$R(\Pbb)(\msa) = \Pbb(R(\msa))$, where
$R(\msa) = \ensembleLigne{t \mapsto \omega_{T-t}}{\omega \in \msa}$.  In what
follows, we assume $\Pbb$ is Markov. It is well-known, see \citep[Theorem
1.2]{leonard2014reciprocal} for instance, that in this case $R(\Pbb)$ is also
Markov. In addition, since $\Pbb$ is Markov, for any $u \in \mathrm{dom}(\generatorb_\Pbb)$ and
$t \in \ccint{0,T}$ there exists $\generator_\Pbb$ such that
$\generatorb_\Pbb u(t, \bfX_{\ccint{0,t}}) = \generator_\Pbb u(t, \bfX_t)$ with
$\generator_\Pbb u: \ \ccint{0,T} \times \msx \to \rset$. Similarly, we define
$\carrechamp_\Pbb(u,v): \ \ccint{0,T} \times \msx \to \rset$ from $\carrechampb_\Pbb(u,v)$.

We are now ready to state the integration by part formula,
\citep[Theorem 3.17]{cattiaux2021time}. 

\begin{theorem}
  \label{thm:ibp_cattiaux}
  Let $u, v \in \mathcal{U}_{\Pbb, 2}$. The following hold:
  \begin{enumerate}[label= (\alph*),  wide, labelwidth=!, labelindent=0pt]    
  \item If
  $u \in \dom(\generator_{R(\Pbb)})$ and
  $\generator_{R(\Pbb)}u \in \mathrm{L}^1(\Pbb)$ then for almost any $t \in \ccint{0,T}$
  \begin{equation}
    \expeLigne{\{\generator_\Pbb u(t, \bfX_t) + \generator_{R(\Pbb)} u (T-t, \bfX_t)\}v(\bfX_t) + \carrechamp_\Pbb(u,v)(t, \bfX_t)} = 0  .       
  \end{equation}  
\item If the following hold:
  \begin{enumerate}[label=\roman*)]
  \item $\carrechamp_\Pbb(u,v) \in \rmc(\ccint{0,T} \times \msx, \rset)$.
  \item $\mathcal{U}_{2, \Pbb}$ determines the weak convergence of Borel measure.
  \item $\mu$ defines a finite measure on $\ccint{0,T} \times \msx$ where for
    any $\omega \in \bar{\mathcal{U}}_{2, \Pbb}$ we have
    \begin{equation}
      \textstyle{\mu[\omega] = \expeLigne{\int_0^T \carrechamp_\Pbb(u,\omega_t)(t, \bfX_t) \rmd t  ,}}
    \end{equation}
    where
    $\bar{\mathcal{U}}_{2, \Pbb} = \ensembleLigne{\omega \in \rmc(\ccint{0,T}
      \times \msx, \rset)}{\omega(t, \cdot) \in \mathcal{U}_{2, \Pbb}\ \
      \text{for any $t \in \ccint{0,T}$}}$.
  \end{enumerate}
  Then $u \in \dom(\generator_{R(\Pbb)})$ and
  $\generator_{R(\Pbb)}u \in \mathrm{L}^1(\Pbb)$.
  \end{enumerate}
\end{theorem}

Note that this theorem is a simplified version of \citet[Theorem
3.17]{cattiaux2021time} where we restrict ourselves to the case of Markov path
measures. In what follows, we wish to apply \cref{thm:ibp_cattiaux} to diffusion
processes on manifolds. To do so, we will verify that under a finite entropy
assumption, the conditions $u \in \dom(\generator_{R(\Pbb)})$ and
$\generator_{R(\Pbb)}u \in \mathrm{L}^1(\Pbb)$ are fullfilled for a class of
regular functions $u$. These integrability results are obtained using Girsanov
theory.

\subsubsection{Girsanov theory on compact Riemannian manifolds}
\label{sec:girs-theory-comp}

In this section, we will consider two types of martingale problems: one on
Euclidean spaces and one on the compact Riemannian manifold $\M$. Let
$\Pbb \in \Pens(\rmc(\ccint{0,T}, \rset^p))$. We say that $\Pbb$ satisfies the
(Euclidean) martingale problem with infinitesimal generator
$\generator: \ \ccint{0,T} \times \rmc^2(\rset^p) \times \rset^p \to \rset$ if
for any $u \in \rmc_c^2(\rset^p)$, $(\bfM_t)_{t \in \ccint{0,T}}$ is a
$\Pbb$-martingale where for any $t \in \ccint{0,T}$ we have
\begin{equation}
  \textstyle{
    \bfM_t = \bfM_0 + \int_0^t \generator(t, u)(\bfX_s) \rmd s  ,
    }
  \end{equation}
  where $(\bfX_t)_{t \in \ccint{0,T}}$ has distribution $\Pbb$ and
  $\int_0^T \absLigne{\generator(t, u)(\bfX_s) \rmd t} <+\infty$, $\Pbb$-a.s.
  Let $\Pbb \in \Pens(\rmc(\ccint{0,T}, \M))$. We say that $\Pbb$ satisfies the
  (Riemannian) martingale problem with infinitesimal generator
  $\generatort: \ \ccint{0,T} \times \rmc^2(\M) \times \M \to \rset$ if for any
  $u \in \rmc^2(\M)$, $(\bfM_t)_{t \in \ccint{0,T}}$ is a $\Pbb$-martingale
  where for any $t \in \ccint{0,T}$ we have
\begin{equation}
  \textstyle{
    \bfM_t = \bfM_0 + \int_0^t \generatort(t, u)(\bfX_s) \rmd s  ,
    }
  \end{equation}
  where $(\bfX_t)_{t \in \ccint{0,T}}$ has distribution $\Pbb$ and 
  $\int_0^T \absLigne{\generatort(t, u)(\bfX_s) \rmd t} <+\infty$, $\Pbb$-a.s.
  We now prove the following theorem.

  \begin{proposition}
    \label{prop:girsanov_manifold}
    Let $\Qbb$ be the path measure of a Brownian motion on $\M$. Let $\Pbb$ be a
    Markov path measure on $\rmc(\ccint{0,T}, \M)$ such that $\KL{\Pbb}{\Qbb} < +\infty$. Then there exists
    $\beta$ such that for any $t \in \ccint{0,T}$ and
    $x \in \M$, $\beta(t,x) \in \mathrm{T}_x \M$. In addition, we have that
    $\Pbb$ satisfies the martingale problem with infinitesimal generator
    $\generator$ where for any $t \in \ccint{0,T}$, $u \in \rmc^2(\M)$ and
    $x \in \M$ we have
    \begin{equation}
      \generator(t,u)(x) = \langle \beta(t,x), \nabla u(x) \rangle_\M + (1/2) \Delta u(x)  . 
    \end{equation}
    In addition, we have that
    \begin{equation}
      \textstyle{\KL{\Pbb}{\Qbb} = \KL{\Pbb_0}{\Qbb_0} + (1/2) \int_0^T \expeLigne{\norm{\beta(t, \bfX_t)}^2} \rmd t  ,}
    \end{equation}
    where $(\bfX_t)_{t \in \ccint{0,T}}$ has distribution $\Pbb$.
  \end{proposition}


  \begin{proof}
    First, we extend $(\bfB_t^\M)_{t \in \ccint{0,T}}$ to $\rset^p$ using the
    Nash embedding theorem \citep[see][]{gunther1991isometric}.
    $(\bfB_t^\M)_{t \in \ccint{0,T}}$ can be seen as a process on $\rset^p$ (for
    some $p \in \nset$) which satisfies in a weak sense
    \begin{equation}
      \textstyle{
        \rmd \bfB_t^\M = \sum_{i=1}^p P_i(\bfB_t^\M) \circ \rmd \bfB_t^i  = P(\bfB_t^\M) \circ \rmd \bfB_t  ,
        }
    \end{equation}
    where $(\bfB_t)_{t \in \ccint{0,T}}$ is a $p$-dimensional Brownian motion
    and $P \in \rmc^\infty(\rset^p, \rset^{p\times p})$ is such that for any
    $x \in \M$, $P(x)$ is the projection onto $\mathrm{T}_x \M$ and for any
    $i \in \{1, \dots, p\}$, $P_i \in \rmc^\infty(\rset^p, \rset^p)$ with
    $P_i = P e_i$ where $\{e_j\}_{j=1}^d$ is the canonical basis of $\rset^p$.
    Using the link between Stratanovitch and It\^o integral, there exists
    $\bar{b} \in \rmc^\infty(\rset^p, \rset^p)$ such that
    $(\bfB_t^\M)_{t \in \ccint{0,T}}$ can be seen as a process on $\rset^p$
    which satisfies in a weak sense
    \begin{equation}
      \textstyle{
        \rmd \bfB_t^\M = \bar{b}(\bfB_t^\M) \rmd t +  P(\bfB_t^\M)  \rmd \bfB_t  .
        }
      \end{equation}
      For any $u \in \rmc^2(\M)$, we consider $\bar{u}$ an extension to $\rmc^2_c(\rset^p)$ and we have for any $s, t \in \ccint{0,T}$
      \begin{align}
        &\textstyle{\expeLigne{\bar{v}(\bfB_s^\M) \int_s^t (1/2) \Delta u(\bfB_u^\M) \rmd u}} \\
        & \qquad =  \textstyle{\expeLigne{\bar{v}(\bfB_s^\M) \int_s^t \{ \langle \nabla \bar{u}(\bfB_u^\M), \bar{b}(\bfB_u^\M) \rangle + (1/2) \langle P(\bfB_u^\M), \nabla^2 \bar{u}(\bfB_u^\M) \rangle \} \rmd u}  . }
      \end{align}
      In particular, we get that for any $x \in \M$,
      $\Delta u(x) = 2 \langle \bar{u}(x), \bar{b}(x) \rangle + \Delta
      \bar{u}(x)$ \valentin{prove that for the projection this is okay}. Note
      that $(\bfB_t^\M)_{t \in \ccint{0,T}}$ (seen as a process on $\rset^p$)
      satisfies the condition $\mathrm{(U)}$ in
      \cite{leonard2012girsanov}. Therefore applying \cite[Theorem
      2.1]{leonard2012girsanov}, \citep[Claim 4.5]{cattiaux2021time}, there
      exists $\bar{\beta}: \ \ccint{0,T} \times \rset^p \to \rset^p$ such that
      \begin{equation}
        \label{eq:KL_ineq}
      \textstyle{\KL{\Pbb}{\Qbb} = \KL{\Pbb_0}{\Qbb_0} + (1/2) \int_0^T \expeLigne{\normLigne{P(\bfX_t) \bar{\beta}(t, \bfX_t)}^2} \rmd t  .}
    \end{equation}
    In addition, $\Pbb$ (seen as a process on $\rset^p$) satisfies a martingale
    problem with infinitesimal generator
    $\generatorb: \ \ccint{0,T} \times \rmc^2_c(\rset^p) \times \rset^p \to \rset$ such that
    for any $t \in \ccint{0,T}$, $\bar{u} \in \rmc^2(\rset^p)$ and $x \in \rset^p$
    \begin{equation}
      \generatorb(t,\bar{u})(x) = \langle \bar{b}(x) + P(x)\bar{\beta}(t,x), \nabla \bar{u}(x) \rangle + (1/2) \Delta \bar{u}(x)  . 
    \end{equation}
    Let $\beta: \ \ccint{0,T} \times \M$ such that for any $t \in \ccint{0,T}$
    and $x \in \M$ we have $\beta(t,x) = P(x) \bar{\beta}(t,x)$. In particular,
    we have that for any $x \in \M$, $\beta(t,x) \in \mathrm{T}_x\M$. Let
    $u \in \rmc^2(\M)$ \valentin{dire que c'est okay pour le delta et pour le
      gradient si on prend u bar = u circ p} and consider an extension $\bar{u}$
    to $\rmc^2(\rset^p)$. For any $t \in \ccint{0,T}$ and $x \in \M$ we have
    \begin{align}
      \generatorb(t,\bar{u})(x) &= \langle \bar{b}(x) + P(x)\bar{\beta}(t,x), \nabla \bar{u}(x) \rangle + (1/2) \Delta \bar{u}(x) \\
                               &= \langle  \beta(t,x), \nabla \bar{u}(x) \rangle + (1/2) \Delta u(x) \\
                               &= \langle P(x) \beta(t,x), P(x) \nabla \bar{u}(x) \rangle + (1/2) \Delta u(x) = \langle \beta(t,x), \nabla u(x) \rangle + (1/2) \Delta u(x)  . 
    \end{align}
    In particular, we have that $\Pbb$ (seen as a process on $\M$) satisfies a
    martingale with infinitesimal generator
    $\generatorb: \ \ccint{0,T} \times \rmc^2_c(\M) \times \M \to \rset$ such that
    for any $t \in \ccint{0,T}$, $u \in \rmc^2(\rset^p)$ and $x \in \M$
    \begin{equation}
      \generator(t,\bar{u})(x) = \langle \beta(t,x), \nabla u(x) \rangle_\M + (1/2) \Delta u(x)  . 
    \end{equation}
    In addition, rewriting \eqref{eq:KL_ineq} we have
      \begin{equation}
        \label{eq:KL_ineq}
      \textstyle{\KL{\Pbb}{\Qbb} = \KL{\Pbb_0}{\Qbb_0} + (1/2) \int_0^T \expeLigne{\normLigne{\beta(t, \bfX_t)}^2} \rmd t  ,}
    \end{equation}
    which concludes the proof.
  \end{proof}
  
  Once this proposition is established, we can obtain the following
  straightforward extension of \citet[Proposition 4.6]{cattiaux2021time}.

  \begin{proposition}
    \label{prop:hyp_317}
    Let $\Qbb$ be a Brownian motion with $\Qbb_0 = \piinv$ and $\Pbb$ a path
    measure on $\rmc(\ccint{0,T}, \M)$ such that $\KL{\Pbb}{\Qbb} <
    +\infty$. Then, there exist $\beta_\Pbb, \beta_{R(\Pbb)}: \ \ccint{0,T} \times \M \to $
    such that for any $t \in \ccint{0,T}$ and $x \in \M$,
    $\beta_\Pbb(t,x), \beta_{R(\Pbb)}(t,x) \in \mathrm{T}_x \M$. In addition, we have that
    $\Pbb$ and $R(\Pbb)$ satisfy martingale problems with infinitesimal generator
    $\generator_{\Pbb}$, respectively $\generator_{R(\Pbb)}$ where for any $t \in \ccint{0,T}$, $u \in \rmc^2(\M)$ and
    $x \in \M$ we have
    \begin{align}
      &\generator_{\Pbb}(t,u)(x) = \langle \beta_\Pbb(t,x), \nabla u(x) \rangle_\M + (1/2) \Delta u(x)  , \\
      &\generator_{R(\Pbb)}(t,u)(x) = \langle \beta_{R(\Pbb)}(t,x), \nabla u(x) \rangle_\M + (1/2) \Delta u(x)  . 
    \end{align}
    Finally, we have that
    \begin{equation}
      \textstyle{
        \int_0^T \expeLigne{\norm{\beta_\Pbb(t, \bfX_t)}^2} \rmd t + \int_0^T \expeLigne{\norm{\beta_{R(\Pbb)}(t, \bfX_{T-t})}^2} \rmd t < +\infty  ,
        }
    \end{equation}
    where $(\bfX_t)_{t \in \ccint{0,T}}$ has distribution $\Pbb$.
  \end{proposition}

  \begin{proof}
    The proof is straightforward upon combining \cref{prop:girsanov_manifold}
    and the fact that
    $\KL{\Pbb}{\Qbb} = \KL{R(\Pbb)}{R(\Qbb)} = \KL{R(\Pbb)}{\Qbb} < +\infty$,
    using that $\Qbb$ is stationary.
  \end{proof}

  We conclude this section, with the following application of \cref{thm:ibp_cattiaux}.

  \begin{proposition}
    \label{prop:cattiaux_spec}
    For any $u, v \in \rmc^\infty(\M)$, we have that for almost any $t \in \ccint{0,T}$
    \begin{equation}
      \label{eq:equalitu}
      \expeLigne{v(\bfX_t) \langle \beta_\Pbb(t, \bfX_t) + \beta_{R(\Pbb)}(T-t, \bfX_t), \nabla u(\bfX_t) \rangle_\M + \langle \nabla u(\bfX_t), \nabla v(\bfX_t) \rangle} = 0  . 
    \end{equation}
  \end{proposition}

  \begin{proof}
  Remark that $\rmc^2(\M) \subset \dom(\carrechamp_\Pbb)$ and
  $\rmc^2(\M) \subset \dom(\carrechamp_{R(\Pbb)})$. In addition, we have that for any
  $u,v \in \rmc^2(\M)$,
  $\carrechamp_\Pbb(u,v) = \carrechamp_{R(\Pbb)}(u,v) = \langle u, v \rangle$. Note that
  by \cref{prop:hyp_317} and \cref{thm:ibp_cattiaux} we immediately have that
  for any $u, v \in \rmc^\infty(\M)$, \eqref{eq:equalitu} holds.    
  \end{proof}
\subsubsection{Concluding the proof}
\label{sec:concluding-proof}

Using \cref{prop:cattiaux_spec} we can now conclude the proof of \cref{thm:time_reversal_manifold}.
First, remark that we can identify $\beta_\Pbb = b$. Let $u, v \in \rmc^\infty(\M)$, we have that 
    \begin{equation}
      \label{eq:equality_fin}
      \expeLigne{v(\bfX_t) \langle b(\bfX_t) + \beta_{R(\Pbb)}(T-t, \bfX_t), \nabla u(\bfX_t) \rangle + \Delta u(\bfX_t) v(\bfX_t)+ \langle \nabla u(\bfX_t), \nabla v(\bfX_t) \rangle} = 0  . 
    \end{equation}
    Using that for any $t \in \ccint{0,T}$, $\Pbb_t$ admits a smooth positive
    density w.r.t. $\piinv$ denoted $p_t$ and the divergence theorem, see
    \citep[p.51]{lee2018introduction}, we have that for any $t \in \ccint{0,T}$,
\begin{align}
  &    \textstyle{\int_{\M} \{ \langle \beta_{R(\Pbb)}(T-t, x), \nabla u(x) \rangle + \langle b(x), \nabla u(x) \rangle \} v(x) p_t(x) \rmd \piinv(x)} \\
    & \qquad \qquad \qquad \qquad = \textstyle{\int_\M \langle \nabla u(x) p_t(x), \nabla v(x) \rangle \rmd \piinv(x) } \\
   & \qquad \qquad \qquad \qquad = - \textstyle{\int_\M \{ \Delta u (x) + \langle \nabla \log p_t(x), \nabla u(x) \rangle \} v(x) p_t(x)\rmd \piinv(x) }  . 
\end{align}
Therefore, we get that for any $t \in \ccint{0,T}$ and $x \in \M$,
$\langle \beta_{R(\Pbb)}(T-t, x), \nabla u(x) \rangle = \langle 
-b(x) + \nabla\log p_t(x), \nabla u(x) \rangle$, which concludes the proof.

    
%%% Local Variables:
%%% mode: latex
%%% TeX-master: "main"
%%% End:
